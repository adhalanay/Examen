\documentclass{article}
\usepackage[T1]{fontenc}
\usepackage[utf8]{inputenc}
\usepackage{lmodern}
\usepackage{nopageno}
\usepackage{fontspec}
\usepackage{amsmath}
\usepackage{amsfonts}
\usepackage{amssymb}
\title{Probleme}
\author{ }
\date{ }

\begin{document}
\begin{flushright}
Nume:\_\_\_\_\_\_\_\_\_\_\_\_\_\_
 
 
Grupa:\_\_\_\_\_\_\_\_\_\_\_\_\_\_
\end{flushright}
\begin{center}
\vspace{2cm}
{\Large Probleme}
\vspace{2cm}
\end{center}
\begin{enumerate}
 \item Fie morfismul $f:\mathbb{R}^4 \to \mathbb{R}^3$ al cărui matrice în raport cu bazele canonice este
$$\begin{pmatrix}
0&-2&0&-2\\
-2&3&-3&3\\
-2&-2&-3&-2
\end{pmatrix}$$

\begin{enumerate}
\item Determinați cîte o bază în $Ker(f)$ și $Im(f)$;
\item Fie vectorul $v=(1,3,1,3)$ determinați descompunerea acestuia ca suma dintre un vector din $Ker(f)$ și unul din $(Ker(f))^\perp$;
\item Fie $K$ un corp și fie $L=M_n(K)$. Există $X,Y \in L$ astfel încît $XY-YX=I_n$?  
\end{enumerate}
\item Fie forma pătratică:
$$Q= 2x_1^2+5x_2^2+2x_3^2-4x_1x_2-2x_1x_3+4x_2x_3$$

\begin{enumerate}
\item Aduceți forma pătratică la forma canonică prin metoda Gauss;
\item Aduceți forma pătratică la forma canonică prin transformări ortogonale;
\item Fie $\mathfrak{su}(n)=\{ A \in M_n(\mathbb{C}) | A+\bar{A}^t=0\}$. Arătați că $\mathfrak{su}(n)$ este spațiu vectorial real, dar nu complex.
Determinați $dim_{\mathbb{R}}\mathfrak{su}(n)$.
\end{enumerate}
\end{enumerate}
\newpage
\begin{flushright}
Nume:\_\_\_\_\_\_\_\_\_\_\_\_\_\_
 
 
Grupa:\_\_\_\_\_\_\_\_\_\_\_\_\_\_
\end{flushright}
\begin{center}
\vspace{2cm}
{\Large Probleme}
\vspace{2cm}
\end{center}
\begin{enumerate}
 \item Fie morfismul $f:\mathbb{R}^4 \to \mathbb{R}^3$ al cărui matrice în raport cu bazele canonice este
$$\begin{pmatrix}
0&3&2&2\\
-3&-3&2&2\\
-1&-2&0&0
\end{pmatrix}$$

\begin{enumerate}
\item Determinați cîte o bază în $Ker(f)$ și $Im(f)$;
\item Fie vectorul $v=(1,3,1,3)$ determinați descompunerea acestuia ca suma dintre un vector din $Ker(f)$ și unul din $(Ker(f))^\perp$;
\item Fie $K$ un corp și fie $L=M_n(K)$. Există $X,Y \in L$ astfel încît $XY-YX=I_n$?  
\end{enumerate}
\item Fie forma pătratică:
$$Q= x_1^2+5x_2^2+x_3^2+2x_1x_2+6x_1x_3+2x_2x_3$$

\begin{enumerate}
\item Aduceți forma pătratică la forma canonică prin metoda Gauss;
\item Aduceți forma pătratică la forma canonică prin transformări ortogonale;
\item Fie $\mathfrak{su}(n)=\{ A \in M_n(\mathbb{C}) | A+\bar{A}^t=0\}$. Arătați că $\mathfrak{su}(n)$ este spațiu vectorial real, dar nu complex.
Determinați $dim_{\mathbb{R}}\mathfrak{su}(n)$.
\end{enumerate}
\end{enumerate}
\newpage
\begin{flushright}
Nume:\_\_\_\_\_\_\_\_\_\_\_\_\_\_
 
 
Grupa:\_\_\_\_\_\_\_\_\_\_\_\_\_\_
\end{flushright}
\begin{center}
\vspace{2cm}
{\Large Probleme}
\vspace{2cm}
\end{center}
\begin{enumerate}
 \item Fie morfismul $f:\mathbb{R}^4 \to \mathbb{R}^3$ al cărui matrice în raport cu bazele canonice este
$$\begin{pmatrix}
0&0&1&-1\\
-3&-2&-2&3\\
3&2&0&-1
\end{pmatrix}$$

\begin{enumerate}
\item Determinați cîte o bază în $Ker(f)$ și $Im(f)$;
\item Fie vectorul $v=(1,3,1,3)$ determinați descompunerea acestuia ca suma dintre un vector din $Ker(f)$ și unul din $(Ker(f))^\perp$;
\item Fie $K$ un corp și fie $L=M_n(K)$. Există $X,Y \in L$ astfel încît $XY-YX=I_n$?  
\end{enumerate}
\item Fie forma pătratică:
$$Q= x_1^2-2x_2^2+x_3^2+4x_1x_2-10x_1x_3+4x_2x_3$$

\begin{enumerate}
\item Aduceți forma pătratică la forma canonică prin metoda Gauss;
\item Aduceți forma pătratică la forma canonică prin transformări ortogonale;
\item Fie $\mathfrak{su}(n)=\{ A \in M_n(\mathbb{C}) | A+\bar{A}^t=0\}$. Arătați că $\mathfrak{su}(n)$ este spațiu vectorial real, dar nu complex.
Determinați $dim_{\mathbb{R}}\mathfrak{su}(n)$.
\end{enumerate}
\end{enumerate}
\newpage
\begin{flushright}
Nume:\_\_\_\_\_\_\_\_\_\_\_\_\_\_
 
 
Grupa:\_\_\_\_\_\_\_\_\_\_\_\_\_\_
\end{flushright}
\begin{center}
\vspace{2cm}
{\Large Probleme}
\vspace{2cm}
\end{center}
\begin{enumerate}
 \item Fie morfismul $f:\mathbb{R}^4 \to \mathbb{R}^3$ al cărui matrice în raport cu bazele canonice este
$$\begin{pmatrix}
0&0&1&-1\\
-1&1&-3&3\\
-1&1&3&-3
\end{pmatrix}$$

\begin{enumerate}
\item Determinați cîte o bază în $Ker(f)$ și $Im(f)$;
\item Fie vectorul $v=(1,3,1,3)$ determinați descompunerea acestuia ca suma dintre un vector din $Ker(f)$ și unul din $(Ker(f))^\perp$;
\item Fie $K$ un corp și fie $L=M_n(K)$. Există $X,Y \in L$ astfel încît $XY-YX=I_n$?  
\end{enumerate}
\item Fie forma pătratică:
$$Q= 2x_1^2+5x_2^2+2x_3^2-4x_1x_2-2x_1x_3+4x_2x_3$$

\begin{enumerate}
\item Aduceți forma pătratică la forma canonică prin metoda Gauss;
\item Aduceți forma pătratică la forma canonică prin transformări ortogonale;
\item Fie $\mathfrak{su}(n)=\{ A \in M_n(\mathbb{C}) | A+\bar{A}^t=0\}$. Arătați că $\mathfrak{su}(n)$ este spațiu vectorial real, dar nu complex.
Determinați $dim_{\mathbb{R}}\mathfrak{su}(n)$.
\end{enumerate}
\end{enumerate}
\newpage
\begin{flushright}
Nume:\_\_\_\_\_\_\_\_\_\_\_\_\_\_
 
 
Grupa:\_\_\_\_\_\_\_\_\_\_\_\_\_\_
\end{flushright}
\begin{center}
\vspace{2cm}
{\Large Probleme}
\vspace{2cm}
\end{center}
\begin{enumerate}
 \item Fie morfismul $f:\mathbb{R}^4 \to \mathbb{R}^3$ al cărui matrice în raport cu bazele canonice este
$$\begin{pmatrix}
2&-2&3&3\\
0&0&2&0\\
0&0&3&0
\end{pmatrix}$$

\begin{enumerate}
\item Determinați cîte o bază în $Ker(f)$ și $Im(f)$;
\item Fie vectorul $v=(1,3,1,3)$ determinați descompunerea acestuia ca suma dintre un vector din $Ker(f)$ și unul din $(Ker(f))^\perp$;
\item Fie $K$ un corp și fie $L=M_n(K)$. Există $X,Y \in L$ astfel încît $XY-YX=I_n$?  
\end{enumerate}
\item Fie forma pătratică:
$$Q= x_1^2+5x_2^2+x_3^2+2x_1x_2+6x_1x_3+2x_2x_3$$

\begin{enumerate}
\item Aduceți forma pătratică la forma canonică prin metoda Gauss;
\item Aduceți forma pătratică la forma canonică prin transformări ortogonale;
\item Fie $\mathfrak{su}(n)=\{ A \in M_n(\mathbb{C}) | A+\bar{A}^t=0\}$. Arătați că $\mathfrak{su}(n)$ este spațiu vectorial real, dar nu complex.
Determinați $dim_{\mathbb{R}}\mathfrak{su}(n)$.
\end{enumerate}
\end{enumerate}
\newpage
\begin{flushright}
Nume:\_\_\_\_\_\_\_\_\_\_\_\_\_\_
 
 
Grupa:\_\_\_\_\_\_\_\_\_\_\_\_\_\_
\end{flushright}
\begin{center}
\vspace{2cm}
{\Large Probleme}
\vspace{2cm}
\end{center}
\begin{enumerate}
 \item Fie morfismul $f:\mathbb{R}^4 \to \mathbb{R}^3$ al cărui matrice în raport cu bazele canonice este
$$\begin{pmatrix}
0&0&1&-1\\
1&1&0&-1\\
-3&-3&3&0
\end{pmatrix}$$

\begin{enumerate}
\item Determinați cîte o bază în $Ker(f)$ și $Im(f)$;
\item Fie vectorul $v=(1,3,1,3)$ determinați descompunerea acestuia ca suma dintre un vector din $Ker(f)$ și unul din $(Ker(f))^\perp$;
\item Fie $K$ un corp și fie $L=M_n(K)$. Există $X,Y \in L$ astfel încît $XY-YX=I_n$?  
\end{enumerate}
\item Fie forma pătratică:
$$Q= x_1^2-2x_2^2+x_3^2+4x_1x_2-10x_1x_3+4x_2x_3$$

\begin{enumerate}
\item Aduceți forma pătratică la forma canonică prin metoda Gauss;
\item Aduceți forma pătratică la forma canonică prin transformări ortogonale;
\item Fie $\mathfrak{su}(n)=\{ A \in M_n(\mathbb{C}) | A+\bar{A}^t=0\}$. Arătați că $\mathfrak{su}(n)$ este spațiu vectorial real, dar nu complex.
Determinați $dim_{\mathbb{R}}\mathfrak{su}(n)$.
\end{enumerate}
\end{enumerate}
\newpage
\begin{flushright}
Nume:\_\_\_\_\_\_\_\_\_\_\_\_\_\_
 
 
Grupa:\_\_\_\_\_\_\_\_\_\_\_\_\_\_
\end{flushright}
\begin{center}
\vspace{2cm}
{\Large Probleme}
\vspace{2cm}
\end{center}
\begin{enumerate}
 \item Fie morfismul $f:\mathbb{R}^4 \to \mathbb{R}^3$ al cărui matrice în raport cu bazele canonice este
$$\begin{pmatrix}
0&0&1&-1\\
3&-3&3&-3\\
0&0&-1&1
\end{pmatrix}$$

\begin{enumerate}
\item Determinați cîte o bază în $Ker(f)$ și $Im(f)$;
\item Fie vectorul $v=(1,3,1,3)$ determinați descompunerea acestuia ca suma dintre un vector din $Ker(f)$ și unul din $(Ker(f))^\perp$;
\item Fie $K$ un corp și fie $L=M_n(K)$. Există $X,Y \in L$ astfel încît $XY-YX=I_n$?  
\end{enumerate}
\item Fie forma pătratică:
$$Q= 2x_1^2+5x_2^2+2x_3^2-4x_1x_2-2x_1x_3+4x_2x_3$$

\begin{enumerate}
\item Aduceți forma pătratică la forma canonică prin metoda Gauss;
\item Aduceți forma pătratică la forma canonică prin transformări ortogonale;
\item Fie $\mathfrak{su}(n)=\{ A \in M_n(\mathbb{C}) | A+\bar{A}^t=0\}$. Arătați că $\mathfrak{su}(n)$ este spațiu vectorial real, dar nu complex.
Determinați $dim_{\mathbb{R}}\mathfrak{su}(n)$.
\end{enumerate}
\end{enumerate}
\newpage
\begin{flushright}
Nume:\_\_\_\_\_\_\_\_\_\_\_\_\_\_
 
 
Grupa:\_\_\_\_\_\_\_\_\_\_\_\_\_\_
\end{flushright}
\begin{center}
\vspace{2cm}
{\Large Probleme}
\vspace{2cm}
\end{center}
\begin{enumerate}
 \item Fie morfismul $f:\mathbb{R}^4 \to \mathbb{R}^3$ al cărui matrice în raport cu bazele canonice este
$$\begin{pmatrix}
0&-2&0&-2\\
-2&3&-2&-3\\
0&-1&0&-1
\end{pmatrix}$$

\begin{enumerate}
\item Determinați cîte o bază în $Ker(f)$ și $Im(f)$;
\item Fie vectorul $v=(1,3,1,3)$ determinați descompunerea acestuia ca suma dintre un vector din $Ker(f)$ și unul din $(Ker(f))^\perp$;
\item Fie $K$ un corp și fie $L=M_n(K)$. Există $X,Y \in L$ astfel încît $XY-YX=I_n$?  
\end{enumerate}
\item Fie forma pătratică:
$$Q= x_1^2+5x_2^2+x_3^2+2x_1x_2+6x_1x_3+2x_2x_3$$

\begin{enumerate}
\item Aduceți forma pătratică la forma canonică prin metoda Gauss;
\item Aduceți forma pătratică la forma canonică prin transformări ortogonale;
\item Fie $\mathfrak{su}(n)=\{ A \in M_n(\mathbb{C}) | A+\bar{A}^t=0\}$. Arătați că $\mathfrak{su}(n)$ este spațiu vectorial real, dar nu complex.
Determinați $dim_{\mathbb{R}}\mathfrak{su}(n)$.
\end{enumerate}
\end{enumerate}
\newpage
\begin{flushright}
Nume:\_\_\_\_\_\_\_\_\_\_\_\_\_\_
 
 
Grupa:\_\_\_\_\_\_\_\_\_\_\_\_\_\_
\end{flushright}
\begin{center}
\vspace{2cm}
{\Large Probleme}
\vspace{2cm}
\end{center}
\begin{enumerate}
 \item Fie morfismul $f:\mathbb{R}^4 \to \mathbb{R}^3$ al cărui matrice în raport cu bazele canonice este
$$\begin{pmatrix}
0&-2&0&-2\\
1&2&-1&0\\
1&1&-1&-1
\end{pmatrix}$$

\begin{enumerate}
\item Determinați cîte o bază în $Ker(f)$ și $Im(f)$;
\item Fie vectorul $v=(1,3,1,3)$ determinați descompunerea acestuia ca suma dintre un vector din $Ker(f)$ și unul din $(Ker(f))^\perp$;
\item Fie $K$ un corp și fie $L=M_n(K)$. Există $X,Y \in L$ astfel încît $XY-YX=I_n$?  
\end{enumerate}
\item Fie forma pătratică:
$$Q= x_1^2-2x_2^2+x_3^2+4x_1x_2-10x_1x_3+4x_2x_3$$

\begin{enumerate}
\item Aduceți forma pătratică la forma canonică prin metoda Gauss;
\item Aduceți forma pătratică la forma canonică prin transformări ortogonale;
\item Fie $\mathfrak{su}(n)=\{ A \in M_n(\mathbb{C}) | A+\bar{A}^t=0\}$. Arătați că $\mathfrak{su}(n)$ este spațiu vectorial real, dar nu complex.
Determinați $dim_{\mathbb{R}}\mathfrak{su}(n)$.
\end{enumerate}
\end{enumerate}
\newpage
\begin{flushright}
Nume:\_\_\_\_\_\_\_\_\_\_\_\_\_\_
 
 
Grupa:\_\_\_\_\_\_\_\_\_\_\_\_\_\_
\end{flushright}
\begin{center}
\vspace{2cm}
{\Large Probleme}
\vspace{2cm}
\end{center}
\begin{enumerate}
 \item Fie morfismul $f:\mathbb{R}^4 \to \mathbb{R}^3$ al cărui matrice în raport cu bazele canonice este
$$\begin{pmatrix}
0&-2&0&-2\\
-3&1&-3&1\\
2&3&2&3
\end{pmatrix}$$

\begin{enumerate}
\item Determinați cîte o bază în $Ker(f)$ și $Im(f)$;
\item Fie vectorul $v=(1,3,1,3)$ determinați descompunerea acestuia ca suma dintre un vector din $Ker(f)$ și unul din $(Ker(f))^\perp$;
\item Fie $K$ un corp și fie $L=M_n(K)$. Există $X,Y \in L$ astfel încît $XY-YX=I_n$?  
\end{enumerate}
\item Fie forma pătratică:
$$Q= 2x_1^2+5x_2^2+2x_3^2-4x_1x_2-2x_1x_3+4x_2x_3$$

\begin{enumerate}
\item Aduceți forma pătratică la forma canonică prin metoda Gauss;
\item Aduceți forma pătratică la forma canonică prin transformări ortogonale;
\item Fie $\mathfrak{su}(n)=\{ A \in M_n(\mathbb{C}) | A+\bar{A}^t=0\}$. Arătați că $\mathfrak{su}(n)$ este spațiu vectorial real, dar nu complex.
Determinați $dim_{\mathbb{R}}\mathfrak{su}(n)$.
\end{enumerate}
\end{enumerate}
\newpage
\begin{flushright}
Nume:\_\_\_\_\_\_\_\_\_\_\_\_\_\_
 
 
Grupa:\_\_\_\_\_\_\_\_\_\_\_\_\_\_
\end{flushright}
\begin{center}
\vspace{2cm}
{\Large Probleme}
\vspace{2cm}
\end{center}
\begin{enumerate}
 \item Fie morfismul $f:\mathbb{R}^4 \to \mathbb{R}^3$ al cărui matrice în raport cu bazele canonice este
$$\begin{pmatrix}
0&-2&0&-2\\
3&-1&-3&-2\\
3&2&-3&1
\end{pmatrix}$$

\begin{enumerate}
\item Determinați cîte o bază în $Ker(f)$ și $Im(f)$;
\item Fie vectorul $v=(1,3,1,3)$ determinați descompunerea acestuia ca suma dintre un vector din $Ker(f)$ și unul din $(Ker(f))^\perp$;
\item Fie $K$ un corp și fie $L=M_n(K)$. Există $X,Y \in L$ astfel încît $XY-YX=I_n$?  
\end{enumerate}
\item Fie forma pătratică:
$$Q= x_1^2+5x_2^2+x_3^2+2x_1x_2+6x_1x_3+2x_2x_3$$

\begin{enumerate}
\item Aduceți forma pătratică la forma canonică prin metoda Gauss;
\item Aduceți forma pătratică la forma canonică prin transformări ortogonale;
\item Fie $\mathfrak{su}(n)=\{ A \in M_n(\mathbb{C}) | A+\bar{A}^t=0\}$. Arătați că $\mathfrak{su}(n)$ este spațiu vectorial real, dar nu complex.
Determinați $dim_{\mathbb{R}}\mathfrak{su}(n)$.
\end{enumerate}
\end{enumerate}
\newpage
\begin{flushright}
Nume:\_\_\_\_\_\_\_\_\_\_\_\_\_\_
 
 
Grupa:\_\_\_\_\_\_\_\_\_\_\_\_\_\_
\end{flushright}
\begin{center}
\vspace{2cm}
{\Large Probleme}
\vspace{2cm}
\end{center}
\begin{enumerate}
 \item Fie morfismul $f:\mathbb{R}^4 \to \mathbb{R}^3$ al cărui matrice în raport cu bazele canonice este
$$\begin{pmatrix}
0&-2&0&-2\\
-3&0&2&2\\
0&2&0&2
\end{pmatrix}$$

\begin{enumerate}
\item Determinați cîte o bază în $Ker(f)$ și $Im(f)$;
\item Fie vectorul $v=(1,3,1,3)$ determinați descompunerea acestuia ca suma dintre un vector din $Ker(f)$ și unul din $(Ker(f))^\perp$;
\item Fie $K$ un corp și fie $L=M_n(K)$. Există $X,Y \in L$ astfel încît $XY-YX=I_n$?  
\end{enumerate}
\item Fie forma pătratică:
$$Q= x_1^2-2x_2^2+x_3^2+4x_1x_2-10x_1x_3+4x_2x_3$$

\begin{enumerate}
\item Aduceți forma pătratică la forma canonică prin metoda Gauss;
\item Aduceți forma pătratică la forma canonică prin transformări ortogonale;
\item Fie $\mathfrak{su}(n)=\{ A \in M_n(\mathbb{C}) | A+\bar{A}^t=0\}$. Arătați că $\mathfrak{su}(n)$ este spațiu vectorial real, dar nu complex.
Determinați $dim_{\mathbb{R}}\mathfrak{su}(n)$.
\end{enumerate}
\end{enumerate}
\newpage
\begin{flushright}
Nume:\_\_\_\_\_\_\_\_\_\_\_\_\_\_
 
 
Grupa:\_\_\_\_\_\_\_\_\_\_\_\_\_\_
\end{flushright}
\begin{center}
\vspace{2cm}
{\Large Probleme}
\vspace{2cm}
\end{center}
\begin{enumerate}
 \item Fie morfismul $f:\mathbb{R}^4 \to \mathbb{R}^3$ al cărui matrice în raport cu bazele canonice este
$$\begin{pmatrix}
2&-2&3&3\\
-2&2&-1&-3\\
-2&2&0&-3
\end{pmatrix}$$

\begin{enumerate}
\item Determinați cîte o bază în $Ker(f)$ și $Im(f)$;
\item Fie vectorul $v=(1,3,1,3)$ determinați descompunerea acestuia ca suma dintre un vector din $Ker(f)$ și unul din $(Ker(f))^\perp$;
\item Fie $K$ un corp și fie $L=M_n(K)$. Există $X,Y \in L$ astfel încît $XY-YX=I_n$?  
\end{enumerate}
\item Fie forma pătratică:
$$Q= 2x_1^2+5x_2^2+2x_3^2-4x_1x_2-2x_1x_3+4x_2x_3$$

\begin{enumerate}
\item Aduceți forma pătratică la forma canonică prin metoda Gauss;
\item Aduceți forma pătratică la forma canonică prin transformări ortogonale;
\item Fie $\mathfrak{su}(n)=\{ A \in M_n(\mathbb{C}) | A+\bar{A}^t=0\}$. Arătați că $\mathfrak{su}(n)$ este spațiu vectorial real, dar nu complex.
Determinați $dim_{\mathbb{R}}\mathfrak{su}(n)$.
\end{enumerate}
\end{enumerate}
\newpage
\begin{flushright}
Nume:\_\_\_\_\_\_\_\_\_\_\_\_\_\_
 
 
Grupa:\_\_\_\_\_\_\_\_\_\_\_\_\_\_
\end{flushright}
\begin{center}
\vspace{2cm}
{\Large Probleme}
\vspace{2cm}
\end{center}
\begin{enumerate}
 \item Fie morfismul $f:\mathbb{R}^4 \to \mathbb{R}^3$ al cărui matrice în raport cu bazele canonice este
$$\begin{pmatrix}
0&0&1&-1\\
0&0&-1&1\\
2&-3&-3&2
\end{pmatrix}$$

\begin{enumerate}
\item Determinați cîte o bază în $Ker(f)$ și $Im(f)$;
\item Fie vectorul $v=(1,3,1,3)$ determinați descompunerea acestuia ca suma dintre un vector din $Ker(f)$ și unul din $(Ker(f))^\perp$;
\item Fie $K$ un corp și fie $L=M_n(K)$. Există $X,Y \in L$ astfel încît $XY-YX=I_n$?  
\end{enumerate}
\item Fie forma pătratică:
$$Q= x_1^2+5x_2^2+x_3^2+2x_1x_2+6x_1x_3+2x_2x_3$$

\begin{enumerate}
\item Aduceți forma pătratică la forma canonică prin metoda Gauss;
\item Aduceți forma pătratică la forma canonică prin transformări ortogonale;
\item Fie $\mathfrak{su}(n)=\{ A \in M_n(\mathbb{C}) | A+\bar{A}^t=0\}$. Arătați că $\mathfrak{su}(n)$ este spațiu vectorial real, dar nu complex.
Determinați $dim_{\mathbb{R}}\mathfrak{su}(n)$.
\end{enumerate}
\end{enumerate}
\newpage
\begin{flushright}
Nume:\_\_\_\_\_\_\_\_\_\_\_\_\_\_
 
 
Grupa:\_\_\_\_\_\_\_\_\_\_\_\_\_\_
\end{flushright}
\begin{center}
\vspace{2cm}
{\Large Probleme}
\vspace{2cm}
\end{center}
\begin{enumerate}
 \item Fie morfismul $f:\mathbb{R}^4 \to \mathbb{R}^3$ al cărui matrice în raport cu bazele canonice este
$$\begin{pmatrix}
0&-2&0&-2\\
0&-3&0&-3\\
2&1&1&-2
\end{pmatrix}$$

\begin{enumerate}
\item Determinați cîte o bază în $Ker(f)$ și $Im(f)$;
\item Fie vectorul $v=(1,3,1,3)$ determinați descompunerea acestuia ca suma dintre un vector din $Ker(f)$ și unul din $(Ker(f))^\perp$;
\item Fie $K$ un corp și fie $L=M_n(K)$. Există $X,Y \in L$ astfel încît $XY-YX=I_n$?  
\end{enumerate}
\item Fie forma pătratică:
$$Q= x_1^2-2x_2^2+x_3^2+4x_1x_2-10x_1x_3+4x_2x_3$$

\begin{enumerate}
\item Aduceți forma pătratică la forma canonică prin metoda Gauss;
\item Aduceți forma pătratică la forma canonică prin transformări ortogonale;
\item Fie $\mathfrak{su}(n)=\{ A \in M_n(\mathbb{C}) | A+\bar{A}^t=0\}$. Arătați că $\mathfrak{su}(n)$ este spațiu vectorial real, dar nu complex.
Determinați $dim_{\mathbb{R}}\mathfrak{su}(n)$.
\end{enumerate}
\end{enumerate}
\newpage
\begin{flushright}
Nume:\_\_\_\_\_\_\_\_\_\_\_\_\_\_
 
 
Grupa:\_\_\_\_\_\_\_\_\_\_\_\_\_\_
\end{flushright}
\begin{center}
\vspace{2cm}
{\Large Probleme}
\vspace{2cm}
\end{center}
\begin{enumerate}
 \item Fie morfismul $f:\mathbb{R}^4 \to \mathbb{R}^3$ al cărui matrice în raport cu bazele canonice este
$$\begin{pmatrix}
2&-2&3&3\\
2&-2&-3&-3\\
1&-1&-2&-2
\end{pmatrix}$$

\begin{enumerate}
\item Determinați cîte o bază în $Ker(f)$ și $Im(f)$;
\item Fie vectorul $v=(1,3,1,3)$ determinați descompunerea acestuia ca suma dintre un vector din $Ker(f)$ și unul din $(Ker(f))^\perp$;
\item Fie $K$ un corp și fie $L=M_n(K)$. Există $X,Y \in L$ astfel încît $XY-YX=I_n$?  
\end{enumerate}
\item Fie forma pătratică:
$$Q= 2x_1^2+5x_2^2+2x_3^2-4x_1x_2-2x_1x_3+4x_2x_3$$

\begin{enumerate}
\item Aduceți forma pătratică la forma canonică prin metoda Gauss;
\item Aduceți forma pătratică la forma canonică prin transformări ortogonale;
\item Fie $\mathfrak{su}(n)=\{ A \in M_n(\mathbb{C}) | A+\bar{A}^t=0\}$. Arătați că $\mathfrak{su}(n)$ este spațiu vectorial real, dar nu complex.
Determinați $dim_{\mathbb{R}}\mathfrak{su}(n)$.
\end{enumerate}
\end{enumerate}
\newpage
\begin{flushright}
Nume:\_\_\_\_\_\_\_\_\_\_\_\_\_\_
 
 
Grupa:\_\_\_\_\_\_\_\_\_\_\_\_\_\_
\end{flushright}
\begin{center}
\vspace{2cm}
{\Large Probleme}
\vspace{2cm}
\end{center}
\begin{enumerate}
 \item Fie morfismul $f:\mathbb{R}^4 \to \mathbb{R}^3$ al cărui matrice în raport cu bazele canonice este
$$\begin{pmatrix}
0&-2&0&-2\\
0&-1&0&-1\\
3&1&2&-1
\end{pmatrix}$$

\begin{enumerate}
\item Determinați cîte o bază în $Ker(f)$ și $Im(f)$;
\item Fie vectorul $v=(1,3,1,3)$ determinați descompunerea acestuia ca suma dintre un vector din $Ker(f)$ și unul din $(Ker(f))^\perp$;
\item Fie $K$ un corp și fie $L=M_n(K)$. Există $X,Y \in L$ astfel încît $XY-YX=I_n$?  
\end{enumerate}
\item Fie forma pătratică:
$$Q= x_1^2+5x_2^2+x_3^2+2x_1x_2+6x_1x_3+2x_2x_3$$

\begin{enumerate}
\item Aduceți forma pătratică la forma canonică prin metoda Gauss;
\item Aduceți forma pătratică la forma canonică prin transformări ortogonale;
\item Fie $\mathfrak{su}(n)=\{ A \in M_n(\mathbb{C}) | A+\bar{A}^t=0\}$. Arătați că $\mathfrak{su}(n)$ este spațiu vectorial real, dar nu complex.
Determinați $dim_{\mathbb{R}}\mathfrak{su}(n)$.
\end{enumerate}
\end{enumerate}
\newpage
\begin{flushright}
Nume:\_\_\_\_\_\_\_\_\_\_\_\_\_\_
 
 
Grupa:\_\_\_\_\_\_\_\_\_\_\_\_\_\_
\end{flushright}
\begin{center}
\vspace{2cm}
{\Large Probleme}
\vspace{2cm}
\end{center}
\begin{enumerate}
 \item Fie morfismul $f:\mathbb{R}^4 \to \mathbb{R}^3$ al cărui matrice în raport cu bazele canonice este
$$\begin{pmatrix}
0&-2&0&-2\\
0&-3&0&-3\\
1&3&-1&2
\end{pmatrix}$$

\begin{enumerate}
\item Determinați cîte o bază în $Ker(f)$ și $Im(f)$;
\item Fie vectorul $v=(1,3,1,3)$ determinați descompunerea acestuia ca suma dintre un vector din $Ker(f)$ și unul din $(Ker(f))^\perp$;
\item Fie $K$ un corp și fie $L=M_n(K)$. Există $X,Y \in L$ astfel încît $XY-YX=I_n$?  
\end{enumerate}
\item Fie forma pătratică:
$$Q= x_1^2-2x_2^2+x_3^2+4x_1x_2-10x_1x_3+4x_2x_3$$

\begin{enumerate}
\item Aduceți forma pătratică la forma canonică prin metoda Gauss;
\item Aduceți forma pătratică la forma canonică prin transformări ortogonale;
\item Fie $\mathfrak{su}(n)=\{ A \in M_n(\mathbb{C}) | A+\bar{A}^t=0\}$. Arătați că $\mathfrak{su}(n)$ este spațiu vectorial real, dar nu complex.
Determinați $dim_{\mathbb{R}}\mathfrak{su}(n)$.
\end{enumerate}
\end{enumerate}
\newpage
\begin{flushright}
Nume:\_\_\_\_\_\_\_\_\_\_\_\_\_\_
 
 
Grupa:\_\_\_\_\_\_\_\_\_\_\_\_\_\_
\end{flushright}
\begin{center}
\vspace{2cm}
{\Large Probleme}
\vspace{2cm}
\end{center}
\begin{enumerate}
 \item Fie morfismul $f:\mathbb{R}^4 \to \mathbb{R}^3$ al cărui matrice în raport cu bazele canonice este
$$\begin{pmatrix}
0&0&1&-1\\
-3&-2&-1&-3\\
3&2&1&3
\end{pmatrix}$$

\begin{enumerate}
\item Determinați cîte o bază în $Ker(f)$ și $Im(f)$;
\item Fie vectorul $v=(1,3,1,3)$ determinați descompunerea acestuia ca suma dintre un vector din $Ker(f)$ și unul din $(Ker(f))^\perp$;
\item Fie $K$ un corp și fie $L=M_n(K)$. Există $X,Y \in L$ astfel încît $XY-YX=I_n$?  
\end{enumerate}
\item Fie forma pătratică:
$$Q= 2x_1^2+5x_2^2+2x_3^2-4x_1x_2-2x_1x_3+4x_2x_3$$

\begin{enumerate}
\item Aduceți forma pătratică la forma canonică prin metoda Gauss;
\item Aduceți forma pătratică la forma canonică prin transformări ortogonale;
\item Fie $\mathfrak{su}(n)=\{ A \in M_n(\mathbb{C}) | A+\bar{A}^t=0\}$. Arătați că $\mathfrak{su}(n)$ este spațiu vectorial real, dar nu complex.
Determinați $dim_{\mathbb{R}}\mathfrak{su}(n)$.
\end{enumerate}
\end{enumerate}
\newpage
\begin{flushright}
Nume:\_\_\_\_\_\_\_\_\_\_\_\_\_\_
 
 
Grupa:\_\_\_\_\_\_\_\_\_\_\_\_\_\_
\end{flushright}
\begin{center}
\vspace{2cm}
{\Large Probleme}
\vspace{2cm}
\end{center}
\begin{enumerate}
 \item Fie morfismul $f:\mathbb{R}^4 \to \mathbb{R}^3$ al cărui matrice în raport cu bazele canonice este
$$\begin{pmatrix}
0&-2&0&-2\\
3&0&-3&0\\
2&2&-2&2
\end{pmatrix}$$

\begin{enumerate}
\item Determinați cîte o bază în $Ker(f)$ și $Im(f)$;
\item Fie vectorul $v=(1,3,1,3)$ determinați descompunerea acestuia ca suma dintre un vector din $Ker(f)$ și unul din $(Ker(f))^\perp$;
\item Fie $K$ un corp și fie $L=M_n(K)$. Există $X,Y \in L$ astfel încît $XY-YX=I_n$?  
\end{enumerate}
\item Fie forma pătratică:
$$Q= x_1^2+5x_2^2+x_3^2+2x_1x_2+6x_1x_3+2x_2x_3$$

\begin{enumerate}
\item Aduceți forma pătratică la forma canonică prin metoda Gauss;
\item Aduceți forma pătratică la forma canonică prin transformări ortogonale;
\item Fie $\mathfrak{su}(n)=\{ A \in M_n(\mathbb{C}) | A+\bar{A}^t=0\}$. Arătați că $\mathfrak{su}(n)$ este spațiu vectorial real, dar nu complex.
Determinați $dim_{\mathbb{R}}\mathfrak{su}(n)$.
\end{enumerate}
\end{enumerate}
\newpage
\begin{flushright}
Nume:\_\_\_\_\_\_\_\_\_\_\_\_\_\_
 
 
Grupa:\_\_\_\_\_\_\_\_\_\_\_\_\_\_
\end{flushright}
\begin{center}
\vspace{2cm}
{\Large Probleme}
\vspace{2cm}
\end{center}
\begin{enumerate}
 \item Fie morfismul $f:\mathbb{R}^4 \to \mathbb{R}^3$ al cărui matrice în raport cu bazele canonice este
$$\begin{pmatrix}
0&3&2&2\\
-1&3&0&-2\\
1&-3&0&2
\end{pmatrix}$$

\begin{enumerate}
\item Determinați cîte o bază în $Ker(f)$ și $Im(f)$;
\item Fie vectorul $v=(1,3,1,3)$ determinați descompunerea acestuia ca suma dintre un vector din $Ker(f)$ și unul din $(Ker(f))^\perp$;
\item Fie $K$ un corp și fie $L=M_n(K)$. Există $X,Y \in L$ astfel încît $XY-YX=I_n$?  
\end{enumerate}
\item Fie forma pătratică:
$$Q= x_1^2-2x_2^2+x_3^2+4x_1x_2-10x_1x_3+4x_2x_3$$

\begin{enumerate}
\item Aduceți forma pătratică la forma canonică prin metoda Gauss;
\item Aduceți forma pătratică la forma canonică prin transformări ortogonale;
\item Fie $\mathfrak{su}(n)=\{ A \in M_n(\mathbb{C}) | A+\bar{A}^t=0\}$. Arătați că $\mathfrak{su}(n)$ este spațiu vectorial real, dar nu complex.
Determinați $dim_{\mathbb{R}}\mathfrak{su}(n)$.
\end{enumerate}
\end{enumerate}
\newpage
\begin{flushright}
Nume:\_\_\_\_\_\_\_\_\_\_\_\_\_\_
 
 
Grupa:\_\_\_\_\_\_\_\_\_\_\_\_\_\_
\end{flushright}
\begin{center}
\vspace{2cm}
{\Large Probleme}
\vspace{2cm}
\end{center}
\begin{enumerate}
 \item Fie morfismul $f:\mathbb{R}^4 \to \mathbb{R}^3$ al cărui matrice în raport cu bazele canonice este
$$\begin{pmatrix}
0&-2&0&-2\\
0&3&0&3\\
2&2&-1&3
\end{pmatrix}$$

\begin{enumerate}
\item Determinați cîte o bază în $Ker(f)$ și $Im(f)$;
\item Fie vectorul $v=(1,3,1,3)$ determinați descompunerea acestuia ca suma dintre un vector din $Ker(f)$ și unul din $(Ker(f))^\perp$;
\item Fie $K$ un corp și fie $L=M_n(K)$. Există $X,Y \in L$ astfel încît $XY-YX=I_n$?  
\end{enumerate}
\item Fie forma pătratică:
$$Q= 2x_1^2+5x_2^2+2x_3^2-4x_1x_2-2x_1x_3+4x_2x_3$$

\begin{enumerate}
\item Aduceți forma pătratică la forma canonică prin metoda Gauss;
\item Aduceți forma pătratică la forma canonică prin transformări ortogonale;
\item Fie $\mathfrak{su}(n)=\{ A \in M_n(\mathbb{C}) | A+\bar{A}^t=0\}$. Arătați că $\mathfrak{su}(n)$ este spațiu vectorial real, dar nu complex.
Determinați $dim_{\mathbb{R}}\mathfrak{su}(n)$.
\end{enumerate}
\end{enumerate}
\newpage
\begin{flushright}
Nume:\_\_\_\_\_\_\_\_\_\_\_\_\_\_
 
 
Grupa:\_\_\_\_\_\_\_\_\_\_\_\_\_\_
\end{flushright}
\begin{center}
\vspace{2cm}
{\Large Probleme}
\vspace{2cm}
\end{center}
\begin{enumerate}
 \item Fie morfismul $f:\mathbb{R}^4 \to \mathbb{R}^3$ al cărui matrice în raport cu bazele canonice este
$$\begin{pmatrix}
0&-2&0&-2\\
0&-1&0&-1\\
-3&1&-2&3
\end{pmatrix}$$

\begin{enumerate}
\item Determinați cîte o bază în $Ker(f)$ și $Im(f)$;
\item Fie vectorul $v=(1,3,1,3)$ determinați descompunerea acestuia ca suma dintre un vector din $Ker(f)$ și unul din $(Ker(f))^\perp$;
\item Fie $K$ un corp și fie $L=M_n(K)$. Există $X,Y \in L$ astfel încît $XY-YX=I_n$?  
\end{enumerate}
\item Fie forma pătratică:
$$Q= x_1^2+5x_2^2+x_3^2+2x_1x_2+6x_1x_3+2x_2x_3$$

\begin{enumerate}
\item Aduceți forma pătratică la forma canonică prin metoda Gauss;
\item Aduceți forma pătratică la forma canonică prin transformări ortogonale;
\item Fie $\mathfrak{su}(n)=\{ A \in M_n(\mathbb{C}) | A+\bar{A}^t=0\}$. Arătați că $\mathfrak{su}(n)$ este spațiu vectorial real, dar nu complex.
Determinați $dim_{\mathbb{R}}\mathfrak{su}(n)$.
\end{enumerate}
\end{enumerate}
\newpage
\begin{flushright}
Nume:\_\_\_\_\_\_\_\_\_\_\_\_\_\_
 
 
Grupa:\_\_\_\_\_\_\_\_\_\_\_\_\_\_
\end{flushright}
\begin{center}
\vspace{2cm}
{\Large Probleme}
\vspace{2cm}
\end{center}
\begin{enumerate}
 \item Fie morfismul $f:\mathbb{R}^4 \to \mathbb{R}^3$ al cărui matrice în raport cu bazele canonice este
$$\begin{pmatrix}
0&0&1&-1\\
2&-3&1&1\\
0&0&3&-3
\end{pmatrix}$$

\begin{enumerate}
\item Determinați cîte o bază în $Ker(f)$ și $Im(f)$;
\item Fie vectorul $v=(1,3,1,3)$ determinați descompunerea acestuia ca suma dintre un vector din $Ker(f)$ și unul din $(Ker(f))^\perp$;
\item Fie $K$ un corp și fie $L=M_n(K)$. Există $X,Y \in L$ astfel încît $XY-YX=I_n$?  
\end{enumerate}
\item Fie forma pătratică:
$$Q= x_1^2-2x_2^2+x_3^2+4x_1x_2-10x_1x_3+4x_2x_3$$

\begin{enumerate}
\item Aduceți forma pătratică la forma canonică prin metoda Gauss;
\item Aduceți forma pătratică la forma canonică prin transformări ortogonale;
\item Fie $\mathfrak{su}(n)=\{ A \in M_n(\mathbb{C}) | A+\bar{A}^t=0\}$. Arătați că $\mathfrak{su}(n)$ este spațiu vectorial real, dar nu complex.
Determinați $dim_{\mathbb{R}}\mathfrak{su}(n)$.
\end{enumerate}
\end{enumerate}
\newpage
\begin{flushright}
Nume:\_\_\_\_\_\_\_\_\_\_\_\_\_\_
 
 
Grupa:\_\_\_\_\_\_\_\_\_\_\_\_\_\_
\end{flushright}
\begin{center}
\vspace{2cm}
{\Large Probleme}
\vspace{2cm}
\end{center}
\begin{enumerate}
 \item Fie morfismul $f:\mathbb{R}^4 \to \mathbb{R}^3$ al cărui matrice în raport cu bazele canonice este
$$\begin{pmatrix}
2&-2&3&3\\
2&1&-2&2\\
-2&-1&2&-2
\end{pmatrix}$$

\begin{enumerate}
\item Determinați cîte o bază în $Ker(f)$ și $Im(f)$;
\item Fie vectorul $v=(1,3,1,3)$ determinați descompunerea acestuia ca suma dintre un vector din $Ker(f)$ și unul din $(Ker(f))^\perp$;
\item Fie $K$ un corp și fie $L=M_n(K)$. Există $X,Y \in L$ astfel încît $XY-YX=I_n$?  
\end{enumerate}
\item Fie forma pătratică:
$$Q= 2x_1^2+5x_2^2+2x_3^2-4x_1x_2-2x_1x_3+4x_2x_3$$

\begin{enumerate}
\item Aduceți forma pătratică la forma canonică prin metoda Gauss;
\item Aduceți forma pătratică la forma canonică prin transformări ortogonale;
\item Fie $\mathfrak{su}(n)=\{ A \in M_n(\mathbb{C}) | A+\bar{A}^t=0\}$. Arătați că $\mathfrak{su}(n)$ este spațiu vectorial real, dar nu complex.
Determinați $dim_{\mathbb{R}}\mathfrak{su}(n)$.
\end{enumerate}
\end{enumerate}
\newpage
\begin{flushright}
Nume:\_\_\_\_\_\_\_\_\_\_\_\_\_\_
 
 
Grupa:\_\_\_\_\_\_\_\_\_\_\_\_\_\_
\end{flushright}
\begin{center}
\vspace{2cm}
{\Large Probleme}
\vspace{2cm}
\end{center}
\begin{enumerate}
 \item Fie morfismul $f:\mathbb{R}^4 \to \mathbb{R}^3$ al cărui matrice în raport cu bazele canonice este
$$\begin{pmatrix}
0&0&1&-1\\
-2&3&-2&3\\
0&0&-3&3
\end{pmatrix}$$

\begin{enumerate}
\item Determinați cîte o bază în $Ker(f)$ și $Im(f)$;
\item Fie vectorul $v=(1,3,1,3)$ determinați descompunerea acestuia ca suma dintre un vector din $Ker(f)$ și unul din $(Ker(f))^\perp$;
\item Fie $K$ un corp și fie $L=M_n(K)$. Există $X,Y \in L$ astfel încît $XY-YX=I_n$?  
\end{enumerate}
\item Fie forma pătratică:
$$Q= x_1^2+5x_2^2+x_3^2+2x_1x_2+6x_1x_3+2x_2x_3$$

\begin{enumerate}
\item Aduceți forma pătratică la forma canonică prin metoda Gauss;
\item Aduceți forma pătratică la forma canonică prin transformări ortogonale;
\item Fie $\mathfrak{su}(n)=\{ A \in M_n(\mathbb{C}) | A+\bar{A}^t=0\}$. Arătați că $\mathfrak{su}(n)$ este spațiu vectorial real, dar nu complex.
Determinați $dim_{\mathbb{R}}\mathfrak{su}(n)$.
\end{enumerate}
\end{enumerate}
\newpage
\begin{flushright}
Nume:\_\_\_\_\_\_\_\_\_\_\_\_\_\_
 
 
Grupa:\_\_\_\_\_\_\_\_\_\_\_\_\_\_
\end{flushright}
\begin{center}
\vspace{2cm}
{\Large Probleme}
\vspace{2cm}
\end{center}
\begin{enumerate}
 \item Fie morfismul $f:\mathbb{R}^4 \to \mathbb{R}^3$ al cărui matrice în raport cu bazele canonice este
$$\begin{pmatrix}
0&-2&0&-2\\
-3&-2&-1&-3\\
-3&-1&-1&-2
\end{pmatrix}$$

\begin{enumerate}
\item Determinați cîte o bază în $Ker(f)$ și $Im(f)$;
\item Fie vectorul $v=(1,3,1,3)$ determinați descompunerea acestuia ca suma dintre un vector din $Ker(f)$ și unul din $(Ker(f))^\perp$;
\item Fie $K$ un corp și fie $L=M_n(K)$. Există $X,Y \in L$ astfel încît $XY-YX=I_n$?  
\end{enumerate}
\item Fie forma pătratică:
$$Q= x_1^2-2x_2^2+x_3^2+4x_1x_2-10x_1x_3+4x_2x_3$$

\begin{enumerate}
\item Aduceți forma pătratică la forma canonică prin metoda Gauss;
\item Aduceți forma pătratică la forma canonică prin transformări ortogonale;
\item Fie $\mathfrak{su}(n)=\{ A \in M_n(\mathbb{C}) | A+\bar{A}^t=0\}$. Arătați că $\mathfrak{su}(n)$ este spațiu vectorial real, dar nu complex.
Determinați $dim_{\mathbb{R}}\mathfrak{su}(n)$.
\end{enumerate}
\end{enumerate}
\newpage
\begin{flushright}
Nume:\_\_\_\_\_\_\_\_\_\_\_\_\_\_
 
 
Grupa:\_\_\_\_\_\_\_\_\_\_\_\_\_\_
\end{flushright}
\begin{center}
\vspace{2cm}
{\Large Probleme}
\vspace{2cm}
\end{center}
\begin{enumerate}
 \item Fie morfismul $f:\mathbb{R}^4 \to \mathbb{R}^3$ al cărui matrice în raport cu bazele canonice este
$$\begin{pmatrix}
0&-2&0&-2\\
-2&-2&3&-2\\
-2&1&3&1
\end{pmatrix}$$

\begin{enumerate}
\item Determinați cîte o bază în $Ker(f)$ și $Im(f)$;
\item Fie vectorul $v=(1,3,1,3)$ determinați descompunerea acestuia ca suma dintre un vector din $Ker(f)$ și unul din $(Ker(f))^\perp$;
\item Fie $K$ un corp și fie $L=M_n(K)$. Există $X,Y \in L$ astfel încît $XY-YX=I_n$?  
\end{enumerate}
\item Fie forma pătratică:
$$Q= 2x_1^2+5x_2^2+2x_3^2-4x_1x_2-2x_1x_3+4x_2x_3$$

\begin{enumerate}
\item Aduceți forma pătratică la forma canonică prin metoda Gauss;
\item Aduceți forma pătratică la forma canonică prin transformări ortogonale;
\item Fie $\mathfrak{su}(n)=\{ A \in M_n(\mathbb{C}) | A+\bar{A}^t=0\}$. Arătați că $\mathfrak{su}(n)$ este spațiu vectorial real, dar nu complex.
Determinați $dim_{\mathbb{R}}\mathfrak{su}(n)$.
\end{enumerate}
\end{enumerate}
\newpage
\begin{flushright}
Nume:\_\_\_\_\_\_\_\_\_\_\_\_\_\_
 
 
Grupa:\_\_\_\_\_\_\_\_\_\_\_\_\_\_
\end{flushright}
\begin{center}
\vspace{2cm}
{\Large Probleme}
\vspace{2cm}
\end{center}
\begin{enumerate}
 \item Fie morfismul $f:\mathbb{R}^4 \to \mathbb{R}^3$ al cărui matrice în raport cu bazele canonice este
$$\begin{pmatrix}
0&0&1&-1\\
3&3&1&-2\\
3&3&2&-3
\end{pmatrix}$$

\begin{enumerate}
\item Determinați cîte o bază în $Ker(f)$ și $Im(f)$;
\item Fie vectorul $v=(1,3,1,3)$ determinați descompunerea acestuia ca suma dintre un vector din $Ker(f)$ și unul din $(Ker(f))^\perp$;
\item Fie $K$ un corp și fie $L=M_n(K)$. Există $X,Y \in L$ astfel încît $XY-YX=I_n$?  
\end{enumerate}
\item Fie forma pătratică:
$$Q= x_1^2+5x_2^2+x_3^2+2x_1x_2+6x_1x_3+2x_2x_3$$

\begin{enumerate}
\item Aduceți forma pătratică la forma canonică prin metoda Gauss;
\item Aduceți forma pătratică la forma canonică prin transformări ortogonale;
\item Fie $\mathfrak{su}(n)=\{ A \in M_n(\mathbb{C}) | A+\bar{A}^t=0\}$. Arătați că $\mathfrak{su}(n)$ este spațiu vectorial real, dar nu complex.
Determinați $dim_{\mathbb{R}}\mathfrak{su}(n)$.
\end{enumerate}
\end{enumerate}
\newpage
\begin{flushright}
Nume:\_\_\_\_\_\_\_\_\_\_\_\_\_\_
 
 
Grupa:\_\_\_\_\_\_\_\_\_\_\_\_\_\_
\end{flushright}
\begin{center}
\vspace{2cm}
{\Large Probleme}
\vspace{2cm}
\end{center}
\begin{enumerate}
 \item Fie morfismul $f:\mathbb{R}^4 \to \mathbb{R}^3$ al cărui matrice în raport cu bazele canonice este
$$\begin{pmatrix}
0&0&1&-1\\
0&0&2&-2\\
3&3&0&3
\end{pmatrix}$$

\begin{enumerate}
\item Determinați cîte o bază în $Ker(f)$ și $Im(f)$;
\item Fie vectorul $v=(1,3,1,3)$ determinați descompunerea acestuia ca suma dintre un vector din $Ker(f)$ și unul din $(Ker(f))^\perp$;
\item Fie $K$ un corp și fie $L=M_n(K)$. Există $X,Y \in L$ astfel încît $XY-YX=I_n$?  
\end{enumerate}
\item Fie forma pătratică:
$$Q= x_1^2-2x_2^2+x_3^2+4x_1x_2-10x_1x_3+4x_2x_3$$

\begin{enumerate}
\item Aduceți forma pătratică la forma canonică prin metoda Gauss;
\item Aduceți forma pătratică la forma canonică prin transformări ortogonale;
\item Fie $\mathfrak{su}(n)=\{ A \in M_n(\mathbb{C}) | A+\bar{A}^t=0\}$. Arătați că $\mathfrak{su}(n)$ este spațiu vectorial real, dar nu complex.
Determinați $dim_{\mathbb{R}}\mathfrak{su}(n)$.
\end{enumerate}
\end{enumerate}
\newpage
\begin{flushright}
Nume:\_\_\_\_\_\_\_\_\_\_\_\_\_\_
 
 
Grupa:\_\_\_\_\_\_\_\_\_\_\_\_\_\_
\end{flushright}
\begin{center}
\vspace{2cm}
{\Large Probleme}
\vspace{2cm}
\end{center}
\begin{enumerate}
 \item Fie morfismul $f:\mathbb{R}^4 \to \mathbb{R}^3$ al cărui matrice în raport cu bazele canonice este
$$\begin{pmatrix}
0&-2&0&-2\\
-2&0&-2&2\\
1&-2&1&-3
\end{pmatrix}$$

\begin{enumerate}
\item Determinați cîte o bază în $Ker(f)$ și $Im(f)$;
\item Fie vectorul $v=(1,3,1,3)$ determinați descompunerea acestuia ca suma dintre un vector din $Ker(f)$ și unul din $(Ker(f))^\perp$;
\item Fie $K$ un corp și fie $L=M_n(K)$. Există $X,Y \in L$ astfel încît $XY-YX=I_n$?  
\end{enumerate}
\item Fie forma pătratică:
$$Q= 2x_1^2+5x_2^2+2x_3^2-4x_1x_2-2x_1x_3+4x_2x_3$$

\begin{enumerate}
\item Aduceți forma pătratică la forma canonică prin metoda Gauss;
\item Aduceți forma pătratică la forma canonică prin transformări ortogonale;
\item Fie $\mathfrak{su}(n)=\{ A \in M_n(\mathbb{C}) | A+\bar{A}^t=0\}$. Arătați că $\mathfrak{su}(n)$ este spațiu vectorial real, dar nu complex.
Determinați $dim_{\mathbb{R}}\mathfrak{su}(n)$.
\end{enumerate}
\end{enumerate}
\newpage
\begin{flushright}
Nume:\_\_\_\_\_\_\_\_\_\_\_\_\_\_
 
 
Grupa:\_\_\_\_\_\_\_\_\_\_\_\_\_\_
\end{flushright}
\begin{center}
\vspace{2cm}
{\Large Probleme}
\vspace{2cm}
\end{center}
\begin{enumerate}
 \item Fie morfismul $f:\mathbb{R}^4 \to \mathbb{R}^3$ al cărui matrice în raport cu bazele canonice este
$$\begin{pmatrix}
0&-2&0&-2\\
3&0&3&-2\\
3&-1&3&-3
\end{pmatrix}$$

\begin{enumerate}
\item Determinați cîte o bază în $Ker(f)$ și $Im(f)$;
\item Fie vectorul $v=(1,3,1,3)$ determinați descompunerea acestuia ca suma dintre un vector din $Ker(f)$ și unul din $(Ker(f))^\perp$;
\item Fie $K$ un corp și fie $L=M_n(K)$. Există $X,Y \in L$ astfel încît $XY-YX=I_n$?  
\end{enumerate}
\item Fie forma pătratică:
$$Q= x_1^2+5x_2^2+x_3^2+2x_1x_2+6x_1x_3+2x_2x_3$$

\begin{enumerate}
\item Aduceți forma pătratică la forma canonică prin metoda Gauss;
\item Aduceți forma pătratică la forma canonică prin transformări ortogonale;
\item Fie $\mathfrak{su}(n)=\{ A \in M_n(\mathbb{C}) | A+\bar{A}^t=0\}$. Arătați că $\mathfrak{su}(n)$ este spațiu vectorial real, dar nu complex.
Determinați $dim_{\mathbb{R}}\mathfrak{su}(n)$.
\end{enumerate}
\end{enumerate}
\newpage
\begin{flushright}
Nume:\_\_\_\_\_\_\_\_\_\_\_\_\_\_
 
 
Grupa:\_\_\_\_\_\_\_\_\_\_\_\_\_\_
\end{flushright}
\begin{center}
\vspace{2cm}
{\Large Probleme}
\vspace{2cm}
\end{center}
\begin{enumerate}
 \item Fie morfismul $f:\mathbb{R}^4 \to \mathbb{R}^3$ al cărui matrice în raport cu bazele canonice este
$$\begin{pmatrix}
2&-2&3&3\\
1&-1&0&1\\
-1&1&0&-1
\end{pmatrix}$$

\begin{enumerate}
\item Determinați cîte o bază în $Ker(f)$ și $Im(f)$;
\item Fie vectorul $v=(1,3,1,3)$ determinați descompunerea acestuia ca suma dintre un vector din $Ker(f)$ și unul din $(Ker(f))^\perp$;
\item Fie $K$ un corp și fie $L=M_n(K)$. Există $X,Y \in L$ astfel încît $XY-YX=I_n$?  
\end{enumerate}
\item Fie forma pătratică:
$$Q= x_1^2-2x_2^2+x_3^2+4x_1x_2-10x_1x_3+4x_2x_3$$

\begin{enumerate}
\item Aduceți forma pătratică la forma canonică prin metoda Gauss;
\item Aduceți forma pătratică la forma canonică prin transformări ortogonale;
\item Fie $\mathfrak{su}(n)=\{ A \in M_n(\mathbb{C}) | A+\bar{A}^t=0\}$. Arătați că $\mathfrak{su}(n)$ este spațiu vectorial real, dar nu complex.
Determinați $dim_{\mathbb{R}}\mathfrak{su}(n)$.
\end{enumerate}
\end{enumerate}
\newpage
\begin{flushright}
Nume:\_\_\_\_\_\_\_\_\_\_\_\_\_\_
 
 
Grupa:\_\_\_\_\_\_\_\_\_\_\_\_\_\_
\end{flushright}
\begin{center}
\vspace{2cm}
{\Large Probleme}
\vspace{2cm}
\end{center}
\begin{enumerate}
 \item Fie morfismul $f:\mathbb{R}^4 \to \mathbb{R}^3$ al cărui matrice în raport cu bazele canonice este
$$\begin{pmatrix}
0&0&1&-1\\
3&3&3&0\\
-2&-2&1&-3
\end{pmatrix}$$

\begin{enumerate}
\item Determinați cîte o bază în $Ker(f)$ și $Im(f)$;
\item Fie vectorul $v=(1,3,1,3)$ determinați descompunerea acestuia ca suma dintre un vector din $Ker(f)$ și unul din $(Ker(f))^\perp$;
\item Fie $K$ un corp și fie $L=M_n(K)$. Există $X,Y \in L$ astfel încît $XY-YX=I_n$?  
\end{enumerate}
\item Fie forma pătratică:
$$Q= 2x_1^2+5x_2^2+2x_3^2-4x_1x_2-2x_1x_3+4x_2x_3$$

\begin{enumerate}
\item Aduceți forma pătratică la forma canonică prin metoda Gauss;
\item Aduceți forma pătratică la forma canonică prin transformări ortogonale;
\item Fie $\mathfrak{su}(n)=\{ A \in M_n(\mathbb{C}) | A+\bar{A}^t=0\}$. Arătați că $\mathfrak{su}(n)$ este spațiu vectorial real, dar nu complex.
Determinați $dim_{\mathbb{R}}\mathfrak{su}(n)$.
\end{enumerate}
\end{enumerate}
\newpage
\begin{flushright}
Nume:\_\_\_\_\_\_\_\_\_\_\_\_\_\_
 
 
Grupa:\_\_\_\_\_\_\_\_\_\_\_\_\_\_
\end{flushright}
\begin{center}
\vspace{2cm}
{\Large Probleme}
\vspace{2cm}
\end{center}
\begin{enumerate}
 \item Fie morfismul $f:\mathbb{R}^4 \to \mathbb{R}^3$ al cărui matrice în raport cu bazele canonice este
$$\begin{pmatrix}
0&-2&0&-2\\
0&2&0&2\\
-2&-3&-1&0
\end{pmatrix}$$

\begin{enumerate}
\item Determinați cîte o bază în $Ker(f)$ și $Im(f)$;
\item Fie vectorul $v=(1,3,1,3)$ determinați descompunerea acestuia ca suma dintre un vector din $Ker(f)$ și unul din $(Ker(f))^\perp$;
\item Fie $K$ un corp și fie $L=M_n(K)$. Există $X,Y \in L$ astfel încît $XY-YX=I_n$?  
\end{enumerate}
\item Fie forma pătratică:
$$Q= x_1^2+5x_2^2+x_3^2+2x_1x_2+6x_1x_3+2x_2x_3$$

\begin{enumerate}
\item Aduceți forma pătratică la forma canonică prin metoda Gauss;
\item Aduceți forma pătratică la forma canonică prin transformări ortogonale;
\item Fie $\mathfrak{su}(n)=\{ A \in M_n(\mathbb{C}) | A+\bar{A}^t=0\}$. Arătați că $\mathfrak{su}(n)$ este spațiu vectorial real, dar nu complex.
Determinați $dim_{\mathbb{R}}\mathfrak{su}(n)$.
\end{enumerate}
\end{enumerate}
\newpage
\begin{flushright}
Nume:\_\_\_\_\_\_\_\_\_\_\_\_\_\_
 
 
Grupa:\_\_\_\_\_\_\_\_\_\_\_\_\_\_
\end{flushright}
\begin{center}
\vspace{2cm}
{\Large Probleme}
\vspace{2cm}
\end{center}
\begin{enumerate}
 \item Fie morfismul $f:\mathbb{R}^4 \to \mathbb{R}^3$ al cărui matrice în raport cu bazele canonice este
$$\begin{pmatrix}
2&-2&3&3\\
0&-2&-2&-2\\
0&1&1&1
\end{pmatrix}$$

\begin{enumerate}
\item Determinați cîte o bază în $Ker(f)$ și $Im(f)$;
\item Fie vectorul $v=(1,3,1,3)$ determinați descompunerea acestuia ca suma dintre un vector din $Ker(f)$ și unul din $(Ker(f))^\perp$;
\item Fie $K$ un corp și fie $L=M_n(K)$. Există $X,Y \in L$ astfel încît $XY-YX=I_n$?  
\end{enumerate}
\item Fie forma pătratică:
$$Q= x_1^2-2x_2^2+x_3^2+4x_1x_2-10x_1x_3+4x_2x_3$$

\begin{enumerate}
\item Aduceți forma pătratică la forma canonică prin metoda Gauss;
\item Aduceți forma pătratică la forma canonică prin transformări ortogonale;
\item Fie $\mathfrak{su}(n)=\{ A \in M_n(\mathbb{C}) | A+\bar{A}^t=0\}$. Arătați că $\mathfrak{su}(n)$ este spațiu vectorial real, dar nu complex.
Determinați $dim_{\mathbb{R}}\mathfrak{su}(n)$.
\end{enumerate}
\end{enumerate}
\newpage
\begin{flushright}
Nume:\_\_\_\_\_\_\_\_\_\_\_\_\_\_
 
 
Grupa:\_\_\_\_\_\_\_\_\_\_\_\_\_\_
\end{flushright}
\begin{center}
\vspace{2cm}
{\Large Probleme}
\vspace{2cm}
\end{center}
\begin{enumerate}
 \item Fie morfismul $f:\mathbb{R}^4 \to \mathbb{R}^3$ al cărui matrice în raport cu bazele canonice este
$$\begin{pmatrix}
2&-2&3&3\\
0&0&-1&-1\\
1&-1&0&0
\end{pmatrix}$$

\begin{enumerate}
\item Determinați cîte o bază în $Ker(f)$ și $Im(f)$;
\item Fie vectorul $v=(1,3,1,3)$ determinați descompunerea acestuia ca suma dintre un vector din $Ker(f)$ și unul din $(Ker(f))^\perp$;
\item Fie $K$ un corp și fie $L=M_n(K)$. Există $X,Y \in L$ astfel încît $XY-YX=I_n$?  
\end{enumerate}
\item Fie forma pătratică:
$$Q= 2x_1^2+5x_2^2+2x_3^2-4x_1x_2-2x_1x_3+4x_2x_3$$

\begin{enumerate}
\item Aduceți forma pătratică la forma canonică prin metoda Gauss;
\item Aduceți forma pătratică la forma canonică prin transformări ortogonale;
\item Fie $\mathfrak{su}(n)=\{ A \in M_n(\mathbb{C}) | A+\bar{A}^t=0\}$. Arătați că $\mathfrak{su}(n)$ este spațiu vectorial real, dar nu complex.
Determinați $dim_{\mathbb{R}}\mathfrak{su}(n)$.
\end{enumerate}
\end{enumerate}
\newpage
\begin{flushright}
Nume:\_\_\_\_\_\_\_\_\_\_\_\_\_\_
 
 
Grupa:\_\_\_\_\_\_\_\_\_\_\_\_\_\_
\end{flushright}
\begin{center}
\vspace{2cm}
{\Large Probleme}
\vspace{2cm}
\end{center}
\begin{enumerate}
 \item Fie morfismul $f:\mathbb{R}^4 \to \mathbb{R}^3$ al cărui matrice în raport cu bazele canonice este
$$\begin{pmatrix}
0&-2&0&-2\\
-1&0&2&-2\\
1&-1&-2&1
\end{pmatrix}$$

\begin{enumerate}
\item Determinați cîte o bază în $Ker(f)$ și $Im(f)$;
\item Fie vectorul $v=(1,3,1,3)$ determinați descompunerea acestuia ca suma dintre un vector din $Ker(f)$ și unul din $(Ker(f))^\perp$;
\item Fie $K$ un corp și fie $L=M_n(K)$. Există $X,Y \in L$ astfel încît $XY-YX=I_n$?  
\end{enumerate}
\item Fie forma pătratică:
$$Q= x_1^2+5x_2^2+x_3^2+2x_1x_2+6x_1x_3+2x_2x_3$$

\begin{enumerate}
\item Aduceți forma pătratică la forma canonică prin metoda Gauss;
\item Aduceți forma pătratică la forma canonică prin transformări ortogonale;
\item Fie $\mathfrak{su}(n)=\{ A \in M_n(\mathbb{C}) | A+\bar{A}^t=0\}$. Arătați că $\mathfrak{su}(n)$ este spațiu vectorial real, dar nu complex.
Determinați $dim_{\mathbb{R}}\mathfrak{su}(n)$.
\end{enumerate}
\end{enumerate}
\newpage
\begin{flushright}
Nume:\_\_\_\_\_\_\_\_\_\_\_\_\_\_
 
 
Grupa:\_\_\_\_\_\_\_\_\_\_\_\_\_\_
\end{flushright}
\begin{center}
\vspace{2cm}
{\Large Probleme}
\vspace{2cm}
\end{center}
\begin{enumerate}
 \item Fie morfismul $f:\mathbb{R}^4 \to \mathbb{R}^3$ al cărui matrice în raport cu bazele canonice este
$$\begin{pmatrix}
0&-2&0&-2\\
0&-3&0&-3\\
1&-1&-2&-2
\end{pmatrix}$$

\begin{enumerate}
\item Determinați cîte o bază în $Ker(f)$ și $Im(f)$;
\item Fie vectorul $v=(1,3,1,3)$ determinați descompunerea acestuia ca suma dintre un vector din $Ker(f)$ și unul din $(Ker(f))^\perp$;
\item Fie $K$ un corp și fie $L=M_n(K)$. Există $X,Y \in L$ astfel încît $XY-YX=I_n$?  
\end{enumerate}
\item Fie forma pătratică:
$$Q= x_1^2-2x_2^2+x_3^2+4x_1x_2-10x_1x_3+4x_2x_3$$

\begin{enumerate}
\item Aduceți forma pătratică la forma canonică prin metoda Gauss;
\item Aduceți forma pătratică la forma canonică prin transformări ortogonale;
\item Fie $\mathfrak{su}(n)=\{ A \in M_n(\mathbb{C}) | A+\bar{A}^t=0\}$. Arătați că $\mathfrak{su}(n)$ este spațiu vectorial real, dar nu complex.
Determinați $dim_{\mathbb{R}}\mathfrak{su}(n)$.
\end{enumerate}
\end{enumerate}
\newpage
\begin{flushright}
Nume:\_\_\_\_\_\_\_\_\_\_\_\_\_\_
 
 
Grupa:\_\_\_\_\_\_\_\_\_\_\_\_\_\_
\end{flushright}
\begin{center}
\vspace{2cm}
{\Large Probleme}
\vspace{2cm}
\end{center}
\begin{enumerate}
 \item Fie morfismul $f:\mathbb{R}^4 \to \mathbb{R}^3$ al cărui matrice în raport cu bazele canonice este
$$\begin{pmatrix}
0&-2&0&-2\\
-1&2&-3&-2\\
0&1&0&1
\end{pmatrix}$$

\begin{enumerate}
\item Determinați cîte o bază în $Ker(f)$ și $Im(f)$;
\item Fie vectorul $v=(1,3,1,3)$ determinați descompunerea acestuia ca suma dintre un vector din $Ker(f)$ și unul din $(Ker(f))^\perp$;
\item Fie $K$ un corp și fie $L=M_n(K)$. Există $X,Y \in L$ astfel încît $XY-YX=I_n$?  
\end{enumerate}
\item Fie forma pătratică:
$$Q= 2x_1^2+5x_2^2+2x_3^2-4x_1x_2-2x_1x_3+4x_2x_3$$

\begin{enumerate}
\item Aduceți forma pătratică la forma canonică prin metoda Gauss;
\item Aduceți forma pătratică la forma canonică prin transformări ortogonale;
\item Fie $\mathfrak{su}(n)=\{ A \in M_n(\mathbb{C}) | A+\bar{A}^t=0\}$. Arătați că $\mathfrak{su}(n)$ este spațiu vectorial real, dar nu complex.
Determinați $dim_{\mathbb{R}}\mathfrak{su}(n)$.
\end{enumerate}
\end{enumerate}
\newpage
\begin{flushright}
Nume:\_\_\_\_\_\_\_\_\_\_\_\_\_\_
 
 
Grupa:\_\_\_\_\_\_\_\_\_\_\_\_\_\_
\end{flushright}
\begin{center}
\vspace{2cm}
{\Large Probleme}
\vspace{2cm}
\end{center}
\begin{enumerate}
 \item Fie morfismul $f:\mathbb{R}^4 \to \mathbb{R}^3$ al cărui matrice în raport cu bazele canonice este
$$\begin{pmatrix}
0&-2&0&-2\\
-1&0&-3&2\\
0&3&0&3
\end{pmatrix}$$

\begin{enumerate}
\item Determinați cîte o bază în $Ker(f)$ și $Im(f)$;
\item Fie vectorul $v=(1,3,1,3)$ determinați descompunerea acestuia ca suma dintre un vector din $Ker(f)$ și unul din $(Ker(f))^\perp$;
\item Fie $K$ un corp și fie $L=M_n(K)$. Există $X,Y \in L$ astfel încît $XY-YX=I_n$?  
\end{enumerate}
\item Fie forma pătratică:
$$Q= x_1^2+5x_2^2+x_3^2+2x_1x_2+6x_1x_3+2x_2x_3$$

\begin{enumerate}
\item Aduceți forma pătratică la forma canonică prin metoda Gauss;
\item Aduceți forma pătratică la forma canonică prin transformări ortogonale;
\item Fie $\mathfrak{su}(n)=\{ A \in M_n(\mathbb{C}) | A+\bar{A}^t=0\}$. Arătați că $\mathfrak{su}(n)$ este spațiu vectorial real, dar nu complex.
Determinați $dim_{\mathbb{R}}\mathfrak{su}(n)$.
\end{enumerate}
\end{enumerate}
\newpage
\begin{flushright}
Nume:\_\_\_\_\_\_\_\_\_\_\_\_\_\_
 
 
Grupa:\_\_\_\_\_\_\_\_\_\_\_\_\_\_
\end{flushright}
\begin{center}
\vspace{2cm}
{\Large Probleme}
\vspace{2cm}
\end{center}
\begin{enumerate}
 \item Fie morfismul $f:\mathbb{R}^4 \to \mathbb{R}^3$ al cărui matrice în raport cu bazele canonice este
$$\begin{pmatrix}
0&0&1&-1\\
2&2&-3&1\\
2&2&0&-2
\end{pmatrix}$$

\begin{enumerate}
\item Determinați cîte o bază în $Ker(f)$ și $Im(f)$;
\item Fie vectorul $v=(1,3,1,3)$ determinați descompunerea acestuia ca suma dintre un vector din $Ker(f)$ și unul din $(Ker(f))^\perp$;
\item Fie $K$ un corp și fie $L=M_n(K)$. Există $X,Y \in L$ astfel încît $XY-YX=I_n$?  
\end{enumerate}
\item Fie forma pătratică:
$$Q= x_1^2-2x_2^2+x_3^2+4x_1x_2-10x_1x_3+4x_2x_3$$

\begin{enumerate}
\item Aduceți forma pătratică la forma canonică prin metoda Gauss;
\item Aduceți forma pătratică la forma canonică prin transformări ortogonale;
\item Fie $\mathfrak{su}(n)=\{ A \in M_n(\mathbb{C}) | A+\bar{A}^t=0\}$. Arătați că $\mathfrak{su}(n)$ este spațiu vectorial real, dar nu complex.
Determinați $dim_{\mathbb{R}}\mathfrak{su}(n)$.
\end{enumerate}
\end{enumerate}
\newpage
\begin{flushright}
Nume:\_\_\_\_\_\_\_\_\_\_\_\_\_\_
 
 
Grupa:\_\_\_\_\_\_\_\_\_\_\_\_\_\_
\end{flushright}
\begin{center}
\vspace{2cm}
{\Large Probleme}
\vspace{2cm}
\end{center}
\begin{enumerate}
 \item Fie morfismul $f:\mathbb{R}^4 \to \mathbb{R}^3$ al cărui matrice în raport cu bazele canonice este
$$\begin{pmatrix}
2&-2&3&3\\
2&-2&-2&-2\\
2&-2&2&2
\end{pmatrix}$$

\begin{enumerate}
\item Determinați cîte o bază în $Ker(f)$ și $Im(f)$;
\item Fie vectorul $v=(1,3,1,3)$ determinați descompunerea acestuia ca suma dintre un vector din $Ker(f)$ și unul din $(Ker(f))^\perp$;
\item Fie $K$ un corp și fie $L=M_n(K)$. Există $X,Y \in L$ astfel încît $XY-YX=I_n$?  
\end{enumerate}
\item Fie forma pătratică:
$$Q= 2x_1^2+5x_2^2+2x_3^2-4x_1x_2-2x_1x_3+4x_2x_3$$

\begin{enumerate}
\item Aduceți forma pătratică la forma canonică prin metoda Gauss;
\item Aduceți forma pătratică la forma canonică prin transformări ortogonale;
\item Fie $\mathfrak{su}(n)=\{ A \in M_n(\mathbb{C}) | A+\bar{A}^t=0\}$. Arătați că $\mathfrak{su}(n)$ este spațiu vectorial real, dar nu complex.
Determinați $dim_{\mathbb{R}}\mathfrak{su}(n)$.
\end{enumerate}
\end{enumerate}
\newpage
\begin{flushright}
Nume:\_\_\_\_\_\_\_\_\_\_\_\_\_\_
 
 
Grupa:\_\_\_\_\_\_\_\_\_\_\_\_\_\_
\end{flushright}
\begin{center}
\vspace{2cm}
{\Large Probleme}
\vspace{2cm}
\end{center}
\begin{enumerate}
 \item Fie morfismul $f:\mathbb{R}^4 \to \mathbb{R}^3$ al cărui matrice în raport cu bazele canonice este
$$\begin{pmatrix}
0&3&2&2\\
-1&1&-3&1\\
1&-1&3&-1
\end{pmatrix}$$

\begin{enumerate}
\item Determinați cîte o bază în $Ker(f)$ și $Im(f)$;
\item Fie vectorul $v=(1,3,1,3)$ determinați descompunerea acestuia ca suma dintre un vector din $Ker(f)$ și unul din $(Ker(f))^\perp$;
\item Fie $K$ un corp și fie $L=M_n(K)$. Există $X,Y \in L$ astfel încît $XY-YX=I_n$?  
\end{enumerate}
\item Fie forma pătratică:
$$Q= x_1^2+5x_2^2+x_3^2+2x_1x_2+6x_1x_3+2x_2x_3$$

\begin{enumerate}
\item Aduceți forma pătratică la forma canonică prin metoda Gauss;
\item Aduceți forma pătratică la forma canonică prin transformări ortogonale;
\item Fie $\mathfrak{su}(n)=\{ A \in M_n(\mathbb{C}) | A+\bar{A}^t=0\}$. Arătați că $\mathfrak{su}(n)$ este spațiu vectorial real, dar nu complex.
Determinați $dim_{\mathbb{R}}\mathfrak{su}(n)$.
\end{enumerate}
\end{enumerate}
\newpage
\begin{flushright}
Nume:\_\_\_\_\_\_\_\_\_\_\_\_\_\_
 
 
Grupa:\_\_\_\_\_\_\_\_\_\_\_\_\_\_
\end{flushright}
\begin{center}
\vspace{2cm}
{\Large Probleme}
\vspace{2cm}
\end{center}
\begin{enumerate}
 \item Fie morfismul $f:\mathbb{R}^4 \to \mathbb{R}^3$ al cărui matrice în raport cu bazele canonice este
$$\begin{pmatrix}
0&-2&0&-2\\
-1&-1&2&1\\
1&-1&-2&-3
\end{pmatrix}$$

\begin{enumerate}
\item Determinați cîte o bază în $Ker(f)$ și $Im(f)$;
\item Fie vectorul $v=(1,3,1,3)$ determinați descompunerea acestuia ca suma dintre un vector din $Ker(f)$ și unul din $(Ker(f))^\perp$;
\item Fie $K$ un corp și fie $L=M_n(K)$. Există $X,Y \in L$ astfel încît $XY-YX=I_n$?  
\end{enumerate}
\item Fie forma pătratică:
$$Q= x_1^2-2x_2^2+x_3^2+4x_1x_2-10x_1x_3+4x_2x_3$$

\begin{enumerate}
\item Aduceți forma pătratică la forma canonică prin metoda Gauss;
\item Aduceți forma pătratică la forma canonică prin transformări ortogonale;
\item Fie $\mathfrak{su}(n)=\{ A \in M_n(\mathbb{C}) | A+\bar{A}^t=0\}$. Arătați că $\mathfrak{su}(n)$ este spațiu vectorial real, dar nu complex.
Determinați $dim_{\mathbb{R}}\mathfrak{su}(n)$.
\end{enumerate}
\end{enumerate}
\newpage
\begin{flushright}
Nume:\_\_\_\_\_\_\_\_\_\_\_\_\_\_
 
 
Grupa:\_\_\_\_\_\_\_\_\_\_\_\_\_\_
\end{flushright}
\begin{center}
\vspace{2cm}
{\Large Probleme}
\vspace{2cm}
\end{center}
\begin{enumerate}
 \item Fie morfismul $f:\mathbb{R}^4 \to \mathbb{R}^3$ al cărui matrice în raport cu bazele canonice este
$$\begin{pmatrix}
0&0&1&-1\\
-1&1&-1&1\\
0&0&-1&1
\end{pmatrix}$$

\begin{enumerate}
\item Determinați cîte o bază în $Ker(f)$ și $Im(f)$;
\item Fie vectorul $v=(1,3,1,3)$ determinați descompunerea acestuia ca suma dintre un vector din $Ker(f)$ și unul din $(Ker(f))^\perp$;
\item Fie $K$ un corp și fie $L=M_n(K)$. Există $X,Y \in L$ astfel încît $XY-YX=I_n$?  
\end{enumerate}
\item Fie forma pătratică:
$$Q= 2x_1^2+5x_2^2+2x_3^2-4x_1x_2-2x_1x_3+4x_2x_3$$

\begin{enumerate}
\item Aduceți forma pătratică la forma canonică prin metoda Gauss;
\item Aduceți forma pătratică la forma canonică prin transformări ortogonale;
\item Fie $\mathfrak{su}(n)=\{ A \in M_n(\mathbb{C}) | A+\bar{A}^t=0\}$. Arătați că $\mathfrak{su}(n)$ este spațiu vectorial real, dar nu complex.
Determinați $dim_{\mathbb{R}}\mathfrak{su}(n)$.
\end{enumerate}
\end{enumerate}
\newpage
\begin{flushright}
Nume:\_\_\_\_\_\_\_\_\_\_\_\_\_\_
 
 
Grupa:\_\_\_\_\_\_\_\_\_\_\_\_\_\_
\end{flushright}
\begin{center}
\vspace{2cm}
{\Large Probleme}
\vspace{2cm}
\end{center}
\begin{enumerate}
 \item Fie morfismul $f:\mathbb{R}^4 \to \mathbb{R}^3$ al cărui matrice în raport cu bazele canonice este
$$\begin{pmatrix}
0&-2&0&-2\\
2&-2&1&-3\\
-2&-3&-1&-2
\end{pmatrix}$$

\begin{enumerate}
\item Determinați cîte o bază în $Ker(f)$ și $Im(f)$;
\item Fie vectorul $v=(1,3,1,3)$ determinați descompunerea acestuia ca suma dintre un vector din $Ker(f)$ și unul din $(Ker(f))^\perp$;
\item Fie $K$ un corp și fie $L=M_n(K)$. Există $X,Y \in L$ astfel încît $XY-YX=I_n$?  
\end{enumerate}
\item Fie forma pătratică:
$$Q= x_1^2+5x_2^2+x_3^2+2x_1x_2+6x_1x_3+2x_2x_3$$

\begin{enumerate}
\item Aduceți forma pătratică la forma canonică prin metoda Gauss;
\item Aduceți forma pătratică la forma canonică prin transformări ortogonale;
\item Fie $\mathfrak{su}(n)=\{ A \in M_n(\mathbb{C}) | A+\bar{A}^t=0\}$. Arătați că $\mathfrak{su}(n)$ este spațiu vectorial real, dar nu complex.
Determinați $dim_{\mathbb{R}}\mathfrak{su}(n)$.
\end{enumerate}
\end{enumerate}
\newpage
\begin{flushright}
Nume:\_\_\_\_\_\_\_\_\_\_\_\_\_\_
 
 
Grupa:\_\_\_\_\_\_\_\_\_\_\_\_\_\_
\end{flushright}
\begin{center}
\vspace{2cm}
{\Large Probleme}
\vspace{2cm}
\end{center}
\begin{enumerate}
 \item Fie morfismul $f:\mathbb{R}^4 \to \mathbb{R}^3$ al cărui matrice în raport cu bazele canonice este
$$\begin{pmatrix}
0&0&1&-1\\
2&3&0&-2\\
0&0&3&-3
\end{pmatrix}$$

\begin{enumerate}
\item Determinați cîte o bază în $Ker(f)$ și $Im(f)$;
\item Fie vectorul $v=(1,3,1,3)$ determinați descompunerea acestuia ca suma dintre un vector din $Ker(f)$ și unul din $(Ker(f))^\perp$;
\item Fie $K$ un corp și fie $L=M_n(K)$. Există $X,Y \in L$ astfel încît $XY-YX=I_n$?  
\end{enumerate}
\item Fie forma pătratică:
$$Q= x_1^2-2x_2^2+x_3^2+4x_1x_2-10x_1x_3+4x_2x_3$$

\begin{enumerate}
\item Aduceți forma pătratică la forma canonică prin metoda Gauss;
\item Aduceți forma pătratică la forma canonică prin transformări ortogonale;
\item Fie $\mathfrak{su}(n)=\{ A \in M_n(\mathbb{C}) | A+\bar{A}^t=0\}$. Arătați că $\mathfrak{su}(n)$ este spațiu vectorial real, dar nu complex.
Determinați $dim_{\mathbb{R}}\mathfrak{su}(n)$.
\end{enumerate}
\end{enumerate}
\newpage
\begin{flushright}
Nume:\_\_\_\_\_\_\_\_\_\_\_\_\_\_
 
 
Grupa:\_\_\_\_\_\_\_\_\_\_\_\_\_\_
\end{flushright}
\begin{center}
\vspace{2cm}
{\Large Probleme}
\vspace{2cm}
\end{center}
\begin{enumerate}
 \item Fie morfismul $f:\mathbb{R}^4 \to \mathbb{R}^3$ al cărui matrice în raport cu bazele canonice este
$$\begin{pmatrix}
2&-2&3&3\\
-3&0&0&-1\\
-2&2&-3&-3
\end{pmatrix}$$

\begin{enumerate}
\item Determinați cîte o bază în $Ker(f)$ și $Im(f)$;
\item Fie vectorul $v=(1,3,1,3)$ determinați descompunerea acestuia ca suma dintre un vector din $Ker(f)$ și unul din $(Ker(f))^\perp$;
\item Fie $K$ un corp și fie $L=M_n(K)$. Există $X,Y \in L$ astfel încît $XY-YX=I_n$?  
\end{enumerate}
\item Fie forma pătratică:
$$Q= 2x_1^2+5x_2^2+2x_3^2-4x_1x_2-2x_1x_3+4x_2x_3$$

\begin{enumerate}
\item Aduceți forma pătratică la forma canonică prin metoda Gauss;
\item Aduceți forma pătratică la forma canonică prin transformări ortogonale;
\item Fie $\mathfrak{su}(n)=\{ A \in M_n(\mathbb{C}) | A+\bar{A}^t=0\}$. Arătați că $\mathfrak{su}(n)$ este spațiu vectorial real, dar nu complex.
Determinați $dim_{\mathbb{R}}\mathfrak{su}(n)$.
\end{enumerate}
\end{enumerate}
\newpage
\begin{flushright}
Nume:\_\_\_\_\_\_\_\_\_\_\_\_\_\_
 
 
Grupa:\_\_\_\_\_\_\_\_\_\_\_\_\_\_
\end{flushright}
\begin{center}
\vspace{2cm}
{\Large Probleme}
\vspace{2cm}
\end{center}
\begin{enumerate}
 \item Fie morfismul $f:\mathbb{R}^4 \to \mathbb{R}^3$ al cărui matrice în raport cu bazele canonice este
$$\begin{pmatrix}
0&-2&0&-2\\
2&1&-2&-1\\
-1&1&1&2
\end{pmatrix}$$

\begin{enumerate}
\item Determinați cîte o bază în $Ker(f)$ și $Im(f)$;
\item Fie vectorul $v=(1,3,1,3)$ determinați descompunerea acestuia ca suma dintre un vector din $Ker(f)$ și unul din $(Ker(f))^\perp$;
\item Fie $K$ un corp și fie $L=M_n(K)$. Există $X,Y \in L$ astfel încît $XY-YX=I_n$?  
\end{enumerate}
\item Fie forma pătratică:
$$Q= x_1^2+5x_2^2+x_3^2+2x_1x_2+6x_1x_3+2x_2x_3$$

\begin{enumerate}
\item Aduceți forma pătratică la forma canonică prin metoda Gauss;
\item Aduceți forma pătratică la forma canonică prin transformări ortogonale;
\item Fie $\mathfrak{su}(n)=\{ A \in M_n(\mathbb{C}) | A+\bar{A}^t=0\}$. Arătați că $\mathfrak{su}(n)$ este spațiu vectorial real, dar nu complex.
Determinați $dim_{\mathbb{R}}\mathfrak{su}(n)$.
\end{enumerate}
\end{enumerate}
\newpage
\begin{flushright}
Nume:\_\_\_\_\_\_\_\_\_\_\_\_\_\_
 
 
Grupa:\_\_\_\_\_\_\_\_\_\_\_\_\_\_
\end{flushright}
\begin{center}
\vspace{2cm}
{\Large Probleme}
\vspace{2cm}
\end{center}
\begin{enumerate}
 \item Fie morfismul $f:\mathbb{R}^4 \to \mathbb{R}^3$ al cărui matrice în raport cu bazele canonice este
$$\begin{pmatrix}
0&0&1&-1\\
2&-1&-2&3\\
0&0&-3&3
\end{pmatrix}$$

\begin{enumerate}
\item Determinați cîte o bază în $Ker(f)$ și $Im(f)$;
\item Fie vectorul $v=(1,3,1,3)$ determinați descompunerea acestuia ca suma dintre un vector din $Ker(f)$ și unul din $(Ker(f))^\perp$;
\item Fie $K$ un corp și fie $L=M_n(K)$. Există $X,Y \in L$ astfel încît $XY-YX=I_n$?  
\end{enumerate}
\item Fie forma pătratică:
$$Q= x_1^2-2x_2^2+x_3^2+4x_1x_2-10x_1x_3+4x_2x_3$$

\begin{enumerate}
\item Aduceți forma pătratică la forma canonică prin metoda Gauss;
\item Aduceți forma pătratică la forma canonică prin transformări ortogonale;
\item Fie $\mathfrak{su}(n)=\{ A \in M_n(\mathbb{C}) | A+\bar{A}^t=0\}$. Arătați că $\mathfrak{su}(n)$ este spațiu vectorial real, dar nu complex.
Determinați $dim_{\mathbb{R}}\mathfrak{su}(n)$.
\end{enumerate}
\end{enumerate}
\newpage
\begin{flushright}
Nume:\_\_\_\_\_\_\_\_\_\_\_\_\_\_
 
 
Grupa:\_\_\_\_\_\_\_\_\_\_\_\_\_\_
\end{flushright}
\begin{center}
\vspace{2cm}
{\Large Probleme}
\vspace{2cm}
\end{center}
\begin{enumerate}
 \item Fie morfismul $f:\mathbb{R}^4 \to \mathbb{R}^3$ al cărui matrice în raport cu bazele canonice este
$$\begin{pmatrix}
0&0&1&-1\\
0&0&3&-3\\
-2&-2&3&-3
\end{pmatrix}$$

\begin{enumerate}
\item Determinați cîte o bază în $Ker(f)$ și $Im(f)$;
\item Fie vectorul $v=(1,3,1,3)$ determinați descompunerea acestuia ca suma dintre un vector din $Ker(f)$ și unul din $(Ker(f))^\perp$;
\item Fie $K$ un corp și fie $L=M_n(K)$. Există $X,Y \in L$ astfel încît $XY-YX=I_n$?  
\end{enumerate}
\item Fie forma pătratică:
$$Q= 2x_1^2+5x_2^2+2x_3^2-4x_1x_2-2x_1x_3+4x_2x_3$$

\begin{enumerate}
\item Aduceți forma pătratică la forma canonică prin metoda Gauss;
\item Aduceți forma pătratică la forma canonică prin transformări ortogonale;
\item Fie $\mathfrak{su}(n)=\{ A \in M_n(\mathbb{C}) | A+\bar{A}^t=0\}$. Arătați că $\mathfrak{su}(n)$ este spațiu vectorial real, dar nu complex.
Determinați $dim_{\mathbb{R}}\mathfrak{su}(n)$.
\end{enumerate}
\end{enumerate}
\newpage
\begin{flushright}
Nume:\_\_\_\_\_\_\_\_\_\_\_\_\_\_
 
 
Grupa:\_\_\_\_\_\_\_\_\_\_\_\_\_\_
\end{flushright}
\begin{center}
\vspace{2cm}
{\Large Probleme}
\vspace{2cm}
\end{center}
\begin{enumerate}
 \item Fie morfismul $f:\mathbb{R}^4 \to \mathbb{R}^3$ al cărui matrice în raport cu bazele canonice este
$$\begin{pmatrix}
0&-2&0&-2\\
3&-2&2&-3\\
-3&-3&-2&-2
\end{pmatrix}$$

\begin{enumerate}
\item Determinați cîte o bază în $Ker(f)$ și $Im(f)$;
\item Fie vectorul $v=(1,3,1,3)$ determinați descompunerea acestuia ca suma dintre un vector din $Ker(f)$ și unul din $(Ker(f))^\perp$;
\item Fie $K$ un corp și fie $L=M_n(K)$. Există $X,Y \in L$ astfel încît $XY-YX=I_n$?  
\end{enumerate}
\item Fie forma pătratică:
$$Q= x_1^2+5x_2^2+x_3^2+2x_1x_2+6x_1x_3+2x_2x_3$$

\begin{enumerate}
\item Aduceți forma pătratică la forma canonică prin metoda Gauss;
\item Aduceți forma pătratică la forma canonică prin transformări ortogonale;
\item Fie $\mathfrak{su}(n)=\{ A \in M_n(\mathbb{C}) | A+\bar{A}^t=0\}$. Arătați că $\mathfrak{su}(n)$ este spațiu vectorial real, dar nu complex.
Determinați $dim_{\mathbb{R}}\mathfrak{su}(n)$.
\end{enumerate}
\end{enumerate}
\newpage
\begin{flushright}
Nume:\_\_\_\_\_\_\_\_\_\_\_\_\_\_
 
 
Grupa:\_\_\_\_\_\_\_\_\_\_\_\_\_\_
\end{flushright}
\begin{center}
\vspace{2cm}
{\Large Probleme}
\vspace{2cm}
\end{center}
\begin{enumerate}
 \item Fie morfismul $f:\mathbb{R}^4 \to \mathbb{R}^3$ al cărui matrice în raport cu bazele canonice este
$$\begin{pmatrix}
0&3&2&2\\
3&0&0&-3\\
1&0&0&-1
\end{pmatrix}$$

\begin{enumerate}
\item Determinați cîte o bază în $Ker(f)$ și $Im(f)$;
\item Fie vectorul $v=(1,3,1,3)$ determinați descompunerea acestuia ca suma dintre un vector din $Ker(f)$ și unul din $(Ker(f))^\perp$;
\item Fie $K$ un corp și fie $L=M_n(K)$. Există $X,Y \in L$ astfel încît $XY-YX=I_n$?  
\end{enumerate}
\item Fie forma pătratică:
$$Q= x_1^2-2x_2^2+x_3^2+4x_1x_2-10x_1x_3+4x_2x_3$$

\begin{enumerate}
\item Aduceți forma pătratică la forma canonică prin metoda Gauss;
\item Aduceți forma pătratică la forma canonică prin transformări ortogonale;
\item Fie $\mathfrak{su}(n)=\{ A \in M_n(\mathbb{C}) | A+\bar{A}^t=0\}$. Arătați că $\mathfrak{su}(n)$ este spațiu vectorial real, dar nu complex.
Determinați $dim_{\mathbb{R}}\mathfrak{su}(n)$.
\end{enumerate}
\end{enumerate}
\newpage
\begin{flushright}
Nume:\_\_\_\_\_\_\_\_\_\_\_\_\_\_
 
 
Grupa:\_\_\_\_\_\_\_\_\_\_\_\_\_\_
\end{flushright}
\begin{center}
\vspace{2cm}
{\Large Probleme}
\vspace{2cm}
\end{center}
\begin{enumerate}
 \item Fie morfismul $f:\mathbb{R}^4 \to \mathbb{R}^3$ al cărui matrice în raport cu bazele canonice este
$$\begin{pmatrix}
0&-2&0&-2\\
1&0&3&-1\\
0&-3&0&-3
\end{pmatrix}$$

\begin{enumerate}
\item Determinați cîte o bază în $Ker(f)$ și $Im(f)$;
\item Fie vectorul $v=(1,3,1,3)$ determinați descompunerea acestuia ca suma dintre un vector din $Ker(f)$ și unul din $(Ker(f))^\perp$;
\item Fie $K$ un corp și fie $L=M_n(K)$. Există $X,Y \in L$ astfel încît $XY-YX=I_n$?  
\end{enumerate}
\item Fie forma pătratică:
$$Q= 2x_1^2+5x_2^2+2x_3^2-4x_1x_2-2x_1x_3+4x_2x_3$$

\begin{enumerate}
\item Aduceți forma pătratică la forma canonică prin metoda Gauss;
\item Aduceți forma pătratică la forma canonică prin transformări ortogonale;
\item Fie $\mathfrak{su}(n)=\{ A \in M_n(\mathbb{C}) | A+\bar{A}^t=0\}$. Arătați că $\mathfrak{su}(n)$ este spațiu vectorial real, dar nu complex.
Determinați $dim_{\mathbb{R}}\mathfrak{su}(n)$.
\end{enumerate}
\end{enumerate}
\newpage
\begin{flushright}
Nume:\_\_\_\_\_\_\_\_\_\_\_\_\_\_
 
 
Grupa:\_\_\_\_\_\_\_\_\_\_\_\_\_\_
\end{flushright}
\begin{center}
\vspace{2cm}
{\Large Probleme}
\vspace{2cm}
\end{center}
\begin{enumerate}
 \item Fie morfismul $f:\mathbb{R}^4 \to \mathbb{R}^3$ al cărui matrice în raport cu bazele canonice este
$$\begin{pmatrix}
0&-2&0&-2\\
3&-1&-1&2\\
-3&2&1&-1
\end{pmatrix}$$

\begin{enumerate}
\item Determinați cîte o bază în $Ker(f)$ și $Im(f)$;
\item Fie vectorul $v=(1,3,1,3)$ determinați descompunerea acestuia ca suma dintre un vector din $Ker(f)$ și unul din $(Ker(f))^\perp$;
\item Fie $K$ un corp și fie $L=M_n(K)$. Există $X,Y \in L$ astfel încît $XY-YX=I_n$?  
\end{enumerate}
\item Fie forma pătratică:
$$Q= x_1^2+5x_2^2+x_3^2+2x_1x_2+6x_1x_3+2x_2x_3$$

\begin{enumerate}
\item Aduceți forma pătratică la forma canonică prin metoda Gauss;
\item Aduceți forma pătratică la forma canonică prin transformări ortogonale;
\item Fie $\mathfrak{su}(n)=\{ A \in M_n(\mathbb{C}) | A+\bar{A}^t=0\}$. Arătați că $\mathfrak{su}(n)$ este spațiu vectorial real, dar nu complex.
Determinați $dim_{\mathbb{R}}\mathfrak{su}(n)$.
\end{enumerate}
\end{enumerate}
\newpage
\begin{flushright}
Nume:\_\_\_\_\_\_\_\_\_\_\_\_\_\_
 
 
Grupa:\_\_\_\_\_\_\_\_\_\_\_\_\_\_
\end{flushright}
\begin{center}
\vspace{2cm}
{\Large Probleme}
\vspace{2cm}
\end{center}
\begin{enumerate}
 \item Fie morfismul $f:\mathbb{R}^4 \to \mathbb{R}^3$ al cărui matrice în raport cu bazele canonice este
$$\begin{pmatrix}
0&-2&0&-2\\
3&-3&-2&-2\\
-3&3&2&2
\end{pmatrix}$$

\begin{enumerate}
\item Determinați cîte o bază în $Ker(f)$ și $Im(f)$;
\item Fie vectorul $v=(1,3,1,3)$ determinați descompunerea acestuia ca suma dintre un vector din $Ker(f)$ și unul din $(Ker(f))^\perp$;
\item Fie $K$ un corp și fie $L=M_n(K)$. Există $X,Y \in L$ astfel încît $XY-YX=I_n$?  
\end{enumerate}
\item Fie forma pătratică:
$$Q= x_1^2-2x_2^2+x_3^2+4x_1x_2-10x_1x_3+4x_2x_3$$

\begin{enumerate}
\item Aduceți forma pătratică la forma canonică prin metoda Gauss;
\item Aduceți forma pătratică la forma canonică prin transformări ortogonale;
\item Fie $\mathfrak{su}(n)=\{ A \in M_n(\mathbb{C}) | A+\bar{A}^t=0\}$. Arătați că $\mathfrak{su}(n)$ este spațiu vectorial real, dar nu complex.
Determinați $dim_{\mathbb{R}}\mathfrak{su}(n)$.
\end{enumerate}
\end{enumerate}
\newpage
\begin{flushright}
Nume:\_\_\_\_\_\_\_\_\_\_\_\_\_\_
 
 
Grupa:\_\_\_\_\_\_\_\_\_\_\_\_\_\_
\end{flushright}
\begin{center}
\vspace{2cm}
{\Large Probleme}
\vspace{2cm}
\end{center}
\begin{enumerate}
 \item Fie morfismul $f:\mathbb{R}^4 \to \mathbb{R}^3$ al cărui matrice în raport cu bazele canonice este
$$\begin{pmatrix}
0&-2&0&-2\\
0&2&0&2\\
-3&-2&-2&-3
\end{pmatrix}$$

\begin{enumerate}
\item Determinați cîte o bază în $Ker(f)$ și $Im(f)$;
\item Fie vectorul $v=(1,3,1,3)$ determinați descompunerea acestuia ca suma dintre un vector din $Ker(f)$ și unul din $(Ker(f))^\perp$;
\item Fie $K$ un corp și fie $L=M_n(K)$. Există $X,Y \in L$ astfel încît $XY-YX=I_n$?  
\end{enumerate}
\item Fie forma pătratică:
$$Q= 2x_1^2+5x_2^2+2x_3^2-4x_1x_2-2x_1x_3+4x_2x_3$$

\begin{enumerate}
\item Aduceți forma pătratică la forma canonică prin metoda Gauss;
\item Aduceți forma pătratică la forma canonică prin transformări ortogonale;
\item Fie $\mathfrak{su}(n)=\{ A \in M_n(\mathbb{C}) | A+\bar{A}^t=0\}$. Arătați că $\mathfrak{su}(n)$ este spațiu vectorial real, dar nu complex.
Determinați $dim_{\mathbb{R}}\mathfrak{su}(n)$.
\end{enumerate}
\end{enumerate}
\newpage
\begin{flushright}
Nume:\_\_\_\_\_\_\_\_\_\_\_\_\_\_
 
 
Grupa:\_\_\_\_\_\_\_\_\_\_\_\_\_\_
\end{flushright}
\begin{center}
\vspace{2cm}
{\Large Probleme}
\vspace{2cm}
\end{center}
\begin{enumerate}
 \item Fie morfismul $f:\mathbb{R}^4 \to \mathbb{R}^3$ al cărui matrice în raport cu bazele canonice este
$$\begin{pmatrix}
2&-2&3&3\\
2&2&-3&2\\
-2&2&-3&-3
\end{pmatrix}$$

\begin{enumerate}
\item Determinați cîte o bază în $Ker(f)$ și $Im(f)$;
\item Fie vectorul $v=(1,3,1,3)$ determinați descompunerea acestuia ca suma dintre un vector din $Ker(f)$ și unul din $(Ker(f))^\perp$;
\item Fie $K$ un corp și fie $L=M_n(K)$. Există $X,Y \in L$ astfel încît $XY-YX=I_n$?  
\end{enumerate}
\item Fie forma pătratică:
$$Q= x_1^2+5x_2^2+x_3^2+2x_1x_2+6x_1x_3+2x_2x_3$$

\begin{enumerate}
\item Aduceți forma pătratică la forma canonică prin metoda Gauss;
\item Aduceți forma pătratică la forma canonică prin transformări ortogonale;
\item Fie $\mathfrak{su}(n)=\{ A \in M_n(\mathbb{C}) | A+\bar{A}^t=0\}$. Arătați că $\mathfrak{su}(n)$ este spațiu vectorial real, dar nu complex.
Determinați $dim_{\mathbb{R}}\mathfrak{su}(n)$.
\end{enumerate}
\end{enumerate}
\newpage
\begin{flushright}
Nume:\_\_\_\_\_\_\_\_\_\_\_\_\_\_
 
 
Grupa:\_\_\_\_\_\_\_\_\_\_\_\_\_\_
\end{flushright}
\begin{center}
\vspace{2cm}
{\Large Probleme}
\vspace{2cm}
\end{center}
\begin{enumerate}
 \item Fie morfismul $f:\mathbb{R}^4 \to \mathbb{R}^3$ al cărui matrice în raport cu bazele canonice este
$$\begin{pmatrix}
0&0&1&-1\\
-1&2&-3&2\\
0&0&-3&3
\end{pmatrix}$$

\begin{enumerate}
\item Determinați cîte o bază în $Ker(f)$ și $Im(f)$;
\item Fie vectorul $v=(1,3,1,3)$ determinați descompunerea acestuia ca suma dintre un vector din $Ker(f)$ și unul din $(Ker(f))^\perp$;
\item Fie $K$ un corp și fie $L=M_n(K)$. Există $X,Y \in L$ astfel încît $XY-YX=I_n$?  
\end{enumerate}
\item Fie forma pătratică:
$$Q= x_1^2-2x_2^2+x_3^2+4x_1x_2-10x_1x_3+4x_2x_3$$

\begin{enumerate}
\item Aduceți forma pătratică la forma canonică prin metoda Gauss;
\item Aduceți forma pătratică la forma canonică prin transformări ortogonale;
\item Fie $\mathfrak{su}(n)=\{ A \in M_n(\mathbb{C}) | A+\bar{A}^t=0\}$. Arătați că $\mathfrak{su}(n)$ este spațiu vectorial real, dar nu complex.
Determinați $dim_{\mathbb{R}}\mathfrak{su}(n)$.
\end{enumerate}
\end{enumerate}
\newpage
\begin{flushright}
Nume:\_\_\_\_\_\_\_\_\_\_\_\_\_\_
 
 
Grupa:\_\_\_\_\_\_\_\_\_\_\_\_\_\_
\end{flushright}
\begin{center}
\vspace{2cm}
{\Large Probleme}
\vspace{2cm}
\end{center}
\begin{enumerate}
 \item Fie morfismul $f:\mathbb{R}^4 \to \mathbb{R}^3$ al cărui matrice în raport cu bazele canonice este
$$\begin{pmatrix}
0&0&1&-1\\
0&0&3&-3\\
-1&1&2&-3
\end{pmatrix}$$

\begin{enumerate}
\item Determinați cîte o bază în $Ker(f)$ și $Im(f)$;
\item Fie vectorul $v=(1,3,1,3)$ determinați descompunerea acestuia ca suma dintre un vector din $Ker(f)$ și unul din $(Ker(f))^\perp$;
\item Fie $K$ un corp și fie $L=M_n(K)$. Există $X,Y \in L$ astfel încît $XY-YX=I_n$?  
\end{enumerate}
\item Fie forma pătratică:
$$Q= 2x_1^2+5x_2^2+2x_3^2-4x_1x_2-2x_1x_3+4x_2x_3$$

\begin{enumerate}
\item Aduceți forma pătratică la forma canonică prin metoda Gauss;
\item Aduceți forma pătratică la forma canonică prin transformări ortogonale;
\item Fie $\mathfrak{su}(n)=\{ A \in M_n(\mathbb{C}) | A+\bar{A}^t=0\}$. Arătați că $\mathfrak{su}(n)$ este spațiu vectorial real, dar nu complex.
Determinați $dim_{\mathbb{R}}\mathfrak{su}(n)$.
\end{enumerate}
\end{enumerate}
\newpage
\begin{flushright}
Nume:\_\_\_\_\_\_\_\_\_\_\_\_\_\_
 
 
Grupa:\_\_\_\_\_\_\_\_\_\_\_\_\_\_
\end{flushright}
\begin{center}
\vspace{2cm}
{\Large Probleme}
\vspace{2cm}
\end{center}
\begin{enumerate}
 \item Fie morfismul $f:\mathbb{R}^4 \to \mathbb{R}^3$ al cărui matrice în raport cu bazele canonice este
$$\begin{pmatrix}
0&0&1&-1\\
3&-2&0&1\\
3&-2&1&0
\end{pmatrix}$$

\begin{enumerate}
\item Determinați cîte o bază în $Ker(f)$ și $Im(f)$;
\item Fie vectorul $v=(1,3,1,3)$ determinați descompunerea acestuia ca suma dintre un vector din $Ker(f)$ și unul din $(Ker(f))^\perp$;
\item Fie $K$ un corp și fie $L=M_n(K)$. Există $X,Y \in L$ astfel încît $XY-YX=I_n$?  
\end{enumerate}
\item Fie forma pătratică:
$$Q= x_1^2+5x_2^2+x_3^2+2x_1x_2+6x_1x_3+2x_2x_3$$

\begin{enumerate}
\item Aduceți forma pătratică la forma canonică prin metoda Gauss;
\item Aduceți forma pătratică la forma canonică prin transformări ortogonale;
\item Fie $\mathfrak{su}(n)=\{ A \in M_n(\mathbb{C}) | A+\bar{A}^t=0\}$. Arătați că $\mathfrak{su}(n)$ este spațiu vectorial real, dar nu complex.
Determinați $dim_{\mathbb{R}}\mathfrak{su}(n)$.
\end{enumerate}
\end{enumerate}
\newpage
\begin{flushright}
Nume:\_\_\_\_\_\_\_\_\_\_\_\_\_\_
 
 
Grupa:\_\_\_\_\_\_\_\_\_\_\_\_\_\_
\end{flushright}
\begin{center}
\vspace{2cm}
{\Large Probleme}
\vspace{2cm}
\end{center}
\begin{enumerate}
 \item Fie morfismul $f:\mathbb{R}^4 \to \mathbb{R}^3$ al cărui matrice în raport cu bazele canonice este
$$\begin{pmatrix}
2&-2&3&3\\
2&2&0&-2\\
-2&2&-3&-3
\end{pmatrix}$$

\begin{enumerate}
\item Determinați cîte o bază în $Ker(f)$ și $Im(f)$;
\item Fie vectorul $v=(1,3,1,3)$ determinați descompunerea acestuia ca suma dintre un vector din $Ker(f)$ și unul din $(Ker(f))^\perp$;
\item Fie $K$ un corp și fie $L=M_n(K)$. Există $X,Y \in L$ astfel încît $XY-YX=I_n$?  
\end{enumerate}
\item Fie forma pătratică:
$$Q= x_1^2-2x_2^2+x_3^2+4x_1x_2-10x_1x_3+4x_2x_3$$

\begin{enumerate}
\item Aduceți forma pătratică la forma canonică prin metoda Gauss;
\item Aduceți forma pătratică la forma canonică prin transformări ortogonale;
\item Fie $\mathfrak{su}(n)=\{ A \in M_n(\mathbb{C}) | A+\bar{A}^t=0\}$. Arătați că $\mathfrak{su}(n)$ este spațiu vectorial real, dar nu complex.
Determinați $dim_{\mathbb{R}}\mathfrak{su}(n)$.
\end{enumerate}
\end{enumerate}
\newpage
\begin{flushright}
Nume:\_\_\_\_\_\_\_\_\_\_\_\_\_\_
 
 
Grupa:\_\_\_\_\_\_\_\_\_\_\_\_\_\_
\end{flushright}
\begin{center}
\vspace{2cm}
{\Large Probleme}
\vspace{2cm}
\end{center}
\begin{enumerate}
 \item Fie morfismul $f:\mathbb{R}^4 \to \mathbb{R}^3$ al cărui matrice în raport cu bazele canonice este
$$\begin{pmatrix}
0&-2&0&-2\\
1&2&-1&3\\
-2&3&2&1
\end{pmatrix}$$

\begin{enumerate}
\item Determinați cîte o bază în $Ker(f)$ și $Im(f)$;
\item Fie vectorul $v=(1,3,1,3)$ determinați descompunerea acestuia ca suma dintre un vector din $Ker(f)$ și unul din $(Ker(f))^\perp$;
\item Fie $K$ un corp și fie $L=M_n(K)$. Există $X,Y \in L$ astfel încît $XY-YX=I_n$?  
\end{enumerate}
\item Fie forma pătratică:
$$Q= 2x_1^2+5x_2^2+2x_3^2-4x_1x_2-2x_1x_3+4x_2x_3$$

\begin{enumerate}
\item Aduceți forma pătratică la forma canonică prin metoda Gauss;
\item Aduceți forma pătratică la forma canonică prin transformări ortogonale;
\item Fie $\mathfrak{su}(n)=\{ A \in M_n(\mathbb{C}) | A+\bar{A}^t=0\}$. Arătați că $\mathfrak{su}(n)$ este spațiu vectorial real, dar nu complex.
Determinați $dim_{\mathbb{R}}\mathfrak{su}(n)$.
\end{enumerate}
\end{enumerate}
\newpage
\begin{flushright}
Nume:\_\_\_\_\_\_\_\_\_\_\_\_\_\_
 
 
Grupa:\_\_\_\_\_\_\_\_\_\_\_\_\_\_
\end{flushright}
\begin{center}
\vspace{2cm}
{\Large Probleme}
\vspace{2cm}
\end{center}
\begin{enumerate}
 \item Fie morfismul $f:\mathbb{R}^4 \to \mathbb{R}^3$ al cărui matrice în raport cu bazele canonice este
$$\begin{pmatrix}
0&0&1&-1\\
1&-3&0&1\\
0&0&2&-2
\end{pmatrix}$$

\begin{enumerate}
\item Determinați cîte o bază în $Ker(f)$ și $Im(f)$;
\item Fie vectorul $v=(1,3,1,3)$ determinați descompunerea acestuia ca suma dintre un vector din $Ker(f)$ și unul din $(Ker(f))^\perp$;
\item Fie $K$ un corp și fie $L=M_n(K)$. Există $X,Y \in L$ astfel încît $XY-YX=I_n$?  
\end{enumerate}
\item Fie forma pătratică:
$$Q= x_1^2+5x_2^2+x_3^2+2x_1x_2+6x_1x_3+2x_2x_3$$

\begin{enumerate}
\item Aduceți forma pătratică la forma canonică prin metoda Gauss;
\item Aduceți forma pătratică la forma canonică prin transformări ortogonale;
\item Fie $\mathfrak{su}(n)=\{ A \in M_n(\mathbb{C}) | A+\bar{A}^t=0\}$. Arătați că $\mathfrak{su}(n)$ este spațiu vectorial real, dar nu complex.
Determinați $dim_{\mathbb{R}}\mathfrak{su}(n)$.
\end{enumerate}
\end{enumerate}
\newpage
\begin{flushright}
Nume:\_\_\_\_\_\_\_\_\_\_\_\_\_\_
 
 
Grupa:\_\_\_\_\_\_\_\_\_\_\_\_\_\_
\end{flushright}
\begin{center}
\vspace{2cm}
{\Large Probleme}
\vspace{2cm}
\end{center}
\begin{enumerate}
 \item Fie morfismul $f:\mathbb{R}^4 \to \mathbb{R}^3$ al cărui matrice în raport cu bazele canonice este
$$\begin{pmatrix}
0&-2&0&-2\\
-3&2&1&2\\
3&3&-1&3
\end{pmatrix}$$

\begin{enumerate}
\item Determinați cîte o bază în $Ker(f)$ și $Im(f)$;
\item Fie vectorul $v=(1,3,1,3)$ determinați descompunerea acestuia ca suma dintre un vector din $Ker(f)$ și unul din $(Ker(f))^\perp$;
\item Fie $K$ un corp și fie $L=M_n(K)$. Există $X,Y \in L$ astfel încît $XY-YX=I_n$?  
\end{enumerate}
\item Fie forma pătratică:
$$Q= x_1^2-2x_2^2+x_3^2+4x_1x_2-10x_1x_3+4x_2x_3$$

\begin{enumerate}
\item Aduceți forma pătratică la forma canonică prin metoda Gauss;
\item Aduceți forma pătratică la forma canonică prin transformări ortogonale;
\item Fie $\mathfrak{su}(n)=\{ A \in M_n(\mathbb{C}) | A+\bar{A}^t=0\}$. Arătați că $\mathfrak{su}(n)$ este spațiu vectorial real, dar nu complex.
Determinați $dim_{\mathbb{R}}\mathfrak{su}(n)$.
\end{enumerate}
\end{enumerate}
\newpage
\begin{flushright}
Nume:\_\_\_\_\_\_\_\_\_\_\_\_\_\_
 
 
Grupa:\_\_\_\_\_\_\_\_\_\_\_\_\_\_
\end{flushright}
\begin{center}
\vspace{2cm}
{\Large Probleme}
\vspace{2cm}
\end{center}
\begin{enumerate}
 \item Fie morfismul $f:\mathbb{R}^4 \to \mathbb{R}^3$ al cărui matrice în raport cu bazele canonice este
$$\begin{pmatrix}
0&-2&0&-2\\
0&2&0&2\\
-3&-1&2&0
\end{pmatrix}$$

\begin{enumerate}
\item Determinați cîte o bază în $Ker(f)$ și $Im(f)$;
\item Fie vectorul $v=(1,3,1,3)$ determinați descompunerea acestuia ca suma dintre un vector din $Ker(f)$ și unul din $(Ker(f))^\perp$;
\item Fie $K$ un corp și fie $L=M_n(K)$. Există $X,Y \in L$ astfel încît $XY-YX=I_n$?  
\end{enumerate}
\item Fie forma pătratică:
$$Q= 2x_1^2+5x_2^2+2x_3^2-4x_1x_2-2x_1x_3+4x_2x_3$$

\begin{enumerate}
\item Aduceți forma pătratică la forma canonică prin metoda Gauss;
\item Aduceți forma pătratică la forma canonică prin transformări ortogonale;
\item Fie $\mathfrak{su}(n)=\{ A \in M_n(\mathbb{C}) | A+\bar{A}^t=0\}$. Arătați că $\mathfrak{su}(n)$ este spațiu vectorial real, dar nu complex.
Determinați $dim_{\mathbb{R}}\mathfrak{su}(n)$.
\end{enumerate}
\end{enumerate}
\newpage
\begin{flushright}
Nume:\_\_\_\_\_\_\_\_\_\_\_\_\_\_
 
 
Grupa:\_\_\_\_\_\_\_\_\_\_\_\_\_\_
\end{flushright}
\begin{center}
\vspace{2cm}
{\Large Probleme}
\vspace{2cm}
\end{center}
\begin{enumerate}
 \item Fie morfismul $f:\mathbb{R}^4 \to \mathbb{R}^3$ al cărui matrice în raport cu bazele canonice este
$$\begin{pmatrix}
0&-2&0&-2\\
-3&-3&1&-2\\
-3&2&1&3
\end{pmatrix}$$

\begin{enumerate}
\item Determinați cîte o bază în $Ker(f)$ și $Im(f)$;
\item Fie vectorul $v=(1,3,1,3)$ determinați descompunerea acestuia ca suma dintre un vector din $Ker(f)$ și unul din $(Ker(f))^\perp$;
\item Fie $K$ un corp și fie $L=M_n(K)$. Există $X,Y \in L$ astfel încît $XY-YX=I_n$?  
\end{enumerate}
\item Fie forma pătratică:
$$Q= x_1^2+5x_2^2+x_3^2+2x_1x_2+6x_1x_3+2x_2x_3$$

\begin{enumerate}
\item Aduceți forma pătratică la forma canonică prin metoda Gauss;
\item Aduceți forma pătratică la forma canonică prin transformări ortogonale;
\item Fie $\mathfrak{su}(n)=\{ A \in M_n(\mathbb{C}) | A+\bar{A}^t=0\}$. Arătați că $\mathfrak{su}(n)$ este spațiu vectorial real, dar nu complex.
Determinați $dim_{\mathbb{R}}\mathfrak{su}(n)$.
\end{enumerate}
\end{enumerate}
\newpage
\begin{flushright}
Nume:\_\_\_\_\_\_\_\_\_\_\_\_\_\_
 
 
Grupa:\_\_\_\_\_\_\_\_\_\_\_\_\_\_
\end{flushright}
\begin{center}
\vspace{2cm}
{\Large Probleme}
\vspace{2cm}
\end{center}
\begin{enumerate}
 \item Fie morfismul $f:\mathbb{R}^4 \to \mathbb{R}^3$ al cărui matrice în raport cu bazele canonice este
$$\begin{pmatrix}
0&-2&0&-2\\
2&0&-2&-2\\
0&-3&0&-3
\end{pmatrix}$$

\begin{enumerate}
\item Determinați cîte o bază în $Ker(f)$ și $Im(f)$;
\item Fie vectorul $v=(1,3,1,3)$ determinați descompunerea acestuia ca suma dintre un vector din $Ker(f)$ și unul din $(Ker(f))^\perp$;
\item Fie $K$ un corp și fie $L=M_n(K)$. Există $X,Y \in L$ astfel încît $XY-YX=I_n$?  
\end{enumerate}
\item Fie forma pătratică:
$$Q= x_1^2-2x_2^2+x_3^2+4x_1x_2-10x_1x_3+4x_2x_3$$

\begin{enumerate}
\item Aduceți forma pătratică la forma canonică prin metoda Gauss;
\item Aduceți forma pătratică la forma canonică prin transformări ortogonale;
\item Fie $\mathfrak{su}(n)=\{ A \in M_n(\mathbb{C}) | A+\bar{A}^t=0\}$. Arătați că $\mathfrak{su}(n)$ este spațiu vectorial real, dar nu complex.
Determinați $dim_{\mathbb{R}}\mathfrak{su}(n)$.
\end{enumerate}
\end{enumerate}
\newpage
\begin{flushright}
Nume:\_\_\_\_\_\_\_\_\_\_\_\_\_\_
 
 
Grupa:\_\_\_\_\_\_\_\_\_\_\_\_\_\_
\end{flushright}
\begin{center}
\vspace{2cm}
{\Large Probleme}
\vspace{2cm}
\end{center}
\begin{enumerate}
 \item Fie morfismul $f:\mathbb{R}^4 \to \mathbb{R}^3$ al cărui matrice în raport cu bazele canonice este
$$\begin{pmatrix}
0&0&1&-1\\
3&-2&0&3\\
-3&2&-2&-1
\end{pmatrix}$$

\begin{enumerate}
\item Determinați cîte o bază în $Ker(f)$ și $Im(f)$;
\item Fie vectorul $v=(1,3,1,3)$ determinați descompunerea acestuia ca suma dintre un vector din $Ker(f)$ și unul din $(Ker(f))^\perp$;
\item Fie $K$ un corp și fie $L=M_n(K)$. Există $X,Y \in L$ astfel încît $XY-YX=I_n$?  
\end{enumerate}
\item Fie forma pătratică:
$$Q= 2x_1^2+5x_2^2+2x_3^2-4x_1x_2-2x_1x_3+4x_2x_3$$

\begin{enumerate}
\item Aduceți forma pătratică la forma canonică prin metoda Gauss;
\item Aduceți forma pătratică la forma canonică prin transformări ortogonale;
\item Fie $\mathfrak{su}(n)=\{ A \in M_n(\mathbb{C}) | A+\bar{A}^t=0\}$. Arătați că $\mathfrak{su}(n)$ este spațiu vectorial real, dar nu complex.
Determinați $dim_{\mathbb{R}}\mathfrak{su}(n)$.
\end{enumerate}
\end{enumerate}
\newpage
\begin{flushright}
Nume:\_\_\_\_\_\_\_\_\_\_\_\_\_\_
 
 
Grupa:\_\_\_\_\_\_\_\_\_\_\_\_\_\_
\end{flushright}
\begin{center}
\vspace{2cm}
{\Large Probleme}
\vspace{2cm}
\end{center}
\begin{enumerate}
 \item Fie morfismul $f:\mathbb{R}^4 \to \mathbb{R}^3$ al cărui matrice în raport cu bazele canonice este
$$\begin{pmatrix}
0&-2&0&-2\\
0&-3&0&-3\\
-2&3&-3&3
\end{pmatrix}$$

\begin{enumerate}
\item Determinați cîte o bază în $Ker(f)$ și $Im(f)$;
\item Fie vectorul $v=(1,3,1,3)$ determinați descompunerea acestuia ca suma dintre un vector din $Ker(f)$ și unul din $(Ker(f))^\perp$;
\item Fie $K$ un corp și fie $L=M_n(K)$. Există $X,Y \in L$ astfel încît $XY-YX=I_n$?  
\end{enumerate}
\item Fie forma pătratică:
$$Q= x_1^2+5x_2^2+x_3^2+2x_1x_2+6x_1x_3+2x_2x_3$$

\begin{enumerate}
\item Aduceți forma pătratică la forma canonică prin metoda Gauss;
\item Aduceți forma pătratică la forma canonică prin transformări ortogonale;
\item Fie $\mathfrak{su}(n)=\{ A \in M_n(\mathbb{C}) | A+\bar{A}^t=0\}$. Arătați că $\mathfrak{su}(n)$ este spațiu vectorial real, dar nu complex.
Determinați $dim_{\mathbb{R}}\mathfrak{su}(n)$.
\end{enumerate}
\end{enumerate}
\newpage
\begin{flushright}
Nume:\_\_\_\_\_\_\_\_\_\_\_\_\_\_
 
 
Grupa:\_\_\_\_\_\_\_\_\_\_\_\_\_\_
\end{flushright}
\begin{center}
\vspace{2cm}
{\Large Probleme}
\vspace{2cm}
\end{center}
\begin{enumerate}
 \item Fie morfismul $f:\mathbb{R}^4 \to \mathbb{R}^3$ al cărui matrice în raport cu bazele canonice este
$$\begin{pmatrix}
0&-2&0&-2\\
-3&-1&-3&2\\
2&-1&2&-3
\end{pmatrix}$$

\begin{enumerate}
\item Determinați cîte o bază în $Ker(f)$ și $Im(f)$;
\item Fie vectorul $v=(1,3,1,3)$ determinați descompunerea acestuia ca suma dintre un vector din $Ker(f)$ și unul din $(Ker(f))^\perp$;
\item Fie $K$ un corp și fie $L=M_n(K)$. Există $X,Y \in L$ astfel încît $XY-YX=I_n$?  
\end{enumerate}
\item Fie forma pătratică:
$$Q= x_1^2-2x_2^2+x_3^2+4x_1x_2-10x_1x_3+4x_2x_3$$

\begin{enumerate}
\item Aduceți forma pătratică la forma canonică prin metoda Gauss;
\item Aduceți forma pătratică la forma canonică prin transformări ortogonale;
\item Fie $\mathfrak{su}(n)=\{ A \in M_n(\mathbb{C}) | A+\bar{A}^t=0\}$. Arătați că $\mathfrak{su}(n)$ este spațiu vectorial real, dar nu complex.
Determinați $dim_{\mathbb{R}}\mathfrak{su}(n)$.
\end{enumerate}
\end{enumerate}
\newpage
\begin{flushright}
Nume:\_\_\_\_\_\_\_\_\_\_\_\_\_\_
 
 
Grupa:\_\_\_\_\_\_\_\_\_\_\_\_\_\_
\end{flushright}
\begin{center}
\vspace{2cm}
{\Large Probleme}
\vspace{2cm}
\end{center}
\begin{enumerate}
 \item Fie morfismul $f:\mathbb{R}^4 \to \mathbb{R}^3$ al cărui matrice în raport cu bazele canonice este
$$\begin{pmatrix}
2&-2&3&3\\
2&-1&-2&-2\\
-2&1&2&2
\end{pmatrix}$$

\begin{enumerate}
\item Determinați cîte o bază în $Ker(f)$ și $Im(f)$;
\item Fie vectorul $v=(1,3,1,3)$ determinați descompunerea acestuia ca suma dintre un vector din $Ker(f)$ și unul din $(Ker(f))^\perp$;
\item Fie $K$ un corp și fie $L=M_n(K)$. Există $X,Y \in L$ astfel încît $XY-YX=I_n$?  
\end{enumerate}
\item Fie forma pătratică:
$$Q= 2x_1^2+5x_2^2+2x_3^2-4x_1x_2-2x_1x_3+4x_2x_3$$

\begin{enumerate}
\item Aduceți forma pătratică la forma canonică prin metoda Gauss;
\item Aduceți forma pătratică la forma canonică prin transformări ortogonale;
\item Fie $\mathfrak{su}(n)=\{ A \in M_n(\mathbb{C}) | A+\bar{A}^t=0\}$. Arătați că $\mathfrak{su}(n)$ este spațiu vectorial real, dar nu complex.
Determinați $dim_{\mathbb{R}}\mathfrak{su}(n)$.
\end{enumerate}
\end{enumerate}
\newpage
\begin{flushright}
Nume:\_\_\_\_\_\_\_\_\_\_\_\_\_\_
 
 
Grupa:\_\_\_\_\_\_\_\_\_\_\_\_\_\_
\end{flushright}
\begin{center}
\vspace{2cm}
{\Large Probleme}
\vspace{2cm}
\end{center}
\begin{enumerate}
 \item Fie morfismul $f:\mathbb{R}^4 \to \mathbb{R}^3$ al cărui matrice în raport cu bazele canonice este
$$\begin{pmatrix}
0&-2&0&-2\\
-1&1&-2&1\\
0&2&0&2
\end{pmatrix}$$

\begin{enumerate}
\item Determinați cîte o bază în $Ker(f)$ și $Im(f)$;
\item Fie vectorul $v=(1,3,1,3)$ determinați descompunerea acestuia ca suma dintre un vector din $Ker(f)$ și unul din $(Ker(f))^\perp$;
\item Fie $K$ un corp și fie $L=M_n(K)$. Există $X,Y \in L$ astfel încît $XY-YX=I_n$?  
\end{enumerate}
\item Fie forma pătratică:
$$Q= x_1^2+5x_2^2+x_3^2+2x_1x_2+6x_1x_3+2x_2x_3$$

\begin{enumerate}
\item Aduceți forma pătratică la forma canonică prin metoda Gauss;
\item Aduceți forma pătratică la forma canonică prin transformări ortogonale;
\item Fie $\mathfrak{su}(n)=\{ A \in M_n(\mathbb{C}) | A+\bar{A}^t=0\}$. Arătați că $\mathfrak{su}(n)$ este spațiu vectorial real, dar nu complex.
Determinați $dim_{\mathbb{R}}\mathfrak{su}(n)$.
\end{enumerate}
\end{enumerate}
\newpage
\begin{flushright}
Nume:\_\_\_\_\_\_\_\_\_\_\_\_\_\_
 
 
Grupa:\_\_\_\_\_\_\_\_\_\_\_\_\_\_
\end{flushright}
\begin{center}
\vspace{2cm}
{\Large Probleme}
\vspace{2cm}
\end{center}
\begin{enumerate}
 \item Fie morfismul $f:\mathbb{R}^4 \to \mathbb{R}^3$ al cărui matrice în raport cu bazele canonice este
$$\begin{pmatrix}
0&0&1&-1\\
1&1&2&0\\
3&3&3&3
\end{pmatrix}$$

\begin{enumerate}
\item Determinați cîte o bază în $Ker(f)$ și $Im(f)$;
\item Fie vectorul $v=(1,3,1,3)$ determinați descompunerea acestuia ca suma dintre un vector din $Ker(f)$ și unul din $(Ker(f))^\perp$;
\item Fie $K$ un corp și fie $L=M_n(K)$. Există $X,Y \in L$ astfel încît $XY-YX=I_n$?  
\end{enumerate}
\item Fie forma pătratică:
$$Q= x_1^2-2x_2^2+x_3^2+4x_1x_2-10x_1x_3+4x_2x_3$$

\begin{enumerate}
\item Aduceți forma pătratică la forma canonică prin metoda Gauss;
\item Aduceți forma pătratică la forma canonică prin transformări ortogonale;
\item Fie $\mathfrak{su}(n)=\{ A \in M_n(\mathbb{C}) | A+\bar{A}^t=0\}$. Arătați că $\mathfrak{su}(n)$ este spațiu vectorial real, dar nu complex.
Determinați $dim_{\mathbb{R}}\mathfrak{su}(n)$.
\end{enumerate}
\end{enumerate}
\newpage
\begin{flushright}
Nume:\_\_\_\_\_\_\_\_\_\_\_\_\_\_
 
 
Grupa:\_\_\_\_\_\_\_\_\_\_\_\_\_\_
\end{flushright}
\begin{center}
\vspace{2cm}
{\Large Probleme}
\vspace{2cm}
\end{center}
\begin{enumerate}
 \item Fie morfismul $f:\mathbb{R}^4 \to \mathbb{R}^3$ al cărui matrice în raport cu bazele canonice este
$$\begin{pmatrix}
0&-2&0&-2\\
3&-2&1&-3\\
0&3&0&3
\end{pmatrix}$$

\begin{enumerate}
\item Determinați cîte o bază în $Ker(f)$ și $Im(f)$;
\item Fie vectorul $v=(1,3,1,3)$ determinați descompunerea acestuia ca suma dintre un vector din $Ker(f)$ și unul din $(Ker(f))^\perp$;
\item Fie $K$ un corp și fie $L=M_n(K)$. Există $X,Y \in L$ astfel încît $XY-YX=I_n$?  
\end{enumerate}
\item Fie forma pătratică:
$$Q= 2x_1^2+5x_2^2+2x_3^2-4x_1x_2-2x_1x_3+4x_2x_3$$

\begin{enumerate}
\item Aduceți forma pătratică la forma canonică prin metoda Gauss;
\item Aduceți forma pătratică la forma canonică prin transformări ortogonale;
\item Fie $\mathfrak{su}(n)=\{ A \in M_n(\mathbb{C}) | A+\bar{A}^t=0\}$. Arătați că $\mathfrak{su}(n)$ este spațiu vectorial real, dar nu complex.
Determinați $dim_{\mathbb{R}}\mathfrak{su}(n)$.
\end{enumerate}
\end{enumerate}
\newpage
\begin{flushright}
Nume:\_\_\_\_\_\_\_\_\_\_\_\_\_\_
 
 
Grupa:\_\_\_\_\_\_\_\_\_\_\_\_\_\_
\end{flushright}
\begin{center}
\vspace{2cm}
{\Large Probleme}
\vspace{2cm}
\end{center}
\begin{enumerate}
 \item Fie morfismul $f:\mathbb{R}^4 \to \mathbb{R}^3$ al cărui matrice în raport cu bazele canonice este
$$\begin{pmatrix}
0&-2&0&-2\\
-2&-1&-1&2\\
2&3&1&0
\end{pmatrix}$$

\begin{enumerate}
\item Determinați cîte o bază în $Ker(f)$ și $Im(f)$;
\item Fie vectorul $v=(1,3,1,3)$ determinați descompunerea acestuia ca suma dintre un vector din $Ker(f)$ și unul din $(Ker(f))^\perp$;
\item Fie $K$ un corp și fie $L=M_n(K)$. Există $X,Y \in L$ astfel încît $XY-YX=I_n$?  
\end{enumerate}
\item Fie forma pătratică:
$$Q= x_1^2+5x_2^2+x_3^2+2x_1x_2+6x_1x_3+2x_2x_3$$

\begin{enumerate}
\item Aduceți forma pătratică la forma canonică prin metoda Gauss;
\item Aduceți forma pătratică la forma canonică prin transformări ortogonale;
\item Fie $\mathfrak{su}(n)=\{ A \in M_n(\mathbb{C}) | A+\bar{A}^t=0\}$. Arătați că $\mathfrak{su}(n)$ este spațiu vectorial real, dar nu complex.
Determinați $dim_{\mathbb{R}}\mathfrak{su}(n)$.
\end{enumerate}
\end{enumerate}
\newpage
\begin{flushright}
Nume:\_\_\_\_\_\_\_\_\_\_\_\_\_\_
 
 
Grupa:\_\_\_\_\_\_\_\_\_\_\_\_\_\_
\end{flushright}
\begin{center}
\vspace{2cm}
{\Large Probleme}
\vspace{2cm}
\end{center}
\begin{enumerate}
 \item Fie morfismul $f:\mathbb{R}^4 \to \mathbb{R}^3$ al cărui matrice în raport cu bazele canonice este
$$\begin{pmatrix}
0&0&1&-1\\
3&1&3&-2\\
3&1&2&-1
\end{pmatrix}$$

\begin{enumerate}
\item Determinați cîte o bază în $Ker(f)$ și $Im(f)$;
\item Fie vectorul $v=(1,3,1,3)$ determinați descompunerea acestuia ca suma dintre un vector din $Ker(f)$ și unul din $(Ker(f))^\perp$;
\item Fie $K$ un corp și fie $L=M_n(K)$. Există $X,Y \in L$ astfel încît $XY-YX=I_n$?  
\end{enumerate}
\item Fie forma pătratică:
$$Q= x_1^2-2x_2^2+x_3^2+4x_1x_2-10x_1x_3+4x_2x_3$$

\begin{enumerate}
\item Aduceți forma pătratică la forma canonică prin metoda Gauss;
\item Aduceți forma pătratică la forma canonică prin transformări ortogonale;
\item Fie $\mathfrak{su}(n)=\{ A \in M_n(\mathbb{C}) | A+\bar{A}^t=0\}$. Arătați că $\mathfrak{su}(n)$ este spațiu vectorial real, dar nu complex.
Determinați $dim_{\mathbb{R}}\mathfrak{su}(n)$.
\end{enumerate}
\end{enumerate}
\newpage
\begin{flushright}
Nume:\_\_\_\_\_\_\_\_\_\_\_\_\_\_
 
 
Grupa:\_\_\_\_\_\_\_\_\_\_\_\_\_\_
\end{flushright}
\begin{center}
\vspace{2cm}
{\Large Probleme}
\vspace{2cm}
\end{center}
\begin{enumerate}
 \item Fie morfismul $f:\mathbb{R}^4 \to \mathbb{R}^3$ al cărui matrice în raport cu bazele canonice este
$$\begin{pmatrix}
2&-2&3&3\\
2&-2&2&0\\
-2&2&-3&-3
\end{pmatrix}$$

\begin{enumerate}
\item Determinați cîte o bază în $Ker(f)$ și $Im(f)$;
\item Fie vectorul $v=(1,3,1,3)$ determinați descompunerea acestuia ca suma dintre un vector din $Ker(f)$ și unul din $(Ker(f))^\perp$;
\item Fie $K$ un corp și fie $L=M_n(K)$. Există $X,Y \in L$ astfel încît $XY-YX=I_n$?  
\end{enumerate}
\item Fie forma pătratică:
$$Q= 2x_1^2+5x_2^2+2x_3^2-4x_1x_2-2x_1x_3+4x_2x_3$$

\begin{enumerate}
\item Aduceți forma pătratică la forma canonică prin metoda Gauss;
\item Aduceți forma pătratică la forma canonică prin transformări ortogonale;
\item Fie $\mathfrak{su}(n)=\{ A \in M_n(\mathbb{C}) | A+\bar{A}^t=0\}$. Arătați că $\mathfrak{su}(n)$ este spațiu vectorial real, dar nu complex.
Determinați $dim_{\mathbb{R}}\mathfrak{su}(n)$.
\end{enumerate}
\end{enumerate}
\newpage
\begin{flushright}
Nume:\_\_\_\_\_\_\_\_\_\_\_\_\_\_
 
 
Grupa:\_\_\_\_\_\_\_\_\_\_\_\_\_\_
\end{flushright}
\begin{center}
\vspace{2cm}
{\Large Probleme}
\vspace{2cm}
\end{center}
\begin{enumerate}
 \item Fie morfismul $f:\mathbb{R}^4 \to \mathbb{R}^3$ al cărui matrice în raport cu bazele canonice este
$$\begin{pmatrix}
0&-2&0&-2\\
3&3&-3&-2\\
3&2&-3&-3
\end{pmatrix}$$

\begin{enumerate}
\item Determinați cîte o bază în $Ker(f)$ și $Im(f)$;
\item Fie vectorul $v=(1,3,1,3)$ determinați descompunerea acestuia ca suma dintre un vector din $Ker(f)$ și unul din $(Ker(f))^\perp$;
\item Fie $K$ un corp și fie $L=M_n(K)$. Există $X,Y \in L$ astfel încît $XY-YX=I_n$?  
\end{enumerate}
\item Fie forma pătratică:
$$Q= x_1^2+5x_2^2+x_3^2+2x_1x_2+6x_1x_3+2x_2x_3$$

\begin{enumerate}
\item Aduceți forma pătratică la forma canonică prin metoda Gauss;
\item Aduceți forma pătratică la forma canonică prin transformări ortogonale;
\item Fie $\mathfrak{su}(n)=\{ A \in M_n(\mathbb{C}) | A+\bar{A}^t=0\}$. Arătați că $\mathfrak{su}(n)$ este spațiu vectorial real, dar nu complex.
Determinați $dim_{\mathbb{R}}\mathfrak{su}(n)$.
\end{enumerate}
\end{enumerate}
\newpage
\end{document}
