\documentclass{article}
\usepackage[T1]{fontenc}
\usepackage[utf8]{inputenc}
\usepackage{lmodern}
\usepackage{nopageno}
\usepackage{fontspec}
\usepackage{amsmath}
\usepackage{amsfonts}
\usepackage{amssymb}
\title{Probleme}
\author{ }
\date{ }

\begin{document}
\begin{flushright}
Nume:\_\_\_\_\_\_\_\_\_\_\_\_\_\_
 
 
Grupa:\_\_\_\_\_\_\_\_\_\_\_\_\_\_
\end{flushright}
\begin{center}
\vspace{2cm}
{\Large Probleme}
\vspace{2cm}
\end{center}
\begin{enumerate}
 \item Fie morfismul $f:\mathbb{R}^5 \to \mathbb{R}^4$ al cărui matrice în raport cu bazele canonice este
$$\begin{pmatrix}
-2&-2&1&-1&-2\\
-2&1&1&-2&-1\\
2&2&-1&-1&1\\
-2&-1&1&2&0
\end{pmatrix}$$

\begin{enumerate}
\item Determinați cîte o bază în $Ker(f)$ și $Im(f)$;
\item Fie vectorul $v=(1,3,1,3)$ determinați descompunerea acestuia ca suma dintre un vector din $Im(f)$ și unul din $Im(f)^\perp$;
\item Fie $K$ un corp și fie $L=M_n(K)$. Arătați că pentru orice funcțională $f \in L^*$ există o matrice $A$ astfel încît $f(X)=Tr(AX)$;
\end{enumerate}
\item Fie forma pătratică:
$$Q= 2x_1^2+5x_2^2+2x_3^2-4x_1x_2-2x_1x_3+4x_2x_3$$

\begin{enumerate}
\item Aduceți forma pătratică la forma canonică prin metoda Gauss;
\item Aduceți forma pătratică la forma canonică prin transformări ortogonale;
\item Determinați forma biliniară simetrică $B$ asociată lui $Q$ și calculați dimensiunea subspațiului
$$\{x \in \mathbb{R}^3 | B(X,y)=0,\forall y \in \mathbb{R}^3\}.$$

\end{enumerate}
\end{enumerate}
\newpage
\begin{flushright}
Nume:\_\_\_\_\_\_\_\_\_\_\_\_\_\_
 
 
Grupa:\_\_\_\_\_\_\_\_\_\_\_\_\_\_
\end{flushright}
\begin{center}
\vspace{2cm}
{\Large Probleme}
\vspace{2cm}
\end{center}
\begin{enumerate}
 \item Fie morfismul $f:\mathbb{R}^5 \to \mathbb{R}^4$ al cărui matrice în raport cu bazele canonice este
$$\begin{pmatrix}
-2&-2&1&-1&-2\\
-2&1&1&-2&-1\\
2&2&-1&1&2\\
-2&1&-1&-1&2
\end{pmatrix}$$

\begin{enumerate}
\item Determinați cîte o bază în $Ker(f)$ și $Im(f)$;
\item Fie vectorul $v=(1,3,1,3)$ determinați descompunerea acestuia ca suma dintre un vector din $Im(f)$ și unul din $Im(f)^\perp$;
\item Fie $K$ un corp și fie $L=M_n(K)$. Arătați că pentru orice funcțională $f \in L^*$ există o matrice $A$ astfel încît $f(X)=Tr(AX)$;
\end{enumerate}
\item Fie forma pătratică:
$$Q= -x_1^2+x_2^2-5x_3^2+3x_1x_3+4x_2x_3$$

\begin{enumerate}
\item Aduceți forma pătratică la forma canonică prin metoda Gauss;
\item Aduceți forma pătratică la forma canonică prin transformări ortogonale;
\item Determinați forma biliniară simetrică $B$ asociată lui $Q$ și calculați dimensiunea subspațiului
$$\{x \in \mathbb{R}^3 | B(X,y)=0,\forall y \in \mathbb{R}^3\}.$$

\end{enumerate}
\end{enumerate}
\newpage
\begin{flushright}
Nume:\_\_\_\_\_\_\_\_\_\_\_\_\_\_
 
 
Grupa:\_\_\_\_\_\_\_\_\_\_\_\_\_\_
\end{flushright}
\begin{center}
\vspace{2cm}
{\Large Probleme}
\vspace{2cm}
\end{center}
\begin{enumerate}
 \item Fie morfismul $f:\mathbb{R}^5 \to \mathbb{R}^4$ al cărui matrice în raport cu bazele canonice este
$$\begin{pmatrix}
-2&-2&1&-1&-2\\
-2&1&1&-2&-1\\
2&2&-1&1&2\\
1&-2&1&1&1
\end{pmatrix}$$

\begin{enumerate}
\item Determinați cîte o bază în $Ker(f)$ și $Im(f)$;
\item Fie vectorul $v=(1,3,1,3)$ determinați descompunerea acestuia ca suma dintre un vector din $Im(f)$ și unul din $Im(f)^\perp$;
\item Fie $K$ un corp și fie $L=M_n(K)$. Arătați că pentru orice funcțională $f \in L^*$ există o matrice $A$ astfel încît $f(X)=Tr(AX)$;
\end{enumerate}
\item Fie forma pătratică:
$$Q= x_1^2+5x_2^2+x_3^2+2x_1x_2+6X_1x_3+2x_2x_3$$

\begin{enumerate}
\item Aduceți forma pătratică la forma canonică prin metoda Gauss;
\item Aduceți forma pătratică la forma canonică prin transformări ortogonale;
\item Determinați forma biliniară simetrică $B$ asociată lui $Q$ și calculați dimensiunea subspațiului
$$\{x \in \mathbb{R}^3 | B(X,y)=0,\forall y \in \mathbb{R}^3\}.$$

\end{enumerate}
\end{enumerate}
\newpage
\begin{flushright}
Nume:\_\_\_\_\_\_\_\_\_\_\_\_\_\_
 
 
Grupa:\_\_\_\_\_\_\_\_\_\_\_\_\_\_
\end{flushright}
\begin{center}
\vspace{2cm}
{\Large Probleme}
\vspace{2cm}
\end{center}
\begin{enumerate}
 \item Fie morfismul $f:\mathbb{R}^5 \to \mathbb{R}^4$ al cărui matrice în raport cu bazele canonice este
$$\begin{pmatrix}
-2&-2&1&-1&-2\\
-2&1&1&-2&-1\\
-1&2&2&-1&-2\\
0&-2&-2&0&2
\end{pmatrix}$$

\begin{enumerate}
\item Determinați cîte o bază în $Ker(f)$ și $Im(f)$;
\item Fie vectorul $v=(1,3,1,3)$ determinați descompunerea acestuia ca suma dintre un vector din $Im(f)$ și unul din $Im(f)^\perp$;
\item Fie $K$ un corp și fie $L=M_n(K)$. Arătați că pentru orice funcțională $f \in L^*$ există o matrice $A$ astfel încît $f(X)=Tr(AX)$;
\end{enumerate}
\item Fie forma pătratică:
$$Q= x_1^2-2x_2^2+x_3^2+4x_1x_2-10x_1x_3+4x_2x_3$$

\begin{enumerate}
\item Aduceți forma pătratică la forma canonică prin metoda Gauss;
\item Aduceți forma pătratică la forma canonică prin transformări ortogonale;
\item Determinați forma biliniară simetrică $B$ asociată lui $Q$ și calculați dimensiunea subspațiului
$$\{x \in \mathbb{R}^3 | B(X,y)=0,\forall y \in \mathbb{R}^3\}.$$

\end{enumerate}
\end{enumerate}
\newpage
\begin{flushright}
Nume:\_\_\_\_\_\_\_\_\_\_\_\_\_\_
 
 
Grupa:\_\_\_\_\_\_\_\_\_\_\_\_\_\_
\end{flushright}
\begin{center}
\vspace{2cm}
{\Large Probleme}
\vspace{2cm}
\end{center}
\begin{enumerate}
 \item Fie morfismul $f:\mathbb{R}^5 \to \mathbb{R}^4$ al cărui matrice în raport cu bazele canonice este
$$\begin{pmatrix}
-2&-2&1&-1&-2\\
-2&1&1&-2&-1\\
-2&1&1&-2&2\\
2&2&-1&1&2
\end{pmatrix}$$

\begin{enumerate}
\item Determinați cîte o bază în $Ker(f)$ și $Im(f)$;
\item Fie vectorul $v=(1,3,1,3)$ determinați descompunerea acestuia ca suma dintre un vector din $Im(f)$ și unul din $Im(f)^\perp$;
\item Fie $K$ un corp și fie $L=M_n(K)$. Arătați că pentru orice funcțională $f \in L^*$ există o matrice $A$ astfel încît $f(X)=Tr(AX)$;
\end{enumerate}
\item Fie forma pătratică:
$$Q= 2x_1^2+5x_2^2+2x_3^2-4x_1x_2-2x_1x_3+4x_2x_3$$

\begin{enumerate}
\item Aduceți forma pătratică la forma canonică prin metoda Gauss;
\item Aduceți forma pătratică la forma canonică prin transformări ortogonale;
\item Determinați forma biliniară simetrică $B$ asociată lui $Q$ și calculați dimensiunea subspațiului
$$\{x \in \mathbb{R}^3 | B(X,y)=0,\forall y \in \mathbb{R}^3\}.$$

\end{enumerate}
\end{enumerate}
\newpage
\begin{flushright}
Nume:\_\_\_\_\_\_\_\_\_\_\_\_\_\_
 
 
Grupa:\_\_\_\_\_\_\_\_\_\_\_\_\_\_
\end{flushright}
\begin{center}
\vspace{2cm}
{\Large Probleme}
\vspace{2cm}
\end{center}
\begin{enumerate}
 \item Fie morfismul $f:\mathbb{R}^5 \to \mathbb{R}^4$ al cărui matrice în raport cu bazele canonice este
$$\begin{pmatrix}
-2&-2&1&-1&-2\\
-2&1&1&-2&-1\\
1&1&-2&0&0\\
1&-2&1&2&1
\end{pmatrix}$$

\begin{enumerate}
\item Determinați cîte o bază în $Ker(f)$ și $Im(f)$;
\item Fie vectorul $v=(1,3,1,3)$ determinați descompunerea acestuia ca suma dintre un vector din $Im(f)$ și unul din $Im(f)^\perp$;
\item Fie $K$ un corp și fie $L=M_n(K)$. Arătați că pentru orice funcțională $f \in L^*$ există o matrice $A$ astfel încît $f(X)=Tr(AX)$;
\end{enumerate}
\item Fie forma pătratică:
$$Q= -x_1^2+x_2^2-5x_3^2+3x_1x_3+4x_2x_3$$

\begin{enumerate}
\item Aduceți forma pătratică la forma canonică prin metoda Gauss;
\item Aduceți forma pătratică la forma canonică prin transformări ortogonale;
\item Determinați forma biliniară simetrică $B$ asociată lui $Q$ și calculați dimensiunea subspațiului
$$\{x \in \mathbb{R}^3 | B(X,y)=0,\forall y \in \mathbb{R}^3\}.$$

\end{enumerate}
\end{enumerate}
\newpage
\begin{flushright}
Nume:\_\_\_\_\_\_\_\_\_\_\_\_\_\_
 
 
Grupa:\_\_\_\_\_\_\_\_\_\_\_\_\_\_
\end{flushright}
\begin{center}
\vspace{2cm}
{\Large Probleme}
\vspace{2cm}
\end{center}
\begin{enumerate}
 \item Fie morfismul $f:\mathbb{R}^5 \to \mathbb{R}^4$ al cărui matrice în raport cu bazele canonice este
$$\begin{pmatrix}
-2&-2&1&-1&-2\\
-2&1&1&-2&-1\\
-1&-2&2&2&1\\
2&2&-1&1&2
\end{pmatrix}$$

\begin{enumerate}
\item Determinați cîte o bază în $Ker(f)$ și $Im(f)$;
\item Fie vectorul $v=(1,3,1,3)$ determinați descompunerea acestuia ca suma dintre un vector din $Im(f)$ și unul din $Im(f)^\perp$;
\item Fie $K$ un corp și fie $L=M_n(K)$. Arătați că pentru orice funcțională $f \in L^*$ există o matrice $A$ astfel încît $f(X)=Tr(AX)$;
\end{enumerate}
\item Fie forma pătratică:
$$Q= x_1^2+5x_2^2+x_3^2+2x_1x_2+6X_1x_3+2x_2x_3$$

\begin{enumerate}
\item Aduceți forma pătratică la forma canonică prin metoda Gauss;
\item Aduceți forma pătratică la forma canonică prin transformări ortogonale;
\item Determinați forma biliniară simetrică $B$ asociată lui $Q$ și calculați dimensiunea subspațiului
$$\{x \in \mathbb{R}^3 | B(X,y)=0,\forall y \in \mathbb{R}^3\}.$$

\end{enumerate}
\end{enumerate}
\newpage
\begin{flushright}
Nume:\_\_\_\_\_\_\_\_\_\_\_\_\_\_
 
 
Grupa:\_\_\_\_\_\_\_\_\_\_\_\_\_\_
\end{flushright}
\begin{center}
\vspace{2cm}
{\Large Probleme}
\vspace{2cm}
\end{center}
\begin{enumerate}
 \item Fie morfismul $f:\mathbb{R}^5 \to \mathbb{R}^4$ al cărui matrice în raport cu bazele canonice este
$$\begin{pmatrix}
-2&-2&1&-1&-2\\
-2&1&1&-2&-1\\
-1&1&2&-1&1\\
1&0&1&1&2
\end{pmatrix}$$

\begin{enumerate}
\item Determinați cîte o bază în $Ker(f)$ și $Im(f)$;
\item Fie vectorul $v=(1,3,1,3)$ determinați descompunerea acestuia ca suma dintre un vector din $Im(f)$ și unul din $Im(f)^\perp$;
\item Fie $K$ un corp și fie $L=M_n(K)$. Arătați că pentru orice funcțională $f \in L^*$ există o matrice $A$ astfel încît $f(X)=Tr(AX)$;
\end{enumerate}
\item Fie forma pătratică:
$$Q= x_1^2-2x_2^2+x_3^2+4x_1x_2-10x_1x_3+4x_2x_3$$

\begin{enumerate}
\item Aduceți forma pătratică la forma canonică prin metoda Gauss;
\item Aduceți forma pătratică la forma canonică prin transformări ortogonale;
\item Determinați forma biliniară simetrică $B$ asociată lui $Q$ și calculați dimensiunea subspațiului
$$\{x \in \mathbb{R}^3 | B(X,y)=0,\forall y \in \mathbb{R}^3\}.$$

\end{enumerate}
\end{enumerate}
\newpage
\begin{flushright}
Nume:\_\_\_\_\_\_\_\_\_\_\_\_\_\_
 
 
Grupa:\_\_\_\_\_\_\_\_\_\_\_\_\_\_
\end{flushright}
\begin{center}
\vspace{2cm}
{\Large Probleme}
\vspace{2cm}
\end{center}
\begin{enumerate}
 \item Fie morfismul $f:\mathbb{R}^5 \to \mathbb{R}^4$ al cărui matrice în raport cu bazele canonice este
$$\begin{pmatrix}
-2&-2&1&-1&-2\\
-2&1&1&-2&-1\\
1&2&2&-1&-2\\
-1&-2&-2&1&2
\end{pmatrix}$$

\begin{enumerate}
\item Determinați cîte o bază în $Ker(f)$ și $Im(f)$;
\item Fie vectorul $v=(1,3,1,3)$ determinați descompunerea acestuia ca suma dintre un vector din $Im(f)$ și unul din $Im(f)^\perp$;
\item Fie $K$ un corp și fie $L=M_n(K)$. Arătați că pentru orice funcțională $f \in L^*$ există o matrice $A$ astfel încît $f(X)=Tr(AX)$;
\end{enumerate}
\item Fie forma pătratică:
$$Q= 2x_1^2+5x_2^2+2x_3^2-4x_1x_2-2x_1x_3+4x_2x_3$$

\begin{enumerate}
\item Aduceți forma pătratică la forma canonică prin metoda Gauss;
\item Aduceți forma pătratică la forma canonică prin transformări ortogonale;
\item Determinați forma biliniară simetrică $B$ asociată lui $Q$ și calculați dimensiunea subspațiului
$$\{x \in \mathbb{R}^3 | B(X,y)=0,\forall y \in \mathbb{R}^3\}.$$

\end{enumerate}
\end{enumerate}
\newpage
\begin{flushright}
Nume:\_\_\_\_\_\_\_\_\_\_\_\_\_\_
 
 
Grupa:\_\_\_\_\_\_\_\_\_\_\_\_\_\_
\end{flushright}
\begin{center}
\vspace{2cm}
{\Large Probleme}
\vspace{2cm}
\end{center}
\begin{enumerate}
 \item Fie morfismul $f:\mathbb{R}^5 \to \mathbb{R}^4$ al cărui matrice în raport cu bazele canonice este
$$\begin{pmatrix}
-2&-2&1&-1&-2\\
-2&1&1&-2&-1\\
-2&0&0&2&-2\\
-2&1&1&-2&-1
\end{pmatrix}$$

\begin{enumerate}
\item Determinați cîte o bază în $Ker(f)$ și $Im(f)$;
\item Fie vectorul $v=(1,3,1,3)$ determinați descompunerea acestuia ca suma dintre un vector din $Im(f)$ și unul din $Im(f)^\perp$;
\item Fie $K$ un corp și fie $L=M_n(K)$. Arătați că pentru orice funcțională $f \in L^*$ există o matrice $A$ astfel încît $f(X)=Tr(AX)$;
\end{enumerate}
\item Fie forma pătratică:
$$Q= -x_1^2+x_2^2-5x_3^2+3x_1x_3+4x_2x_3$$

\begin{enumerate}
\item Aduceți forma pătratică la forma canonică prin metoda Gauss;
\item Aduceți forma pătratică la forma canonică prin transformări ortogonale;
\item Determinați forma biliniară simetrică $B$ asociată lui $Q$ și calculați dimensiunea subspațiului
$$\{x \in \mathbb{R}^3 | B(X,y)=0,\forall y \in \mathbb{R}^3\}.$$

\end{enumerate}
\end{enumerate}
\newpage
\begin{flushright}
Nume:\_\_\_\_\_\_\_\_\_\_\_\_\_\_
 
 
Grupa:\_\_\_\_\_\_\_\_\_\_\_\_\_\_
\end{flushright}
\begin{center}
\vspace{2cm}
{\Large Probleme}
\vspace{2cm}
\end{center}
\begin{enumerate}
 \item Fie morfismul $f:\mathbb{R}^5 \to \mathbb{R}^4$ al cărui matrice în raport cu bazele canonice este
$$\begin{pmatrix}
-2&-2&1&-1&-2\\
-2&1&1&-2&-1\\
2&0&-2&-2&0\\
2&2&-1&1&2
\end{pmatrix}$$

\begin{enumerate}
\item Determinați cîte o bază în $Ker(f)$ și $Im(f)$;
\item Fie vectorul $v=(1,3,1,3)$ determinați descompunerea acestuia ca suma dintre un vector din $Im(f)$ și unul din $Im(f)^\perp$;
\item Fie $K$ un corp și fie $L=M_n(K)$. Arătați că pentru orice funcțională $f \in L^*$ există o matrice $A$ astfel încît $f(X)=Tr(AX)$;
\end{enumerate}
\item Fie forma pătratică:
$$Q= x_1^2+5x_2^2+x_3^2+2x_1x_2+6X_1x_3+2x_2x_3$$

\begin{enumerate}
\item Aduceți forma pătratică la forma canonică prin metoda Gauss;
\item Aduceți forma pătratică la forma canonică prin transformări ortogonale;
\item Determinați forma biliniară simetrică $B$ asociată lui $Q$ și calculați dimensiunea subspațiului
$$\{x \in \mathbb{R}^3 | B(X,y)=0,\forall y \in \mathbb{R}^3\}.$$

\end{enumerate}
\end{enumerate}
\newpage
\begin{flushright}
Nume:\_\_\_\_\_\_\_\_\_\_\_\_\_\_
 
 
Grupa:\_\_\_\_\_\_\_\_\_\_\_\_\_\_
\end{flushright}
\begin{center}
\vspace{2cm}
{\Large Probleme}
\vspace{2cm}
\end{center}
\begin{enumerate}
 \item Fie morfismul $f:\mathbb{R}^5 \to \mathbb{R}^4$ al cărui matrice în raport cu bazele canonice este
$$\begin{pmatrix}
-2&-2&1&-1&-2\\
-2&1&1&-2&-1\\
-1&2&-1&1&0\\
-2&1&1&-2&-1
\end{pmatrix}$$

\begin{enumerate}
\item Determinați cîte o bază în $Ker(f)$ și $Im(f)$;
\item Fie vectorul $v=(1,3,1,3)$ determinați descompunerea acestuia ca suma dintre un vector din $Im(f)$ și unul din $Im(f)^\perp$;
\item Fie $K$ un corp și fie $L=M_n(K)$. Arătați că pentru orice funcțională $f \in L^*$ există o matrice $A$ astfel încît $f(X)=Tr(AX)$;
\end{enumerate}
\item Fie forma pătratică:
$$Q= x_1^2-2x_2^2+x_3^2+4x_1x_2-10x_1x_3+4x_2x_3$$

\begin{enumerate}
\item Aduceți forma pătratică la forma canonică prin metoda Gauss;
\item Aduceți forma pătratică la forma canonică prin transformări ortogonale;
\item Determinați forma biliniară simetrică $B$ asociată lui $Q$ și calculați dimensiunea subspațiului
$$\{x \in \mathbb{R}^3 | B(X,y)=0,\forall y \in \mathbb{R}^3\}.$$

\end{enumerate}
\end{enumerate}
\newpage
\begin{flushright}
Nume:\_\_\_\_\_\_\_\_\_\_\_\_\_\_
 
 
Grupa:\_\_\_\_\_\_\_\_\_\_\_\_\_\_
\end{flushright}
\begin{center}
\vspace{2cm}
{\Large Probleme}
\vspace{2cm}
\end{center}
\begin{enumerate}
 \item Fie morfismul $f:\mathbb{R}^5 \to \mathbb{R}^4$ al cărui matrice în raport cu bazele canonice este
$$\begin{pmatrix}
-2&-2&1&-1&-2\\
-2&1&1&-2&-1\\
2&2&-1&1&2\\
-1&-1&-1&0&1
\end{pmatrix}$$

\begin{enumerate}
\item Determinați cîte o bază în $Ker(f)$ și $Im(f)$;
\item Fie vectorul $v=(1,3,1,3)$ determinați descompunerea acestuia ca suma dintre un vector din $Im(f)$ și unul din $Im(f)^\perp$;
\item Fie $K$ un corp și fie $L=M_n(K)$. Arătați că pentru orice funcțională $f \in L^*$ există o matrice $A$ astfel încît $f(X)=Tr(AX)$;
\end{enumerate}
\item Fie forma pătratică:
$$Q= 2x_1^2+5x_2^2+2x_3^2-4x_1x_2-2x_1x_3+4x_2x_3$$

\begin{enumerate}
\item Aduceți forma pătratică la forma canonică prin metoda Gauss;
\item Aduceți forma pătratică la forma canonică prin transformări ortogonale;
\item Determinați forma biliniară simetrică $B$ asociată lui $Q$ și calculați dimensiunea subspațiului
$$\{x \in \mathbb{R}^3 | B(X,y)=0,\forall y \in \mathbb{R}^3\}.$$

\end{enumerate}
\end{enumerate}
\newpage
\begin{flushright}
Nume:\_\_\_\_\_\_\_\_\_\_\_\_\_\_
 
 
Grupa:\_\_\_\_\_\_\_\_\_\_\_\_\_\_
\end{flushright}
\begin{center}
\vspace{2cm}
{\Large Probleme}
\vspace{2cm}
\end{center}
\begin{enumerate}
 \item Fie morfismul $f:\mathbb{R}^5 \to \mathbb{R}^4$ al cărui matrice în raport cu bazele canonice este
$$\begin{pmatrix}
-2&-2&1&-1&-2\\
-2&1&1&-2&-1\\
-2&0&-1&-2&0\\
2&0&1&2&0
\end{pmatrix}$$

\begin{enumerate}
\item Determinați cîte o bază în $Ker(f)$ și $Im(f)$;
\item Fie vectorul $v=(1,3,1,3)$ determinați descompunerea acestuia ca suma dintre un vector din $Im(f)$ și unul din $Im(f)^\perp$;
\item Fie $K$ un corp și fie $L=M_n(K)$. Arătați că pentru orice funcțională $f \in L^*$ există o matrice $A$ astfel încît $f(X)=Tr(AX)$;
\end{enumerate}
\item Fie forma pătratică:
$$Q= -x_1^2+x_2^2-5x_3^2+3x_1x_3+4x_2x_3$$

\begin{enumerate}
\item Aduceți forma pătratică la forma canonică prin metoda Gauss;
\item Aduceți forma pătratică la forma canonică prin transformări ortogonale;
\item Determinați forma biliniară simetrică $B$ asociată lui $Q$ și calculați dimensiunea subspațiului
$$\{x \in \mathbb{R}^3 | B(X,y)=0,\forall y \in \mathbb{R}^3\}.$$

\end{enumerate}
\end{enumerate}
\newpage
\begin{flushright}
Nume:\_\_\_\_\_\_\_\_\_\_\_\_\_\_
 
 
Grupa:\_\_\_\_\_\_\_\_\_\_\_\_\_\_
\end{flushright}
\begin{center}
\vspace{2cm}
{\Large Probleme}
\vspace{2cm}
\end{center}
\begin{enumerate}
 \item Fie morfismul $f:\mathbb{R}^5 \to \mathbb{R}^4$ al cărui matrice în raport cu bazele canonice este
$$\begin{pmatrix}
-2&-2&1&-1&-2\\
-2&1&1&-2&-1\\
-2&-2&2&-2&0\\
-2&1&1&-2&-1
\end{pmatrix}$$

\begin{enumerate}
\item Determinați cîte o bază în $Ker(f)$ și $Im(f)$;
\item Fie vectorul $v=(1,3,1,3)$ determinați descompunerea acestuia ca suma dintre un vector din $Im(f)$ și unul din $Im(f)^\perp$;
\item Fie $K$ un corp și fie $L=M_n(K)$. Arătați că pentru orice funcțională $f \in L^*$ există o matrice $A$ astfel încît $f(X)=Tr(AX)$;
\end{enumerate}
\item Fie forma pătratică:
$$Q= x_1^2+5x_2^2+x_3^2+2x_1x_2+6X_1x_3+2x_2x_3$$

\begin{enumerate}
\item Aduceți forma pătratică la forma canonică prin metoda Gauss;
\item Aduceți forma pătratică la forma canonică prin transformări ortogonale;
\item Determinați forma biliniară simetrică $B$ asociată lui $Q$ și calculați dimensiunea subspațiului
$$\{x \in \mathbb{R}^3 | B(X,y)=0,\forall y \in \mathbb{R}^3\}.$$

\end{enumerate}
\end{enumerate}
\newpage
\begin{flushright}
Nume:\_\_\_\_\_\_\_\_\_\_\_\_\_\_
 
 
Grupa:\_\_\_\_\_\_\_\_\_\_\_\_\_\_
\end{flushright}
\begin{center}
\vspace{2cm}
{\Large Probleme}
\vspace{2cm}
\end{center}
\begin{enumerate}
 \item Fie morfismul $f:\mathbb{R}^5 \to \mathbb{R}^4$ al cărui matrice în raport cu bazele canonice este
$$\begin{pmatrix}
-2&-2&1&-1&-2\\
-2&1&1&-2&-1\\
2&2&-1&1&2\\
-2&-1&1&2&-2
\end{pmatrix}$$

\begin{enumerate}
\item Determinați cîte o bază în $Ker(f)$ și $Im(f)$;
\item Fie vectorul $v=(1,3,1,3)$ determinați descompunerea acestuia ca suma dintre un vector din $Im(f)$ și unul din $Im(f)^\perp$;
\item Fie $K$ un corp și fie $L=M_n(K)$. Arătați că pentru orice funcțională $f \in L^*$ există o matrice $A$ astfel încît $f(X)=Tr(AX)$;
\end{enumerate}
\item Fie forma pătratică:
$$Q= x_1^2-2x_2^2+x_3^2+4x_1x_2-10x_1x_3+4x_2x_3$$

\begin{enumerate}
\item Aduceți forma pătratică la forma canonică prin metoda Gauss;
\item Aduceți forma pătratică la forma canonică prin transformări ortogonale;
\item Determinați forma biliniară simetrică $B$ asociată lui $Q$ și calculați dimensiunea subspațiului
$$\{x \in \mathbb{R}^3 | B(X,y)=0,\forall y \in \mathbb{R}^3\}.$$

\end{enumerate}
\end{enumerate}
\newpage
\begin{flushright}
Nume:\_\_\_\_\_\_\_\_\_\_\_\_\_\_
 
 
Grupa:\_\_\_\_\_\_\_\_\_\_\_\_\_\_
\end{flushright}
\begin{center}
\vspace{2cm}
{\Large Probleme}
\vspace{2cm}
\end{center}
\begin{enumerate}
 \item Fie morfismul $f:\mathbb{R}^5 \to \mathbb{R}^4$ al cărui matrice în raport cu bazele canonice este
$$\begin{pmatrix}
-2&-2&1&-1&-2\\
-2&1&1&-2&-1\\
2&-1&-1&1&-1\\
0&1&0&0&1
\end{pmatrix}$$

\begin{enumerate}
\item Determinați cîte o bază în $Ker(f)$ și $Im(f)$;
\item Fie vectorul $v=(1,3,1,3)$ determinați descompunerea acestuia ca suma dintre un vector din $Im(f)$ și unul din $Im(f)^\perp$;
\item Fie $K$ un corp și fie $L=M_n(K)$. Arătați că pentru orice funcțională $f \in L^*$ există o matrice $A$ astfel încît $f(X)=Tr(AX)$;
\end{enumerate}
\item Fie forma pătratică:
$$Q= 2x_1^2+5x_2^2+2x_3^2-4x_1x_2-2x_1x_3+4x_2x_3$$

\begin{enumerate}
\item Aduceți forma pătratică la forma canonică prin metoda Gauss;
\item Aduceți forma pătratică la forma canonică prin transformări ortogonale;
\item Determinați forma biliniară simetrică $B$ asociată lui $Q$ și calculați dimensiunea subspațiului
$$\{x \in \mathbb{R}^3 | B(X,y)=0,\forall y \in \mathbb{R}^3\}.$$

\end{enumerate}
\end{enumerate}
\newpage
\begin{flushright}
Nume:\_\_\_\_\_\_\_\_\_\_\_\_\_\_
 
 
Grupa:\_\_\_\_\_\_\_\_\_\_\_\_\_\_
\end{flushright}
\begin{center}
\vspace{2cm}
{\Large Probleme}
\vspace{2cm}
\end{center}
\begin{enumerate}
 \item Fie morfismul $f:\mathbb{R}^5 \to \mathbb{R}^4$ al cărui matrice în raport cu bazele canonice este
$$\begin{pmatrix}
-2&-2&1&-1&-2\\
-2&1&1&-2&-1\\
-1&0&0&0&0\\
-2&0&0&0&0
\end{pmatrix}$$

\begin{enumerate}
\item Determinați cîte o bază în $Ker(f)$ și $Im(f)$;
\item Fie vectorul $v=(1,3,1,3)$ determinați descompunerea acestuia ca suma dintre un vector din $Im(f)$ și unul din $Im(f)^\perp$;
\item Fie $K$ un corp și fie $L=M_n(K)$. Arătați că pentru orice funcțională $f \in L^*$ există o matrice $A$ astfel încît $f(X)=Tr(AX)$;
\end{enumerate}
\item Fie forma pătratică:
$$Q= -x_1^2+x_2^2-5x_3^2+3x_1x_3+4x_2x_3$$

\begin{enumerate}
\item Aduceți forma pătratică la forma canonică prin metoda Gauss;
\item Aduceți forma pătratică la forma canonică prin transformări ortogonale;
\item Determinați forma biliniară simetrică $B$ asociată lui $Q$ și calculați dimensiunea subspațiului
$$\{x \in \mathbb{R}^3 | B(X,y)=0,\forall y \in \mathbb{R}^3\}.$$

\end{enumerate}
\end{enumerate}
\newpage
\begin{flushright}
Nume:\_\_\_\_\_\_\_\_\_\_\_\_\_\_
 
 
Grupa:\_\_\_\_\_\_\_\_\_\_\_\_\_\_
\end{flushright}
\begin{center}
\vspace{2cm}
{\Large Probleme}
\vspace{2cm}
\end{center}
\begin{enumerate}
 \item Fie morfismul $f:\mathbb{R}^5 \to \mathbb{R}^4$ al cărui matrice în raport cu bazele canonice este
$$\begin{pmatrix}
-2&-2&1&-1&-2\\
-2&1&1&-2&-1\\
-1&-2&-2&-2&1\\
2&2&-1&1&2
\end{pmatrix}$$

\begin{enumerate}
\item Determinați cîte o bază în $Ker(f)$ și $Im(f)$;
\item Fie vectorul $v=(1,3,1,3)$ determinați descompunerea acestuia ca suma dintre un vector din $Im(f)$ și unul din $Im(f)^\perp$;
\item Fie $K$ un corp și fie $L=M_n(K)$. Arătați că pentru orice funcțională $f \in L^*$ există o matrice $A$ astfel încît $f(X)=Tr(AX)$;
\end{enumerate}
\item Fie forma pătratică:
$$Q= x_1^2+5x_2^2+x_3^2+2x_1x_2+6X_1x_3+2x_2x_3$$

\begin{enumerate}
\item Aduceți forma pătratică la forma canonică prin metoda Gauss;
\item Aduceți forma pătratică la forma canonică prin transformări ortogonale;
\item Determinați forma biliniară simetrică $B$ asociată lui $Q$ și calculați dimensiunea subspațiului
$$\{x \in \mathbb{R}^3 | B(X,y)=0,\forall y \in \mathbb{R}^3\}.$$

\end{enumerate}
\end{enumerate}
\newpage
\begin{flushright}
Nume:\_\_\_\_\_\_\_\_\_\_\_\_\_\_
 
 
Grupa:\_\_\_\_\_\_\_\_\_\_\_\_\_\_
\end{flushright}
\begin{center}
\vspace{2cm}
{\Large Probleme}
\vspace{2cm}
\end{center}
\begin{enumerate}
 \item Fie morfismul $f:\mathbb{R}^5 \to \mathbb{R}^4$ al cărui matrice în raport cu bazele canonice este
$$\begin{pmatrix}
-2&-2&1&-1&-2\\
-2&1&1&-2&-1\\
0&1&1&-2&0\\
2&1&1&-2&1
\end{pmatrix}$$

\begin{enumerate}
\item Determinați cîte o bază în $Ker(f)$ și $Im(f)$;
\item Fie vectorul $v=(1,3,1,3)$ determinați descompunerea acestuia ca suma dintre un vector din $Im(f)$ și unul din $Im(f)^\perp$;
\item Fie $K$ un corp și fie $L=M_n(K)$. Arătați că pentru orice funcțională $f \in L^*$ există o matrice $A$ astfel încît $f(X)=Tr(AX)$;
\end{enumerate}
\item Fie forma pătratică:
$$Q= x_1^2-2x_2^2+x_3^2+4x_1x_2-10x_1x_3+4x_2x_3$$

\begin{enumerate}
\item Aduceți forma pătratică la forma canonică prin metoda Gauss;
\item Aduceți forma pătratică la forma canonică prin transformări ortogonale;
\item Determinați forma biliniară simetrică $B$ asociată lui $Q$ și calculați dimensiunea subspațiului
$$\{x \in \mathbb{R}^3 | B(X,y)=0,\forall y \in \mathbb{R}^3\}.$$

\end{enumerate}
\end{enumerate}
\newpage
\begin{flushright}
Nume:\_\_\_\_\_\_\_\_\_\_\_\_\_\_
 
 
Grupa:\_\_\_\_\_\_\_\_\_\_\_\_\_\_
\end{flushright}
\begin{center}
\vspace{2cm}
{\Large Probleme}
\vspace{2cm}
\end{center}
\begin{enumerate}
 \item Fie morfismul $f:\mathbb{R}^5 \to \mathbb{R}^4$ al cărui matrice în raport cu bazele canonice este
$$\begin{pmatrix}
-2&-2&1&-1&-2\\
-2&1&1&-2&-1\\
2&2&-1&1&2\\
1&2&-1&0&0
\end{pmatrix}$$

\begin{enumerate}
\item Determinați cîte o bază în $Ker(f)$ și $Im(f)$;
\item Fie vectorul $v=(1,3,1,3)$ determinați descompunerea acestuia ca suma dintre un vector din $Im(f)$ și unul din $Im(f)^\perp$;
\item Fie $K$ un corp și fie $L=M_n(K)$. Arătați că pentru orice funcțională $f \in L^*$ există o matrice $A$ astfel încît $f(X)=Tr(AX)$;
\end{enumerate}
\item Fie forma pătratică:
$$Q= 2x_1^2+5x_2^2+2x_3^2-4x_1x_2-2x_1x_3+4x_2x_3$$

\begin{enumerate}
\item Aduceți forma pătratică la forma canonică prin metoda Gauss;
\item Aduceți forma pătratică la forma canonică prin transformări ortogonale;
\item Determinați forma biliniară simetrică $B$ asociată lui $Q$ și calculați dimensiunea subspațiului
$$\{x \in \mathbb{R}^3 | B(X,y)=0,\forall y \in \mathbb{R}^3\}.$$

\end{enumerate}
\end{enumerate}
\newpage
\begin{flushright}
Nume:\_\_\_\_\_\_\_\_\_\_\_\_\_\_
 
 
Grupa:\_\_\_\_\_\_\_\_\_\_\_\_\_\_
\end{flushright}
\begin{center}
\vspace{2cm}
{\Large Probleme}
\vspace{2cm}
\end{center}
\begin{enumerate}
 \item Fie morfismul $f:\mathbb{R}^5 \to \mathbb{R}^4$ al cărui matrice în raport cu bazele canonice este
$$\begin{pmatrix}
-2&-2&1&-1&-2\\
-2&1&1&-2&-1\\
2&-1&-1&2&-2\\
-2&1&1&-2&2
\end{pmatrix}$$

\begin{enumerate}
\item Determinați cîte o bază în $Ker(f)$ și $Im(f)$;
\item Fie vectorul $v=(1,3,1,3)$ determinați descompunerea acestuia ca suma dintre un vector din $Im(f)$ și unul din $Im(f)^\perp$;
\item Fie $K$ un corp și fie $L=M_n(K)$. Arătați că pentru orice funcțională $f \in L^*$ există o matrice $A$ astfel încît $f(X)=Tr(AX)$;
\end{enumerate}
\item Fie forma pătratică:
$$Q= -x_1^2+x_2^2-5x_3^2+3x_1x_3+4x_2x_3$$

\begin{enumerate}
\item Aduceți forma pătratică la forma canonică prin metoda Gauss;
\item Aduceți forma pătratică la forma canonică prin transformări ortogonale;
\item Determinați forma biliniară simetrică $B$ asociată lui $Q$ și calculați dimensiunea subspațiului
$$\{x \in \mathbb{R}^3 | B(X,y)=0,\forall y \in \mathbb{R}^3\}.$$

\end{enumerate}
\end{enumerate}
\newpage
\begin{flushright}
Nume:\_\_\_\_\_\_\_\_\_\_\_\_\_\_
 
 
Grupa:\_\_\_\_\_\_\_\_\_\_\_\_\_\_
\end{flushright}
\begin{center}
\vspace{2cm}
{\Large Probleme}
\vspace{2cm}
\end{center}
\begin{enumerate}
 \item Fie morfismul $f:\mathbb{R}^5 \to \mathbb{R}^4$ al cărui matrice în raport cu bazele canonice este
$$\begin{pmatrix}
-2&-2&1&-1&-2\\
-2&1&1&-2&-1\\
-1&1&-1&0&-2\\
-2&1&1&-2&-1
\end{pmatrix}$$

\begin{enumerate}
\item Determinați cîte o bază în $Ker(f)$ și $Im(f)$;
\item Fie vectorul $v=(1,3,1,3)$ determinați descompunerea acestuia ca suma dintre un vector din $Im(f)$ și unul din $Im(f)^\perp$;
\item Fie $K$ un corp și fie $L=M_n(K)$. Arătați că pentru orice funcțională $f \in L^*$ există o matrice $A$ astfel încît $f(X)=Tr(AX)$;
\end{enumerate}
\item Fie forma pătratică:
$$Q= x_1^2+5x_2^2+x_3^2+2x_1x_2+6X_1x_3+2x_2x_3$$

\begin{enumerate}
\item Aduceți forma pătratică la forma canonică prin metoda Gauss;
\item Aduceți forma pătratică la forma canonică prin transformări ortogonale;
\item Determinați forma biliniară simetrică $B$ asociată lui $Q$ și calculați dimensiunea subspațiului
$$\{x \in \mathbb{R}^3 | B(X,y)=0,\forall y \in \mathbb{R}^3\}.$$

\end{enumerate}
\end{enumerate}
\newpage
\begin{flushright}
Nume:\_\_\_\_\_\_\_\_\_\_\_\_\_\_
 
 
Grupa:\_\_\_\_\_\_\_\_\_\_\_\_\_\_
\end{flushright}
\begin{center}
\vspace{2cm}
{\Large Probleme}
\vspace{2cm}
\end{center}
\begin{enumerate}
 \item Fie morfismul $f:\mathbb{R}^5 \to \mathbb{R}^4$ al cărui matrice în raport cu bazele canonice este
$$\begin{pmatrix}
-2&-2&1&-1&-2\\
-2&1&1&-2&-1\\
2&2&-1&1&2\\
0&1&2&-1&1
\end{pmatrix}$$

\begin{enumerate}
\item Determinați cîte o bază în $Ker(f)$ și $Im(f)$;
\item Fie vectorul $v=(1,3,1,3)$ determinați descompunerea acestuia ca suma dintre un vector din $Im(f)$ și unul din $Im(f)^\perp$;
\item Fie $K$ un corp și fie $L=M_n(K)$. Arătați că pentru orice funcțională $f \in L^*$ există o matrice $A$ astfel încît $f(X)=Tr(AX)$;
\end{enumerate}
\item Fie forma pătratică:
$$Q= x_1^2-2x_2^2+x_3^2+4x_1x_2-10x_1x_3+4x_2x_3$$

\begin{enumerate}
\item Aduceți forma pătratică la forma canonică prin metoda Gauss;
\item Aduceți forma pătratică la forma canonică prin transformări ortogonale;
\item Determinați forma biliniară simetrică $B$ asociată lui $Q$ și calculați dimensiunea subspațiului
$$\{x \in \mathbb{R}^3 | B(X,y)=0,\forall y \in \mathbb{R}^3\}.$$

\end{enumerate}
\end{enumerate}
\newpage
\begin{flushright}
Nume:\_\_\_\_\_\_\_\_\_\_\_\_\_\_
 
 
Grupa:\_\_\_\_\_\_\_\_\_\_\_\_\_\_
\end{flushright}
\begin{center}
\vspace{2cm}
{\Large Probleme}
\vspace{2cm}
\end{center}
\begin{enumerate}
 \item Fie morfismul $f:\mathbb{R}^5 \to \mathbb{R}^4$ al cărui matrice în raport cu bazele canonice este
$$\begin{pmatrix}
-2&-2&1&-1&-2\\
-2&1&1&-2&-1\\
0&2&2&2&0\\
0&-2&-2&-2&0
\end{pmatrix}$$

\begin{enumerate}
\item Determinați cîte o bază în $Ker(f)$ și $Im(f)$;
\item Fie vectorul $v=(1,3,1,3)$ determinați descompunerea acestuia ca suma dintre un vector din $Im(f)$ și unul din $Im(f)^\perp$;
\item Fie $K$ un corp și fie $L=M_n(K)$. Arătați că pentru orice funcțională $f \in L^*$ există o matrice $A$ astfel încît $f(X)=Tr(AX)$;
\end{enumerate}
\item Fie forma pătratică:
$$Q= 2x_1^2+5x_2^2+2x_3^2-4x_1x_2-2x_1x_3+4x_2x_3$$

\begin{enumerate}
\item Aduceți forma pătratică la forma canonică prin metoda Gauss;
\item Aduceți forma pătratică la forma canonică prin transformări ortogonale;
\item Determinați forma biliniară simetrică $B$ asociată lui $Q$ și calculați dimensiunea subspațiului
$$\{x \in \mathbb{R}^3 | B(X,y)=0,\forall y \in \mathbb{R}^3\}.$$

\end{enumerate}
\end{enumerate}
\newpage
\begin{flushright}
Nume:\_\_\_\_\_\_\_\_\_\_\_\_\_\_
 
 
Grupa:\_\_\_\_\_\_\_\_\_\_\_\_\_\_
\end{flushright}
\begin{center}
\vspace{2cm}
{\Large Probleme}
\vspace{2cm}
\end{center}
\begin{enumerate}
 \item Fie morfismul $f:\mathbb{R}^5 \to \mathbb{R}^4$ al cărui matrice în raport cu bazele canonice este
$$\begin{pmatrix}
-2&-2&1&-1&-2\\
-2&1&1&-2&-1\\
2&2&-1&1&2\\
1&2&-1&-1&0
\end{pmatrix}$$

\begin{enumerate}
\item Determinați cîte o bază în $Ker(f)$ și $Im(f)$;
\item Fie vectorul $v=(1,3,1,3)$ determinați descompunerea acestuia ca suma dintre un vector din $Im(f)$ și unul din $Im(f)^\perp$;
\item Fie $K$ un corp și fie $L=M_n(K)$. Arătați că pentru orice funcțională $f \in L^*$ există o matrice $A$ astfel încît $f(X)=Tr(AX)$;
\end{enumerate}
\item Fie forma pătratică:
$$Q= -x_1^2+x_2^2-5x_3^2+3x_1x_3+4x_2x_3$$

\begin{enumerate}
\item Aduceți forma pătratică la forma canonică prin metoda Gauss;
\item Aduceți forma pătratică la forma canonică prin transformări ortogonale;
\item Determinați forma biliniară simetrică $B$ asociată lui $Q$ și calculați dimensiunea subspațiului
$$\{x \in \mathbb{R}^3 | B(X,y)=0,\forall y \in \mathbb{R}^3\}.$$

\end{enumerate}
\end{enumerate}
\newpage
\begin{flushright}
Nume:\_\_\_\_\_\_\_\_\_\_\_\_\_\_
 
 
Grupa:\_\_\_\_\_\_\_\_\_\_\_\_\_\_
\end{flushright}
\begin{center}
\vspace{2cm}
{\Large Probleme}
\vspace{2cm}
\end{center}
\begin{enumerate}
 \item Fie morfismul $f:\mathbb{R}^5 \to \mathbb{R}^4$ al cărui matrice în raport cu bazele canonice este
$$\begin{pmatrix}
-2&-2&1&-1&-2\\
-2&1&1&-2&-1\\
2&2&-1&1&2\\
2&0&-2&1&2
\end{pmatrix}$$

\begin{enumerate}
\item Determinați cîte o bază în $Ker(f)$ și $Im(f)$;
\item Fie vectorul $v=(1,3,1,3)$ determinați descompunerea acestuia ca suma dintre un vector din $Im(f)$ și unul din $Im(f)^\perp$;
\item Fie $K$ un corp și fie $L=M_n(K)$. Arătați că pentru orice funcțională $f \in L^*$ există o matrice $A$ astfel încît $f(X)=Tr(AX)$;
\end{enumerate}
\item Fie forma pătratică:
$$Q= x_1^2+5x_2^2+x_3^2+2x_1x_2+6X_1x_3+2x_2x_3$$

\begin{enumerate}
\item Aduceți forma pătratică la forma canonică prin metoda Gauss;
\item Aduceți forma pătratică la forma canonică prin transformări ortogonale;
\item Determinați forma biliniară simetrică $B$ asociată lui $Q$ și calculați dimensiunea subspațiului
$$\{x \in \mathbb{R}^3 | B(X,y)=0,\forall y \in \mathbb{R}^3\}.$$

\end{enumerate}
\end{enumerate}
\newpage
\begin{flushright}
Nume:\_\_\_\_\_\_\_\_\_\_\_\_\_\_
 
 
Grupa:\_\_\_\_\_\_\_\_\_\_\_\_\_\_
\end{flushright}
\begin{center}
\vspace{2cm}
{\Large Probleme}
\vspace{2cm}
\end{center}
\begin{enumerate}
 \item Fie morfismul $f:\mathbb{R}^5 \to \mathbb{R}^4$ al cărui matrice în raport cu bazele canonice este
$$\begin{pmatrix}
-2&-2&1&-1&-2\\
-2&1&1&-2&-1\\
2&2&-1&1&2\\
-2&1&1&0&2
\end{pmatrix}$$

\begin{enumerate}
\item Determinați cîte o bază în $Ker(f)$ și $Im(f)$;
\item Fie vectorul $v=(1,3,1,3)$ determinați descompunerea acestuia ca suma dintre un vector din $Im(f)$ și unul din $Im(f)^\perp$;
\item Fie $K$ un corp și fie $L=M_n(K)$. Arătați că pentru orice funcțională $f \in L^*$ există o matrice $A$ astfel încît $f(X)=Tr(AX)$;
\end{enumerate}
\item Fie forma pătratică:
$$Q= x_1^2-2x_2^2+x_3^2+4x_1x_2-10x_1x_3+4x_2x_3$$

\begin{enumerate}
\item Aduceți forma pătratică la forma canonică prin metoda Gauss;
\item Aduceți forma pătratică la forma canonică prin transformări ortogonale;
\item Determinați forma biliniară simetrică $B$ asociată lui $Q$ și calculați dimensiunea subspațiului
$$\{x \in \mathbb{R}^3 | B(X,y)=0,\forall y \in \mathbb{R}^3\}.$$

\end{enumerate}
\end{enumerate}
\newpage
\begin{flushright}
Nume:\_\_\_\_\_\_\_\_\_\_\_\_\_\_
 
 
Grupa:\_\_\_\_\_\_\_\_\_\_\_\_\_\_
\end{flushright}
\begin{center}
\vspace{2cm}
{\Large Probleme}
\vspace{2cm}
\end{center}
\begin{enumerate}
 \item Fie morfismul $f:\mathbb{R}^5 \to \mathbb{R}^4$ al cărui matrice în raport cu bazele canonice este
$$\begin{pmatrix}
-2&-2&1&-1&-2\\
-2&1&1&-2&-1\\
0&0&0&0&-2\\
2&-1&-1&2&-1
\end{pmatrix}$$

\begin{enumerate}
\item Determinați cîte o bază în $Ker(f)$ și $Im(f)$;
\item Fie vectorul $v=(1,3,1,3)$ determinați descompunerea acestuia ca suma dintre un vector din $Im(f)$ și unul din $Im(f)^\perp$;
\item Fie $K$ un corp și fie $L=M_n(K)$. Arătați că pentru orice funcțională $f \in L^*$ există o matrice $A$ astfel încît $f(X)=Tr(AX)$;
\end{enumerate}
\item Fie forma pătratică:
$$Q= 2x_1^2+5x_2^2+2x_3^2-4x_1x_2-2x_1x_3+4x_2x_3$$

\begin{enumerate}
\item Aduceți forma pătratică la forma canonică prin metoda Gauss;
\item Aduceți forma pătratică la forma canonică prin transformări ortogonale;
\item Determinați forma biliniară simetrică $B$ asociată lui $Q$ și calculați dimensiunea subspațiului
$$\{x \in \mathbb{R}^3 | B(X,y)=0,\forall y \in \mathbb{R}^3\}.$$

\end{enumerate}
\end{enumerate}
\newpage
\begin{flushright}
Nume:\_\_\_\_\_\_\_\_\_\_\_\_\_\_
 
 
Grupa:\_\_\_\_\_\_\_\_\_\_\_\_\_\_
\end{flushright}
\begin{center}
\vspace{2cm}
{\Large Probleme}
\vspace{2cm}
\end{center}
\begin{enumerate}
 \item Fie morfismul $f:\mathbb{R}^5 \to \mathbb{R}^4$ al cărui matrice în raport cu bazele canonice este
$$\begin{pmatrix}
-2&-2&1&-1&-2\\
-2&1&1&-2&-1\\
2&-1&1&2&1\\
2&2&-1&1&2
\end{pmatrix}$$

\begin{enumerate}
\item Determinați cîte o bază în $Ker(f)$ și $Im(f)$;
\item Fie vectorul $v=(1,3,1,3)$ determinați descompunerea acestuia ca suma dintre un vector din $Im(f)$ și unul din $Im(f)^\perp$;
\item Fie $K$ un corp și fie $L=M_n(K)$. Arătați că pentru orice funcțională $f \in L^*$ există o matrice $A$ astfel încît $f(X)=Tr(AX)$;
\end{enumerate}
\item Fie forma pătratică:
$$Q= -x_1^2+x_2^2-5x_3^2+3x_1x_3+4x_2x_3$$

\begin{enumerate}
\item Aduceți forma pătratică la forma canonică prin metoda Gauss;
\item Aduceți forma pătratică la forma canonică prin transformări ortogonale;
\item Determinați forma biliniară simetrică $B$ asociată lui $Q$ și calculați dimensiunea subspațiului
$$\{x \in \mathbb{R}^3 | B(X,y)=0,\forall y \in \mathbb{R}^3\}.$$

\end{enumerate}
\end{enumerate}
\newpage
\begin{flushright}
Nume:\_\_\_\_\_\_\_\_\_\_\_\_\_\_
 
 
Grupa:\_\_\_\_\_\_\_\_\_\_\_\_\_\_
\end{flushright}
\begin{center}
\vspace{2cm}
{\Large Probleme}
\vspace{2cm}
\end{center}
\begin{enumerate}
 \item Fie morfismul $f:\mathbb{R}^5 \to \mathbb{R}^4$ al cărui matrice în raport cu bazele canonice este
$$\begin{pmatrix}
-2&-2&1&-1&-2\\
-2&1&1&-2&-1\\
0&1&-1&-2&1\\
2&0&-2&0&2
\end{pmatrix}$$

\begin{enumerate}
\item Determinați cîte o bază în $Ker(f)$ și $Im(f)$;
\item Fie vectorul $v=(1,3,1,3)$ determinați descompunerea acestuia ca suma dintre un vector din $Im(f)$ și unul din $Im(f)^\perp$;
\item Fie $K$ un corp și fie $L=M_n(K)$. Arătați că pentru orice funcțională $f \in L^*$ există o matrice $A$ astfel încît $f(X)=Tr(AX)$;
\end{enumerate}
\item Fie forma pătratică:
$$Q= x_1^2+5x_2^2+x_3^2+2x_1x_2+6X_1x_3+2x_2x_3$$

\begin{enumerate}
\item Aduceți forma pătratică la forma canonică prin metoda Gauss;
\item Aduceți forma pătratică la forma canonică prin transformări ortogonale;
\item Determinați forma biliniară simetrică $B$ asociată lui $Q$ și calculați dimensiunea subspațiului
$$\{x \in \mathbb{R}^3 | B(X,y)=0,\forall y \in \mathbb{R}^3\}.$$

\end{enumerate}
\end{enumerate}
\newpage
\begin{flushright}
Nume:\_\_\_\_\_\_\_\_\_\_\_\_\_\_
 
 
Grupa:\_\_\_\_\_\_\_\_\_\_\_\_\_\_
\end{flushright}
\begin{center}
\vspace{2cm}
{\Large Probleme}
\vspace{2cm}
\end{center}
\begin{enumerate}
 \item Fie morfismul $f:\mathbb{R}^5 \to \mathbb{R}^4$ al cărui matrice în raport cu bazele canonice este
$$\begin{pmatrix}
-2&-2&1&-1&-2\\
-2&1&1&-2&-1\\
-1&-1&0&0&-1\\
-2&1&-1&0&-1
\end{pmatrix}$$

\begin{enumerate}
\item Determinați cîte o bază în $Ker(f)$ și $Im(f)$;
\item Fie vectorul $v=(1,3,1,3)$ determinați descompunerea acestuia ca suma dintre un vector din $Im(f)$ și unul din $Im(f)^\perp$;
\item Fie $K$ un corp și fie $L=M_n(K)$. Arătați că pentru orice funcțională $f \in L^*$ există o matrice $A$ astfel încît $f(X)=Tr(AX)$;
\end{enumerate}
\item Fie forma pătratică:
$$Q= x_1^2-2x_2^2+x_3^2+4x_1x_2-10x_1x_3+4x_2x_3$$

\begin{enumerate}
\item Aduceți forma pătratică la forma canonică prin metoda Gauss;
\item Aduceți forma pătratică la forma canonică prin transformări ortogonale;
\item Determinați forma biliniară simetrică $B$ asociată lui $Q$ și calculați dimensiunea subspațiului
$$\{x \in \mathbb{R}^3 | B(X,y)=0,\forall y \in \mathbb{R}^3\}.$$

\end{enumerate}
\end{enumerate}
\newpage
\begin{flushright}
Nume:\_\_\_\_\_\_\_\_\_\_\_\_\_\_
 
 
Grupa:\_\_\_\_\_\_\_\_\_\_\_\_\_\_
\end{flushright}
\begin{center}
\vspace{2cm}
{\Large Probleme}
\vspace{2cm}
\end{center}
\begin{enumerate}
 \item Fie morfismul $f:\mathbb{R}^5 \to \mathbb{R}^4$ al cărui matrice în raport cu bazele canonice este
$$\begin{pmatrix}
-2&-2&1&-1&-2\\
-2&1&1&-2&-1\\
1&-2&0&0&-1\\
1&1&0&-1&0
\end{pmatrix}$$

\begin{enumerate}
\item Determinați cîte o bază în $Ker(f)$ și $Im(f)$;
\item Fie vectorul $v=(1,3,1,3)$ determinați descompunerea acestuia ca suma dintre un vector din $Im(f)$ și unul din $Im(f)^\perp$;
\item Fie $K$ un corp și fie $L=M_n(K)$. Arătați că pentru orice funcțională $f \in L^*$ există o matrice $A$ astfel încît $f(X)=Tr(AX)$;
\end{enumerate}
\item Fie forma pătratică:
$$Q= 2x_1^2+5x_2^2+2x_3^2-4x_1x_2-2x_1x_3+4x_2x_3$$

\begin{enumerate}
\item Aduceți forma pătratică la forma canonică prin metoda Gauss;
\item Aduceți forma pătratică la forma canonică prin transformări ortogonale;
\item Determinați forma biliniară simetrică $B$ asociată lui $Q$ și calculați dimensiunea subspațiului
$$\{x \in \mathbb{R}^3 | B(X,y)=0,\forall y \in \mathbb{R}^3\}.$$

\end{enumerate}
\end{enumerate}
\newpage
\begin{flushright}
Nume:\_\_\_\_\_\_\_\_\_\_\_\_\_\_
 
 
Grupa:\_\_\_\_\_\_\_\_\_\_\_\_\_\_
\end{flushright}
\begin{center}
\vspace{2cm}
{\Large Probleme}
\vspace{2cm}
\end{center}
\begin{enumerate}
 \item Fie morfismul $f:\mathbb{R}^5 \to \mathbb{R}^4$ al cărui matrice în raport cu bazele canonice este
$$\begin{pmatrix}
-2&-2&1&-1&-2\\
-2&1&1&-2&-1\\
2&2&-1&1&2\\
0&-1&1&-2&-1
\end{pmatrix}$$

\begin{enumerate}
\item Determinați cîte o bază în $Ker(f)$ și $Im(f)$;
\item Fie vectorul $v=(1,3,1,3)$ determinați descompunerea acestuia ca suma dintre un vector din $Im(f)$ și unul din $Im(f)^\perp$;
\item Fie $K$ un corp și fie $L=M_n(K)$. Arătați că pentru orice funcțională $f \in L^*$ există o matrice $A$ astfel încît $f(X)=Tr(AX)$;
\end{enumerate}
\item Fie forma pătratică:
$$Q= -x_1^2+x_2^2-5x_3^2+3x_1x_3+4x_2x_3$$

\begin{enumerate}
\item Aduceți forma pătratică la forma canonică prin metoda Gauss;
\item Aduceți forma pătratică la forma canonică prin transformări ortogonale;
\item Determinați forma biliniară simetrică $B$ asociată lui $Q$ și calculați dimensiunea subspațiului
$$\{x \in \mathbb{R}^3 | B(X,y)=0,\forall y \in \mathbb{R}^3\}.$$

\end{enumerate}
\end{enumerate}
\newpage
\begin{flushright}
Nume:\_\_\_\_\_\_\_\_\_\_\_\_\_\_
 
 
Grupa:\_\_\_\_\_\_\_\_\_\_\_\_\_\_
\end{flushright}
\begin{center}
\vspace{2cm}
{\Large Probleme}
\vspace{2cm}
\end{center}
\begin{enumerate}
 \item Fie morfismul $f:\mathbb{R}^5 \to \mathbb{R}^4$ al cărui matrice în raport cu bazele canonice este
$$\begin{pmatrix}
-2&-2&1&-1&-2\\
-2&1&1&-2&-1\\
2&2&-1&1&2\\
-2&1&-2&-2&0
\end{pmatrix}$$

\begin{enumerate}
\item Determinați cîte o bază în $Ker(f)$ și $Im(f)$;
\item Fie vectorul $v=(1,3,1,3)$ determinați descompunerea acestuia ca suma dintre un vector din $Im(f)$ și unul din $Im(f)^\perp$;
\item Fie $K$ un corp și fie $L=M_n(K)$. Arătați că pentru orice funcțională $f \in L^*$ există o matrice $A$ astfel încît $f(X)=Tr(AX)$;
\end{enumerate}
\item Fie forma pătratică:
$$Q= x_1^2+5x_2^2+x_3^2+2x_1x_2+6X_1x_3+2x_2x_3$$

\begin{enumerate}
\item Aduceți forma pătratică la forma canonică prin metoda Gauss;
\item Aduceți forma pătratică la forma canonică prin transformări ortogonale;
\item Determinați forma biliniară simetrică $B$ asociată lui $Q$ și calculați dimensiunea subspațiului
$$\{x \in \mathbb{R}^3 | B(X,y)=0,\forall y \in \mathbb{R}^3\}.$$

\end{enumerate}
\end{enumerate}
\newpage
\begin{flushright}
Nume:\_\_\_\_\_\_\_\_\_\_\_\_\_\_
 
 
Grupa:\_\_\_\_\_\_\_\_\_\_\_\_\_\_
\end{flushright}
\begin{center}
\vspace{2cm}
{\Large Probleme}
\vspace{2cm}
\end{center}
\begin{enumerate}
 \item Fie morfismul $f:\mathbb{R}^5 \to \mathbb{R}^4$ al cărui matrice în raport cu bazele canonice este
$$\begin{pmatrix}
-2&-2&1&-1&-2\\
-2&1&1&-2&-1\\
-2&-1&-1&0&-2\\
2&-2&1&1&1
\end{pmatrix}$$

\begin{enumerate}
\item Determinați cîte o bază în $Ker(f)$ și $Im(f)$;
\item Fie vectorul $v=(1,3,1,3)$ determinați descompunerea acestuia ca suma dintre un vector din $Im(f)$ și unul din $Im(f)^\perp$;
\item Fie $K$ un corp și fie $L=M_n(K)$. Arătați că pentru orice funcțională $f \in L^*$ există o matrice $A$ astfel încît $f(X)=Tr(AX)$;
\end{enumerate}
\item Fie forma pătratică:
$$Q= x_1^2-2x_2^2+x_3^2+4x_1x_2-10x_1x_3+4x_2x_3$$

\begin{enumerate}
\item Aduceți forma pătratică la forma canonică prin metoda Gauss;
\item Aduceți forma pătratică la forma canonică prin transformări ortogonale;
\item Determinați forma biliniară simetrică $B$ asociată lui $Q$ și calculați dimensiunea subspațiului
$$\{x \in \mathbb{R}^3 | B(X,y)=0,\forall y \in \mathbb{R}^3\}.$$

\end{enumerate}
\end{enumerate}
\newpage
\begin{flushright}
Nume:\_\_\_\_\_\_\_\_\_\_\_\_\_\_
 
 
Grupa:\_\_\_\_\_\_\_\_\_\_\_\_\_\_
\end{flushright}
\begin{center}
\vspace{2cm}
{\Large Probleme}
\vspace{2cm}
\end{center}
\begin{enumerate}
 \item Fie morfismul $f:\mathbb{R}^5 \to \mathbb{R}^4$ al cărui matrice în raport cu bazele canonice este
$$\begin{pmatrix}
-2&-2&1&-1&-2\\
-2&1&1&-2&-1\\
2&2&-1&1&2\\
2&-1&-2&-2&2
\end{pmatrix}$$

\begin{enumerate}
\item Determinați cîte o bază în $Ker(f)$ și $Im(f)$;
\item Fie vectorul $v=(1,3,1,3)$ determinați descompunerea acestuia ca suma dintre un vector din $Im(f)$ și unul din $Im(f)^\perp$;
\item Fie $K$ un corp și fie $L=M_n(K)$. Arătați că pentru orice funcțională $f \in L^*$ există o matrice $A$ astfel încît $f(X)=Tr(AX)$;
\end{enumerate}
\item Fie forma pătratică:
$$Q= 2x_1^2+5x_2^2+2x_3^2-4x_1x_2-2x_1x_3+4x_2x_3$$

\begin{enumerate}
\item Aduceți forma pătratică la forma canonică prin metoda Gauss;
\item Aduceți forma pătratică la forma canonică prin transformări ortogonale;
\item Determinați forma biliniară simetrică $B$ asociată lui $Q$ și calculați dimensiunea subspațiului
$$\{x \in \mathbb{R}^3 | B(X,y)=0,\forall y \in \mathbb{R}^3\}.$$

\end{enumerate}
\end{enumerate}
\newpage
\begin{flushright}
Nume:\_\_\_\_\_\_\_\_\_\_\_\_\_\_
 
 
Grupa:\_\_\_\_\_\_\_\_\_\_\_\_\_\_
\end{flushright}
\begin{center}
\vspace{2cm}
{\Large Probleme}
\vspace{2cm}
\end{center}
\begin{enumerate}
 \item Fie morfismul $f:\mathbb{R}^5 \to \mathbb{R}^4$ al cărui matrice în raport cu bazele canonice este
$$\begin{pmatrix}
-2&-2&1&-1&-2\\
-2&1&1&-2&-1\\
2&2&-1&1&2\\
-2&1&1&-1&1
\end{pmatrix}$$

\begin{enumerate}
\item Determinați cîte o bază în $Ker(f)$ și $Im(f)$;
\item Fie vectorul $v=(1,3,1,3)$ determinați descompunerea acestuia ca suma dintre un vector din $Im(f)$ și unul din $Im(f)^\perp$;
\item Fie $K$ un corp și fie $L=M_n(K)$. Arătați că pentru orice funcțională $f \in L^*$ există o matrice $A$ astfel încît $f(X)=Tr(AX)$;
\end{enumerate}
\item Fie forma pătratică:
$$Q= -x_1^2+x_2^2-5x_3^2+3x_1x_3+4x_2x_3$$

\begin{enumerate}
\item Aduceți forma pătratică la forma canonică prin metoda Gauss;
\item Aduceți forma pătratică la forma canonică prin transformări ortogonale;
\item Determinați forma biliniară simetrică $B$ asociată lui $Q$ și calculați dimensiunea subspațiului
$$\{x \in \mathbb{R}^3 | B(X,y)=0,\forall y \in \mathbb{R}^3\}.$$

\end{enumerate}
\end{enumerate}
\newpage
\begin{flushright}
Nume:\_\_\_\_\_\_\_\_\_\_\_\_\_\_
 
 
Grupa:\_\_\_\_\_\_\_\_\_\_\_\_\_\_
\end{flushright}
\begin{center}
\vspace{2cm}
{\Large Probleme}
\vspace{2cm}
\end{center}
\begin{enumerate}
 \item Fie morfismul $f:\mathbb{R}^5 \to \mathbb{R}^4$ al cărui matrice în raport cu bazele canonice este
$$\begin{pmatrix}
-2&-2&1&-1&-2\\
-2&1&1&-2&-1\\
0&2&1&0&1\\
2&2&-1&1&2
\end{pmatrix}$$

\begin{enumerate}
\item Determinați cîte o bază în $Ker(f)$ și $Im(f)$;
\item Fie vectorul $v=(1,3,1,3)$ determinați descompunerea acestuia ca suma dintre un vector din $Im(f)$ și unul din $Im(f)^\perp$;
\item Fie $K$ un corp și fie $L=M_n(K)$. Arătați că pentru orice funcțională $f \in L^*$ există o matrice $A$ astfel încît $f(X)=Tr(AX)$;
\end{enumerate}
\item Fie forma pătratică:
$$Q= x_1^2+5x_2^2+x_3^2+2x_1x_2+6X_1x_3+2x_2x_3$$

\begin{enumerate}
\item Aduceți forma pătratică la forma canonică prin metoda Gauss;
\item Aduceți forma pătratică la forma canonică prin transformări ortogonale;
\item Determinați forma biliniară simetrică $B$ asociată lui $Q$ și calculați dimensiunea subspațiului
$$\{x \in \mathbb{R}^3 | B(X,y)=0,\forall y \in \mathbb{R}^3\}.$$

\end{enumerate}
\end{enumerate}
\newpage
\begin{flushright}
Nume:\_\_\_\_\_\_\_\_\_\_\_\_\_\_
 
 
Grupa:\_\_\_\_\_\_\_\_\_\_\_\_\_\_
\end{flushright}
\begin{center}
\vspace{2cm}
{\Large Probleme}
\vspace{2cm}
\end{center}
\begin{enumerate}
 \item Fie morfismul $f:\mathbb{R}^5 \to \mathbb{R}^4$ al cărui matrice în raport cu bazele canonice este
$$\begin{pmatrix}
-2&-2&1&-1&-2\\
-2&1&1&-2&-1\\
2&2&1&-1&2\\
-2&-2&2&-2&-2
\end{pmatrix}$$

\begin{enumerate}
\item Determinați cîte o bază în $Ker(f)$ și $Im(f)$;
\item Fie vectorul $v=(1,3,1,3)$ determinați descompunerea acestuia ca suma dintre un vector din $Im(f)$ și unul din $Im(f)^\perp$;
\item Fie $K$ un corp și fie $L=M_n(K)$. Arătați că pentru orice funcțională $f \in L^*$ există o matrice $A$ astfel încît $f(X)=Tr(AX)$;
\end{enumerate}
\item Fie forma pătratică:
$$Q= x_1^2-2x_2^2+x_3^2+4x_1x_2-10x_1x_3+4x_2x_3$$

\begin{enumerate}
\item Aduceți forma pătratică la forma canonică prin metoda Gauss;
\item Aduceți forma pătratică la forma canonică prin transformări ortogonale;
\item Determinați forma biliniară simetrică $B$ asociată lui $Q$ și calculați dimensiunea subspațiului
$$\{x \in \mathbb{R}^3 | B(X,y)=0,\forall y \in \mathbb{R}^3\}.$$

\end{enumerate}
\end{enumerate}
\newpage
\begin{flushright}
Nume:\_\_\_\_\_\_\_\_\_\_\_\_\_\_
 
 
Grupa:\_\_\_\_\_\_\_\_\_\_\_\_\_\_
\end{flushright}
\begin{center}
\vspace{2cm}
{\Large Probleme}
\vspace{2cm}
\end{center}
\begin{enumerate}
 \item Fie morfismul $f:\mathbb{R}^5 \to \mathbb{R}^4$ al cărui matrice în raport cu bazele canonice este
$$\begin{pmatrix}
-2&-2&1&-1&-2\\
-2&1&1&-2&-1\\
-1&2&2&2&1\\
-2&1&1&-2&-1
\end{pmatrix}$$

\begin{enumerate}
\item Determinați cîte o bază în $Ker(f)$ și $Im(f)$;
\item Fie vectorul $v=(1,3,1,3)$ determinați descompunerea acestuia ca suma dintre un vector din $Im(f)$ și unul din $Im(f)^\perp$;
\item Fie $K$ un corp și fie $L=M_n(K)$. Arătați că pentru orice funcțională $f \in L^*$ există o matrice $A$ astfel încît $f(X)=Tr(AX)$;
\end{enumerate}
\item Fie forma pătratică:
$$Q= 2x_1^2+5x_2^2+2x_3^2-4x_1x_2-2x_1x_3+4x_2x_3$$

\begin{enumerate}
\item Aduceți forma pătratică la forma canonică prin metoda Gauss;
\item Aduceți forma pătratică la forma canonică prin transformări ortogonale;
\item Determinați forma biliniară simetrică $B$ asociată lui $Q$ și calculați dimensiunea subspațiului
$$\{x \in \mathbb{R}^3 | B(X,y)=0,\forall y \in \mathbb{R}^3\}.$$

\end{enumerate}
\end{enumerate}
\newpage
\begin{flushright}
Nume:\_\_\_\_\_\_\_\_\_\_\_\_\_\_
 
 
Grupa:\_\_\_\_\_\_\_\_\_\_\_\_\_\_
\end{flushright}
\begin{center}
\vspace{2cm}
{\Large Probleme}
\vspace{2cm}
\end{center}
\begin{enumerate}
 \item Fie morfismul $f:\mathbb{R}^5 \to \mathbb{R}^4$ al cărui matrice în raport cu bazele canonice este
$$\begin{pmatrix}
-2&-2&1&-1&-2\\
-2&1&1&-2&-1\\
2&2&-1&1&2\\
1&1&-1&-2&2
\end{pmatrix}$$

\begin{enumerate}
\item Determinați cîte o bază în $Ker(f)$ și $Im(f)$;
\item Fie vectorul $v=(1,3,1,3)$ determinați descompunerea acestuia ca suma dintre un vector din $Im(f)$ și unul din $Im(f)^\perp$;
\item Fie $K$ un corp și fie $L=M_n(K)$. Arătați că pentru orice funcțională $f \in L^*$ există o matrice $A$ astfel încît $f(X)=Tr(AX)$;
\end{enumerate}
\item Fie forma pătratică:
$$Q= -x_1^2+x_2^2-5x_3^2+3x_1x_3+4x_2x_3$$

\begin{enumerate}
\item Aduceți forma pătratică la forma canonică prin metoda Gauss;
\item Aduceți forma pătratică la forma canonică prin transformări ortogonale;
\item Determinați forma biliniară simetrică $B$ asociată lui $Q$ și calculați dimensiunea subspațiului
$$\{x \in \mathbb{R}^3 | B(X,y)=0,\forall y \in \mathbb{R}^3\}.$$

\end{enumerate}
\end{enumerate}
\newpage
\begin{flushright}
Nume:\_\_\_\_\_\_\_\_\_\_\_\_\_\_
 
 
Grupa:\_\_\_\_\_\_\_\_\_\_\_\_\_\_
\end{flushright}
\begin{center}
\vspace{2cm}
{\Large Probleme}
\vspace{2cm}
\end{center}
\begin{enumerate}
 \item Fie morfismul $f:\mathbb{R}^5 \to \mathbb{R}^4$ al cărui matrice în raport cu bazele canonice este
$$\begin{pmatrix}
-2&-2&1&-1&-2\\
-2&1&1&-2&-1\\
1&0&1&1&2\\
-1&-2&2&0&0
\end{pmatrix}$$

\begin{enumerate}
\item Determinați cîte o bază în $Ker(f)$ și $Im(f)$;
\item Fie vectorul $v=(1,3,1,3)$ determinați descompunerea acestuia ca suma dintre un vector din $Im(f)$ și unul din $Im(f)^\perp$;
\item Fie $K$ un corp și fie $L=M_n(K)$. Arătați că pentru orice funcțională $f \in L^*$ există o matrice $A$ astfel încît $f(X)=Tr(AX)$;
\end{enumerate}
\item Fie forma pătratică:
$$Q= x_1^2+5x_2^2+x_3^2+2x_1x_2+6X_1x_3+2x_2x_3$$

\begin{enumerate}
\item Aduceți forma pătratică la forma canonică prin metoda Gauss;
\item Aduceți forma pătratică la forma canonică prin transformări ortogonale;
\item Determinați forma biliniară simetrică $B$ asociată lui $Q$ și calculați dimensiunea subspațiului
$$\{x \in \mathbb{R}^3 | B(X,y)=0,\forall y \in \mathbb{R}^3\}.$$

\end{enumerate}
\end{enumerate}
\newpage
\begin{flushright}
Nume:\_\_\_\_\_\_\_\_\_\_\_\_\_\_
 
 
Grupa:\_\_\_\_\_\_\_\_\_\_\_\_\_\_
\end{flushright}
\begin{center}
\vspace{2cm}
{\Large Probleme}
\vspace{2cm}
\end{center}
\begin{enumerate}
 \item Fie morfismul $f:\mathbb{R}^5 \to \mathbb{R}^4$ al cărui matrice în raport cu bazele canonice este
$$\begin{pmatrix}
-2&-2&1&-1&-2\\
-2&1&1&-2&-1\\
0&-1&1&1&2\\
2&-1&-1&2&1
\end{pmatrix}$$

\begin{enumerate}
\item Determinați cîte o bază în $Ker(f)$ și $Im(f)$;
\item Fie vectorul $v=(1,3,1,3)$ determinați descompunerea acestuia ca suma dintre un vector din $Im(f)$ și unul din $Im(f)^\perp$;
\item Fie $K$ un corp și fie $L=M_n(K)$. Arătați că pentru orice funcțională $f \in L^*$ există o matrice $A$ astfel încît $f(X)=Tr(AX)$;
\end{enumerate}
\item Fie forma pătratică:
$$Q= x_1^2-2x_2^2+x_3^2+4x_1x_2-10x_1x_3+4x_2x_3$$

\begin{enumerate}
\item Aduceți forma pătratică la forma canonică prin metoda Gauss;
\item Aduceți forma pătratică la forma canonică prin transformări ortogonale;
\item Determinați forma biliniară simetrică $B$ asociată lui $Q$ și calculați dimensiunea subspațiului
$$\{x \in \mathbb{R}^3 | B(X,y)=0,\forall y \in \mathbb{R}^3\}.$$

\end{enumerate}
\end{enumerate}
\newpage
\begin{flushright}
Nume:\_\_\_\_\_\_\_\_\_\_\_\_\_\_
 
 
Grupa:\_\_\_\_\_\_\_\_\_\_\_\_\_\_
\end{flushright}
\begin{center}
\vspace{2cm}
{\Large Probleme}
\vspace{2cm}
\end{center}
\begin{enumerate}
 \item Fie morfismul $f:\mathbb{R}^5 \to \mathbb{R}^4$ al cărui matrice în raport cu bazele canonice este
$$\begin{pmatrix}
-2&-2&1&-1&-2\\
-2&1&1&-2&-1\\
0&-1&0&2&-2\\
2&2&-1&1&2
\end{pmatrix}$$

\begin{enumerate}
\item Determinați cîte o bază în $Ker(f)$ și $Im(f)$;
\item Fie vectorul $v=(1,3,1,3)$ determinați descompunerea acestuia ca suma dintre un vector din $Im(f)$ și unul din $Im(f)^\perp$;
\item Fie $K$ un corp și fie $L=M_n(K)$. Arătați că pentru orice funcțională $f \in L^*$ există o matrice $A$ astfel încît $f(X)=Tr(AX)$;
\end{enumerate}
\item Fie forma pătratică:
$$Q= 2x_1^2+5x_2^2+2x_3^2-4x_1x_2-2x_1x_3+4x_2x_3$$

\begin{enumerate}
\item Aduceți forma pătratică la forma canonică prin metoda Gauss;
\item Aduceți forma pătratică la forma canonică prin transformări ortogonale;
\item Determinați forma biliniară simetrică $B$ asociată lui $Q$ și calculați dimensiunea subspațiului
$$\{x \in \mathbb{R}^3 | B(X,y)=0,\forall y \in \mathbb{R}^3\}.$$

\end{enumerate}
\end{enumerate}
\newpage
\begin{flushright}
Nume:\_\_\_\_\_\_\_\_\_\_\_\_\_\_
 
 
Grupa:\_\_\_\_\_\_\_\_\_\_\_\_\_\_
\end{flushright}
\begin{center}
\vspace{2cm}
{\Large Probleme}
\vspace{2cm}
\end{center}
\begin{enumerate}
 \item Fie morfismul $f:\mathbb{R}^5 \to \mathbb{R}^4$ al cărui matrice în raport cu bazele canonice este
$$\begin{pmatrix}
-2&-2&1&-1&-2\\
-2&1&1&-2&-1\\
2&2&-1&1&2\\
1&1&-2&-1&2
\end{pmatrix}$$

\begin{enumerate}
\item Determinați cîte o bază în $Ker(f)$ și $Im(f)$;
\item Fie vectorul $v=(1,3,1,3)$ determinați descompunerea acestuia ca suma dintre un vector din $Im(f)$ și unul din $Im(f)^\perp$;
\item Fie $K$ un corp și fie $L=M_n(K)$. Arătați că pentru orice funcțională $f \in L^*$ există o matrice $A$ astfel încît $f(X)=Tr(AX)$;
\end{enumerate}
\item Fie forma pătratică:
$$Q= -x_1^2+x_2^2-5x_3^2+3x_1x_3+4x_2x_3$$

\begin{enumerate}
\item Aduceți forma pătratică la forma canonică prin metoda Gauss;
\item Aduceți forma pătratică la forma canonică prin transformări ortogonale;
\item Determinați forma biliniară simetrică $B$ asociată lui $Q$ și calculați dimensiunea subspațiului
$$\{x \in \mathbb{R}^3 | B(X,y)=0,\forall y \in \mathbb{R}^3\}.$$

\end{enumerate}
\end{enumerate}
\newpage
\begin{flushright}
Nume:\_\_\_\_\_\_\_\_\_\_\_\_\_\_
 
 
Grupa:\_\_\_\_\_\_\_\_\_\_\_\_\_\_
\end{flushright}
\begin{center}
\vspace{2cm}
{\Large Probleme}
\vspace{2cm}
\end{center}
\begin{enumerate}
 \item Fie morfismul $f:\mathbb{R}^5 \to \mathbb{R}^4$ al cărui matrice în raport cu bazele canonice este
$$\begin{pmatrix}
-2&-2&1&-1&-2\\
-2&1&1&-2&-1\\
2&2&-1&1&2\\
2&2&-2&1&2
\end{pmatrix}$$

\begin{enumerate}
\item Determinați cîte o bază în $Ker(f)$ și $Im(f)$;
\item Fie vectorul $v=(1,3,1,3)$ determinați descompunerea acestuia ca suma dintre un vector din $Im(f)$ și unul din $Im(f)^\perp$;
\item Fie $K$ un corp și fie $L=M_n(K)$. Arătați că pentru orice funcțională $f \in L^*$ există o matrice $A$ astfel încît $f(X)=Tr(AX)$;
\end{enumerate}
\item Fie forma pătratică:
$$Q= x_1^2+5x_2^2+x_3^2+2x_1x_2+6X_1x_3+2x_2x_3$$

\begin{enumerate}
\item Aduceți forma pătratică la forma canonică prin metoda Gauss;
\item Aduceți forma pătratică la forma canonică prin transformări ortogonale;
\item Determinați forma biliniară simetrică $B$ asociată lui $Q$ și calculați dimensiunea subspațiului
$$\{x \in \mathbb{R}^3 | B(X,y)=0,\forall y \in \mathbb{R}^3\}.$$

\end{enumerate}
\end{enumerate}
\newpage
\begin{flushright}
Nume:\_\_\_\_\_\_\_\_\_\_\_\_\_\_
 
 
Grupa:\_\_\_\_\_\_\_\_\_\_\_\_\_\_
\end{flushright}
\begin{center}
\vspace{2cm}
{\Large Probleme}
\vspace{2cm}
\end{center}
\begin{enumerate}
 \item Fie morfismul $f:\mathbb{R}^5 \to \mathbb{R}^4$ al cărui matrice în raport cu bazele canonice este
$$\begin{pmatrix}
-2&-2&1&-1&-2\\
-2&1&1&-2&-1\\
-2&2&2&-1&-1\\
-2&-1&2&0&-2
\end{pmatrix}$$

\begin{enumerate}
\item Determinați cîte o bază în $Ker(f)$ și $Im(f)$;
\item Fie vectorul $v=(1,3,1,3)$ determinați descompunerea acestuia ca suma dintre un vector din $Im(f)$ și unul din $Im(f)^\perp$;
\item Fie $K$ un corp și fie $L=M_n(K)$. Arătați că pentru orice funcțională $f \in L^*$ există o matrice $A$ astfel încît $f(X)=Tr(AX)$;
\end{enumerate}
\item Fie forma pătratică:
$$Q= x_1^2-2x_2^2+x_3^2+4x_1x_2-10x_1x_3+4x_2x_3$$

\begin{enumerate}
\item Aduceți forma pătratică la forma canonică prin metoda Gauss;
\item Aduceți forma pătratică la forma canonică prin transformări ortogonale;
\item Determinați forma biliniară simetrică $B$ asociată lui $Q$ și calculați dimensiunea subspațiului
$$\{x \in \mathbb{R}^3 | B(X,y)=0,\forall y \in \mathbb{R}^3\}.$$

\end{enumerate}
\end{enumerate}
\newpage
\begin{flushright}
Nume:\_\_\_\_\_\_\_\_\_\_\_\_\_\_
 
 
Grupa:\_\_\_\_\_\_\_\_\_\_\_\_\_\_
\end{flushright}
\begin{center}
\vspace{2cm}
{\Large Probleme}
\vspace{2cm}
\end{center}
\begin{enumerate}
 \item Fie morfismul $f:\mathbb{R}^5 \to \mathbb{R}^4$ al cărui matrice în raport cu bazele canonice este
$$\begin{pmatrix}
-2&-2&1&-1&-2\\
-2&1&1&-2&-1\\
2&2&-1&1&2\\
0&1&2&2&2
\end{pmatrix}$$

\begin{enumerate}
\item Determinați cîte o bază în $Ker(f)$ și $Im(f)$;
\item Fie vectorul $v=(1,3,1,3)$ determinați descompunerea acestuia ca suma dintre un vector din $Im(f)$ și unul din $Im(f)^\perp$;
\item Fie $K$ un corp și fie $L=M_n(K)$. Arătați că pentru orice funcțională $f \in L^*$ există o matrice $A$ astfel încît $f(X)=Tr(AX)$;
\end{enumerate}
\item Fie forma pătratică:
$$Q= 2x_1^2+5x_2^2+2x_3^2-4x_1x_2-2x_1x_3+4x_2x_3$$

\begin{enumerate}
\item Aduceți forma pătratică la forma canonică prin metoda Gauss;
\item Aduceți forma pătratică la forma canonică prin transformări ortogonale;
\item Determinați forma biliniară simetrică $B$ asociată lui $Q$ și calculați dimensiunea subspațiului
$$\{x \in \mathbb{R}^3 | B(X,y)=0,\forall y \in \mathbb{R}^3\}.$$

\end{enumerate}
\end{enumerate}
\newpage
\begin{flushright}
Nume:\_\_\_\_\_\_\_\_\_\_\_\_\_\_
 
 
Grupa:\_\_\_\_\_\_\_\_\_\_\_\_\_\_
\end{flushright}
\begin{center}
\vspace{2cm}
{\Large Probleme}
\vspace{2cm}
\end{center}
\begin{enumerate}
 \item Fie morfismul $f:\mathbb{R}^5 \to \mathbb{R}^4$ al cărui matrice în raport cu bazele canonice este
$$\begin{pmatrix}
-2&-2&1&-1&-2\\
-2&1&1&-2&-1\\
0&-2&-2&-2&1\\
-2&1&1&-2&-1
\end{pmatrix}$$

\begin{enumerate}
\item Determinați cîte o bază în $Ker(f)$ și $Im(f)$;
\item Fie vectorul $v=(1,3,1,3)$ determinați descompunerea acestuia ca suma dintre un vector din $Im(f)$ și unul din $Im(f)^\perp$;
\item Fie $K$ un corp și fie $L=M_n(K)$. Arătați că pentru orice funcțională $f \in L^*$ există o matrice $A$ astfel încît $f(X)=Tr(AX)$;
\end{enumerate}
\item Fie forma pătratică:
$$Q= -x_1^2+x_2^2-5x_3^2+3x_1x_3+4x_2x_3$$

\begin{enumerate}
\item Aduceți forma pătratică la forma canonică prin metoda Gauss;
\item Aduceți forma pătratică la forma canonică prin transformări ortogonale;
\item Determinați forma biliniară simetrică $B$ asociată lui $Q$ și calculați dimensiunea subspațiului
$$\{x \in \mathbb{R}^3 | B(X,y)=0,\forall y \in \mathbb{R}^3\}.$$

\end{enumerate}
\end{enumerate}
\newpage
\end{document}
